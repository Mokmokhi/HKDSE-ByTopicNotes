\documentclass[12pt]{article}
\usepackage{ctex}
\usepackage[english]{babel}
\usepackage{blindtext}
\usepackage{nameref}
\usepackage{fancyhdr}
\usepackage{color,amsmath,amssymb,amsthm,physics}
\usepackage{graphicx,float}
\usepackage{physics}
\usepackage{pgfplots}
\usepackage[a4paper, total={7in, 9in}]{geometry}
\usepackage{multicol}
\usepackage{framed}
\usepackage{xcolor}

\graphicspath{ {../images/} }

\definecolor{shadecolor}{RGB}{220,220,220}

\pagestyle{fancy}
\fancyhf{}
\fancyhf[HL]{A short course on Expansion}
\fancyhf[HR]{\rightmark}
\fancyhf[CF]{\thepage}
\fancyhf[FL]{\copyright Mok Owen 2024}

\newcommand{\innerprod}[2]{\langle{#1},{#2}\rangle}
\newcommand{\id}{\mathtt{id}}
\newcommand{\cis}[1]{\mathrm{cis}({#1})}

\newtheorem*{definition}{Definition}
\newtheorem*{theorem}{Theorem}
\newtheorem*{corollary}{Corollary}
\newtheorem*{lemma}{Lemma}
\newtheorem*{proposition}{Proposition}
\newtheorem*{remark}{Remark}
\newtheorem*{claim}{Claim}
\newtheorem*{example}{Example}
\newtheorem*{axiom}{Axiom}

\newtheorem*{exercise}{Try}
\newenvironment{solution}{\begin{snugshade*} \underline{\textbf{Solution.}} \par}{\hfill \textit{\dots end of solution} \end{snugshade*}}
\renewenvironment{proof}[1][Proof]{\begin{snugshade*} \underline{\textit{{#1}.}}\\}{\hfill \qedsymbol \end{snugshade*}}

\begin{document}
    \begin{abstract}
        Usually we discuss about trigonometry with geometry sense, however, it would be quite interesting to discuss trigonometry with series and calculus. We shall call it mathematical analysis.
    \end{abstract}

    \section*{Defining trigonometric functions}

    Let $\mathbb{C}$ be the set of complex numbers, and $V=(\mathbb{C}^n,+,\cdot)$ be the $n$-dimensional complex vector space, where $+:\mathbb{C}\times\mathbb{C}\to\mathbb{C}$ and $\cdot:\mathbb{C}\times\mathbb{C}\to\mathbb{C}$ are component-wise addition and multiplication respectively. Define a metric on $V$ by $d(z,w):=\norm*{z-w}$ implicitly as a metric induced by norm, so that a metric (open) ball of radius $\varepsilon\geq 0$ centered at $z_0$, in the universal meaning, is the set \[B(z_0,\varepsilon):=\{z\in V\mid d(z,z_0)<\varepsilon\}.\] A metric sphere of radius $\varepsilon\geq 0$ centered at $z_0$ is the set \[S(z_0,\varepsilon):=\{z\in V\mid d(z,z_0)=\varepsilon\}.\] A metric (closed) ball of radius $\varepsilon\geq 0$ centered at $z_0$ is the set \[\overline{B}(z_0,\varepsilon):=\{z\in V\mid d(z,z_0)\leq\varepsilon\}.\]
    
    Recall some of the meaning:

    \begin{definition}[Projection]
        A \textbf{projection} is a function $P:\mathbb{C}^n\to\mathbb{C}^m\times \mathbb{C}^{n-m}$, where $m\leq n$, such that $P\circ P = P$. We will denote $P^k$ to be the $k$-th composition $P^k=P\circ P^{k-1}$. In particular, $P^k=P$ for $k\in\mathbb{N}$.
    \end{definition}

    \begin{definition}[The Sine Function]
        Define the \textbf{pseudo-sine function} $PS_m:\mathbb{C}^n\to\mathbb{C}^m\times \mathbb{C}^{n-m}$ be a projection from $\mathbb{C}^n$ to $\mathbb{C}^m\times\{0\}^{n-m}$. In particular, for $z=(z_1,z_2,\dots,z_m,z_{m+1},\dots,z_n)\in V$,  \[PS_m(z):=(z_1,z_2,\dots,z_m,0,\dots,0).\]
    \end{definition}

    It is not hard to check the pseudo-sine function is indeed a projection on $\mathbb{C}^m$, and we may observe that:

    \begin{proposition}
        The pseudo-sine function satisfies the following properties:
        \begin{itemize}
            \item Given the component-wise $PS_m(z+w)=PS_m(z)+PS_m(w)$.
            \item $PS_m(zw)=PS_m(z)PS_m(w)$.
        \end{itemize}
    \end{proposition}

    \begin{proof}
        Both equality follows from the component-wise operation on $V$.
    \end{proof}

    The inner product on $V$ can be constructed from the usual sense of complex group. We may define $\innerprod{\cdot}{\cdot}:V\times V\to\mathbb{R}$ by the mapping \[\innerprod{x}{y}=\sum x_i \overline{y}_i\]
\end{document}