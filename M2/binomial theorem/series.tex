\documentclass[12pt]{article}
\usepackage{ctex}
\usepackage[english]{babel}
\usepackage{blindtext}
\usepackage{nameref}
\usepackage{fancyhdr}
\usepackage{color,amsmath,amssymb,amsthm,physics}
\usepackage{graphicx,float}
\usepackage{physics}
\usepackage{pgfplots}
\usepackage[a4paper, total={7in, 9in}]{geometry}
\usepackage{multicol}
\usepackage{framed}
\usepackage{xcolor}

\graphicspath{ {../images/} }

\definecolor{shadecolor}{RGB}{220,220,220}

\pagestyle{fancy}
\fancyhf{}
\fancyhf[HL]{A short course on Expansion}
\fancyhf[HR]{\rightmark}
\fancyhf[CF]{\thepage}
\fancyhf[FL]{\copyright Mok Owen 2024}

\newcommand{\innerprod}[2]{\langle{#1},{#2}\rangle}
\newcommand{\id}{\mathtt{id}}
\newcommand{\cis}[1]{\mathrm{cis}({#1})}

\newtheorem{definition}{Definition}[section]
\newtheorem*{theorem}{Theorem}
\newtheorem*{corollary}{Corollary}
\newtheorem*{lemma}{Lemma}
\newtheorem*{proposition}{Proposition}
\newtheorem*{remark}{Remark}
\newtheorem*{claim}{Claim}
\newtheorem*{example}{Example}
\newtheorem*{axiom}{Axiom}

\newtheorem{exercise}{Essential Practice}[section]
\newenvironment{solution}{\begin{snugshade*} \underline{\textbf{Solution.}} \par}{\hfill \textit{\dots end of solution} \end{snugshade*}}
\renewenvironment{proof}[1][Proof]{\begin{snugshade*} \underline{\textit{{#1}.}}\\}{\hfill \qedsymbol \end{snugshade*}}

\begin{document}
    \begin{abstract}
        To whom it may concern, it is always interesting the generalize notions from real-valued case to complex-valued case. The content will cover multinomial theorem and Taylor's expansion. It would be nice whenever there is add-on from the viewpoint of functional analysis.
    \end{abstract}

    \section{Binomial theorem}

    \subsection{Pascal's triangle}

    Notice the following: for a polynomial with two terms, we would like to call it \textbf{binomials}, and we are always interested in studying its power expansion, as a lovely connection to approximate roots of equation, following Newton's method of fluxion. A \textbf{Pascal's triangle} is a first glance to see what happens to do powers on it. Let $a,b$ be numbers, observe the following\begin{align*}
        (a+b)^0&=1\\
        (a+b)^1&=a+b\\
        (a+b)^2&=a^2+2ab+b^2\\
        (a+b)^3&=a^3+3a^2b+3ab^2+b^3\\
        &\vdots
    \end{align*}
    which is quite fascinating to find the pattern on coefficients $(1)\to(1,1)\to(1,2,1)\to(1,3,3,1)\to\cdots$ which could be built nicely by adding the $k$-th number to $(k+1)$-th number.

    Let us write it properly. Define an infiite vector space \[V:=\mathbb{R}^{\infty}=\{(x_i):i\in\mathbb{N}_0\}\] such that the identity element $e=:(1,0,0,\dots)\in V$. Define a mapping $\pi_2:V\to V$ by $(x_{i+1})\mapsto (x_i+x_{i+1})$ for all $i\in\mathbb{N}_0$. This would be an analogue of Pascal's triangle in $\mathbb{N}_0$.

    \begin{example}
        We may consider $(a+b)^0=e$ as a starting point, and perform $\pi_2$ to increase its power.\begin{align*}
            (a+b)&=\pi_2(e)=(1,1,0,0,\dots)\\
            (a+b)^2&=\pi_2((a+b))=(1,2,1,0,0,\dots)\\
            (a+b)^3&=\pi_2((a+b)^2)=(1,3,3,1,0,0,\dots)\\
            (a+b)^4&=\pi_2((a+b)^3)=(1,4,6,4,1,0,0,\dots)\\
            &\vdots
        \end{align*}
        This indeed matches the intuition of writing the coefficients of expansion.
    \end{example}

    It would be a convenience to figure out an inverse mapping for $\pi_2$. Denote the inverse mapping by $\pi_2^{-1}$. Notice \[\pi_2\circ\pi_2^{-1}=\id,\] which we could see \[(x_{i+1})\overset{\pi_2}{\mapsto}(x_i+x_{i+1})\overset{\pi_2^{-1}}{\mapsto} (x_{i+1})\] and that \[x_{i+1}=(x_{i+1}+x_{i})-x_i\] which implies \[\pi_2^{-1}(x_{i+1})=x_{i+1}-\pi_2^{-1}(x_i).\] Inductively, we obtain \[\pi_2^{-1}(x_{i+1})=x_{i+1}-x_i+x_{i-1}-\cdots+(-1)^i x_1+(-1)^{i+1}.\]

    \subsection{Binomial and multinomial theorem}

    Since then, may we write \[(a+b)^n=\prod_{1\leq i\leq n}(a+b)\] where $a,b$ are numbers. We know that multiplication works over brackets but not inside one bracket, so objects in the same bracket will never multiply each other. This provides us an intuition that binomial power expansion is a combinatorial over $n$ brakets.
    
    The general term will therefore be a combination of $a$ and $b$: for if we need to choose $a^rb^{n-r}$, we are in fact considering $r$ brackets of $a$ and $n-r$ brackets of $b$. This suggest the general term of the expansion to be \[\begin{pmatrix}
        n\\r
    \end{pmatrix}\begin{pmatrix}
        n-r\\n-r
    \end{pmatrix}a^r b^{n-r}=\begin{pmatrix}
        n\\r
    \end{pmatrix}a^r b^{n-r}\] where $\begin{pmatrix}
        n\\r
    \end{pmatrix}:=\dfrac{n!}{r!(n-r)!}$ is the combinatorial symbol. Hence we have the binomial theorem in the following form:

    \begin{theorem}[Binomial Theorem]
        Let $a,b\in\mathbb{F}$ for some commutative ring $\mathbb{F}$. Let $n\in\mathbb{N}$. Then \[(a+b)^n=\sum_{r=0}^{n}\begin{pmatrix}
            n\\r
        \end{pmatrix}a^r b^{n-r}.\]
    \end{theorem}

    \begin{proof}[Proof by induction]
        Suppose $(a+b)^n=\sum_{r=0}^{n}\begin{pmatrix}
            n\\r
        \end{pmatrix}a^r b^{n-r}$ is a true proposition, then \begin{align*}
            (a+b)^{n+1}&=(a+b)^n(a+b)\\
            &=(\sum_{r=0}^{n}\begin{pmatrix}
                n\\r
            \end{pmatrix}a^r b^{n-r})(a+b)\\
            &=\sum_{r=0}^{n}\begin{pmatrix}
                n\\r
            \end{pmatrix}a^{r+1} b^{n-r} + \sum_{r=0}^{n}\begin{pmatrix}
                n\\r
            \end{pmatrix}a^r b^{n+1-r}\\
            &=\sum_{r=1}^{n+1}\begin{pmatrix}
                n\\r-1
            \end{pmatrix}a^{r} b^{n+1-r} + \sum_{r=0}^{n}\begin{pmatrix}
                n\\r
            \end{pmatrix}a^r b^{n+1-r}\\
            &=\sum_{r=0}^{n+1}\begin{pmatrix}
                n+1\\r
            \end{pmatrix}a^{r} b^{n+1-r}
        \end{align*}
        which follows by principle of M.I.
    \end{proof}

    On the other hand, if we insert the notion into pascal's triangle, it is much easier to verify:

    \begin{proof}[Proof follows Pascal's Triangle]
        For the coefficient representation $(x_r)$ where $x_r=\begin{pmatrix}
            n\\r
        \end{pmatrix}$, the mapping results in \[\pi_2((x_r))=(x_{r-1}+x_{r})=(\begin{pmatrix}
            n\\r-1
        \end{pmatrix}+\begin{pmatrix}
            n\\r
        \end{pmatrix})=(\begin{pmatrix}
            n+1\\r
        \end{pmatrix}).\]
    \end{proof}

    In addition, we could verify on Pascal's triangle that \[\pi_2^{-1}(\begin{pmatrix}
        n\\r
    \end{pmatrix})=\begin{pmatrix}
        n-1\\r
    \end{pmatrix}.\]

    \begin{theorem}[Multinomial Theorem]
        Let $\{a_i\}\in\mathbb{F}$ be a sequence in some commutative ring $\mathbb{F}$. Let $n\in\mathbb{N}$. Then \[(\sum_{i=1}^{k}a_i)^n=\sum_{r_1+r_2+\cdots+r_k=n}\begin{pmatrix}
            n\\r_1
        \end{pmatrix}\begin{pmatrix}
            n-r_1\\r_2
        \end{pmatrix}\cdots\begin{pmatrix}
            n-r_1-r_2-\cdots-r_{k-2}\\r_{k-1}
        \end{pmatrix}a_1^{r_1}a_2^{r_2}\cdots a_k^{r_k}.\]
        In addition, we will define \[\begin{pmatrix}
            n\\r_1,r_2,\dots,r_k
        \end{pmatrix}:=\frac{n!}{r_1!r_2!\cdots r_k!}=\begin{pmatrix}
            n\\r_1
        \end{pmatrix}\begin{pmatrix}
            n-r_1\\r_2
        \end{pmatrix}\cdots\begin{pmatrix}
            n-r_1-r_2-\cdots-r_{k-2}\\r_{k-1}
        \end{pmatrix}\] so that the writing can be simplified to \[(\sum_{i=1}^{k}a_i)^n=\sum_{\sum_{i=1}^{k}r_i=n; r_i\geq 0}\begin{pmatrix}
            n\\r_1,r_2,\dots,r_k
        \end{pmatrix}\prod_{i=1}^k a_i^{r_k}.\]
    \end{theorem}

    \begin{proof}
        Proof is left as an exercise:\begin{enumerate}
            \item Show, by mathematical induction, that \[\frac{n!}{r_1!r_2!\cdots r_k!}=\begin{pmatrix}
                n\\r_1
            \end{pmatrix}\begin{pmatrix}
                n-r_1\\r_2
            \end{pmatrix}\cdots\begin{pmatrix}
                n-r_1-r_2-\cdots-r_{k-2}\\r_{k-1}
            \end{pmatrix}\] is true for all positive integer $n$ with $\sum_{i=1}^{k}r_k=n$.
            \item Prove, with the help of (a) and Mathematical induction, that \[(\sum_{i=1}^{k}a_i)^n=\sum_{\sum_{i=1}^{k}r_i=n; r_i\geq 0}\begin{pmatrix}
                n\\r_1,r_2,\dots,r_k
            \end{pmatrix}\prod_{i=1}^k a_i^{r_k}\] is true for all positive integer $n$ with $\sum_{i=1}^{k}r_k=n$.
        \end{enumerate}
    \end{proof}

    \subsection{Binomial Theorem with negative indices}

    Suppose we are going to extend the binomial theorem to negative indices, in polynomial forms, so that the theorem would be easier to be generalized. We identify $\begin{pmatrix}
        n\\r
    \end{pmatrix}$ to the entries discussed in Pascal's triangles, so whenever $r<0$, we set $\begin{pmatrix}
        n\\r
    \end{pmatrix}=0$. To agree with negative indices, we suspend the constraints when $r>n$. Consider the expansions \begin{align*}
        \frac{1}{1+x}&=\sum_{r=0}^{\infty}(-1)^r x^r=(1,-1,1,-1,\dots)=((-1)^r)\in V,\\
        \frac{1}{(1+x)^2}&=(-1)\derivative{x}\sum_{r=0}^{\infty}(-1)^r x^r=\sum_{r=0}^{\infty}(-1)^{r} \begin{pmatrix}
            r+1\\1
        \end{pmatrix}x^r=(1,-2,3,-4,\dots)\in V,\\
        \frac{1}{(1+x)^3}&=\frac{1}{-2}\derivative{x}\sum_{r=0}^{\infty}(-1)^{r} (r+1)x^r=\sum_{r=0}^{\infty}(-1)^{r} \begin{pmatrix}
            r+2\\2
        \end{pmatrix}x^r=(1,-3,6,-10,\dots)\in V
    \end{align*}

    It is valid to develop the theory on this type of binomials dropping the restriction on $x$, as we can apply it to $p$-adic series with satisfactory. It is now the question of how to relate these coefficients to the defined parenthesis symbol. Notice the representation still satisfy Pascal's triangle: \[\pi_2((1,-1,1,-1,\dots))=(1,0,0,\dots)=e\in V,\] which provides us an intuition to compute the negative binomial coefficients.

    \begin{theorem}
        Given $n>0$ and $r\geq 0$ are integers, we have the negative binomial coefficient as \[\begin{pmatrix}
            -n\\r
        \end{pmatrix}=(-1)^r\begin{pmatrix}
            r+n-1\\r
        \end{pmatrix}\]
    \end{theorem}

    \begin{proof}
        Let $n>0$ and $r\geq 0$ be integers. If $\begin{pmatrix}
            -n\\r
        \end{pmatrix}=(-1)^r\begin{pmatrix}
            r+n-1\\r
        \end{pmatrix}$ is true, then \[\pi_2^{-1}(\begin{pmatrix}
            -n\\r
        \end{pmatrix})=(-1)^r \pi_2^{-1}(\begin{pmatrix}
            r+n-1\\r
        \end{pmatrix})=(-1)^r \begin{pmatrix}
            r+n-2\\r
        \end{pmatrix}=\begin{pmatrix}
            -(n+1)\\r
        \end{pmatrix}.\]
    \end{proof}

    To facilitate the coefficient symbol, let's pay attention to how the assignment can be modified. We shall 

\end{document}