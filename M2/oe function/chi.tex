\documentclass[12pt]{article}
\usepackage{ctex}
\usepackage[english]{babel}
\usepackage{blindtext}
\usepackage{nameref}
\usepackage{fancyhdr}
\usepackage{amsmath,amssymb,amsthm}
\usepackage{graphicx,float}
\usepackage{physics}
\usepackage{pgfplots}
\usepackage[a4paper, total={6in, 9in}]{geometry}

\pagestyle{fancy}
\fancyhf{}
\fancyhf[HL]{函數}
\fancyhf[CF]{\thepage}

\newcommand{\innerprod}[2]{\langle{#1},{#2}\rangle}
\newcommand{\id}{\mathtt{id}}

\newtheorem{definition}{定義}
\newtheorem*{theorem}{定理}
\newtheorem*{corollary}{衍理}
\newtheorem*{lemma}{引理}
\newtheorem*{proposition}{設理}
\newtheorem*{remark}{小記}
\newtheorem*{claim}{主張}
\newtheorem*{example}{例子}
\newtheorem*{axiom}{公設}
\renewenvironment*{proof}{\textit{證明.}}{\hfill$\qed$}

\newenvironment*{sol}{\par \textbf{解}.}{\hfill$\blacksquare$}

\begin{document}
    \section*{數集}

    \begin{definition}[元素]
        設$X$為數集,則$$x\in X$$表示$x$為數集$X$的成員;相反,$x\notin X$表示$x$并非數集$X$的成員。 
    \end{definition}

    \begin{example}
        設$X=\{1,2,3,4,5\}$,則$1\in X$,$0\notin X$.
    \end{example}

    \begin{definition}[數集等價]
        設$X$及$Y$皆爲數集,其中$Y$為$X$重複附帶$x\in X$,則寫$Y=X$.
    \end{definition}

    \begin{example}
        設$X=\{1,2,3,4,5\}$及$Y=\{1,1,2,3,4,5\}$,則$X=Y$.
    \end{example}

    \begin{definition}[子集]
        設$X$及$Y$皆爲數集,則$$X\subset Y$$ 表示所有$X$的成員都是$Y$的成員。我們稱$X$為$Y$的子集。
    \end{definition}

    \begin{example}
        設$X=\{1,2,3\}$,$Y=\{1,2,3,4\}$,$Z=\{0,1\}$。則$X\subset Y$ 但$Z\not \subset Y$。
    \end{example}

    \begin{proposition}
        設$X,Y$及$Z$皆爲集合,則以下成立:\begin{enumerate}
            \item $X\subset X$。
            \item 若$X\subset Y$及$Y\subset Z$,則$X\subset Z$。
            \item 若$X\subset Y$及$Y\subset X$,則$X=Y$。
        \end{enumerate}
    \end{proposition}

    \begin{proof}
        \begin{enumerate}
            \item 所有$x\in X$都自然為$X$的成員。
            \item 若$X\subset Y$及$Y\subset Z$,則以下假設成立:\begin{enumerate}
                \item 所有$x\in X$都是$Y$的成員。
                \item 所有$y\in Y$都是$Z$的成員。
            \end{enumerate}
            則$x\in X$令$x\in Y$,則令$x\in Z$。所以 $X\subset Z$.
            \item 所有$x\in X$都是$Y$的成員,因此$Y$有可能擁有$X$以外的成員。但$Y\subset X$,所以$Y$不可能擁有$X$以外的成員。所以$Y=X$。
        \end{enumerate}
    \end{proof}

    由於我們需要面對無限集合,因此我們不可能每一次都明確的寫下所有元素。於是有以下一種定義數集的方式:

    \begin{definition}
        設$X$爲數集和判斷式$P$,則有且僅有一個$X$的子集令$P(x)$對於所有子集的成員$x$皆為正確。此子集記爲$$\{x\in X|P(x)\}\textrm{ 或 }\{x\in X:P(x)\}$$
    \end{definition}

    \begin{example}
        $\{x\in\mathbb{R}:x^2-2=0\}$等價$\{-\sqrt{2},\sqrt{2}\}$.
    \end{example}

    \begin{definition}[空集]
        若$X$沒有成員,我們稱之爲\textbf{空集},記$\emptyset$。
    \end{definition}

    \begin{definition}[運算封閉]
        設$S$爲一個非空數集,若對任意$a,b\in S$均有$a+b\in S$,則説$S$關於加法封閉。同樣可以對減法、乘法、除法定義封閉性。
    \end{definition}

    \begin{definition}[數域]
        設$S$為數集,至少包含兩個數,並且關於加、減、乘、除四則運算封閉,則稱$S$為\textbf{數域}。
    \end{definition}

    \begin{example}[實數域]
        $\mathbb{R}$為包含所有實數之集合,亦是數域。
    \end{example}

    \section*{函數}

    \begin{definition}[映射、定義域、值域]
        若$f$為\textbf{映射},擁有\textbf{定義域}$D$和\textbf{值域}$R$,則以下三種敘述均可描述$f$:\begin{itemize}
            \item $f:D\to R$;
            \item $x\in D$, $f(x)\in R$, $x\mapsto f(x)$;
            \item $f(x)=$以$x$建立/表述的算式。
        \end{itemize}
        代表$f$爲將$D$的成員映射至$R$的方式。
    \end{definition}

    \begin{axiom}[函數基本性質]
        任何可稱爲\textbf{函數}的映射$f:D\to R$,必須符合以下性質:\begin{enumerate}
            \item 存在性:若$x\in D$,則$f(x)\in R$;
            \item 唯一性:若$x,y\in D$且$x=y$,則$f(x)=f(y)$。
        \end{enumerate}
    \end{axiom}

    \begin{definition}[實函數]
        若$f:D\to R$為函數,值域$R=\mathbb{R}$(或$R\subset \mathbb{R}$),我們稱之爲\textbf{實函數}。
    \end{definition}

    \begin{definition}[複函數]
        若$f:D\to R$為函數,值域$R=\mathbb{C}$(或$R\subset \mathbb{C}$),我們稱之爲\textbf{複函數}。
    \end{definition}

    \begin{corollary}
        任何實函數都是複函數。
    \end{corollary}

    由於逆函數的存在性並非必然,故需要作出新的猜想以求其存在條件。實際上,若逆函數存在,則必然牽涉其作爲函數的基本性質。

    於是我們給出以下條件:

    若$f$的逆函數$g$存在,則下列條件成立:\begin{enumerate}
        \item 存在性:若$y\in R$,則$g(y)\in D$;
        \item 唯一性:若$x,y\in R$且$x=y$,則$g(x)=g(y)$。
    \end{enumerate}
    並且$f(g(y))=y,g(f(x))=x$。

    於是,從$f$的角度定義:

    \begin{definition}[單射性]
        設$f:D\to R$為函數。若當$f(x)=f(y)$時,可令$x=y$,則稱$f$為\textbf{單射函數}。
    \end{definition}
    
    \begin{definition}[滿射性]
        設$f:D\to R$為函數。若當$y\in R$時,存在至少一個$x\in D$符合$y=f(x)$,則稱$f$為\textbf{滿射函數}。
    \end{definition}

    以上定義分別對應逆函數的唯一性與存在性。因此若定義:

    \begin{definition}[雙射性]
        函數$f$若同時滿足單射性與滿射性,則稱其爲\textbf{雙射函數}。
    \end{definition}

    \begin{theorem}
        函數$f$的逆函數存在當且僅當$f$為雙射函數。
    \end{theorem}

    對於擁有逆函數的函數$f$,我們稱$f$是\textbf{可逆的},並記其逆函數作$f^{-1}$。

    \begin{example}
        證明$f:\mathbb{R}\to\mathbb{R}$且$x\mapsto x^{2n+1}$為可逆函數,其中$n$為非負整數,並求出$f^{-1}$。

        \begin{sol}
            證明$f$的單射性:\begin{align*}
                f(x)&=f(y)\\
                x^{2n+1}&=y^{2n+1}\\
                x^{2n+1}-y^{2n+1}&=0\\
                (x-y)\sum_{i=0}^{2n}x^i y^{2n-i}&=0
            \end{align*}
            考慮$f(x)=f(y)$則$sgn(x)=sgn(y)$,其中$sgn(x):=\begin{cases}
                +1&x>0\\
                0&x=0\\
                -1&x<0
            \end{cases}$。故$sgn(x^i y^{2n-i})=(\pm 1)^{2n}=+1$。因此$\displaystyle\sum_{i=0}^{2n}x^i y^{2n-i}\neq 0$,則$x=y$。

            證明$f$的滿射性:對於任意$y$,存在$x=y^{\frac{1}{2n+1}}$滿足$x^{2n+1}=y$。

            因此,$f$是雙射的,擁有逆函數$f^{-1}$。$(f^{-1}(y))^{2n+1}=y$,故$f^{-1}(y)=y^{\frac{1}{2n+1}}$。
        \end{sol}
    \end{example}

    \section*{奇偶函數}

    \begin{definition}[奇偶函數]
        設$f:D\to R$為函數。\begin{itemize}
            \item 若對於所有$x\in D$,均使$f(-x)=-f(x)$成立,則$f$為\textbf{奇函數};
            \item 若對於所有$x\in D$,均使$f(-x)=f(x)$成立,則$f$為\textbf{偶函數}。
        \end{itemize}
    \end{definition}

    \begin{example}
        \begin{enumerate}
            \item $f(x)=x$時,$f$為奇函數;
            \item 對於整數$n$,$f(x)=x^{2n+1}$時,$f$為奇函數;
            \item $f(x)=x^2$時,$f$為偶函數;
            \item 對於整數$n$,$f(x)=x^{2n}$時,$f$為偶函數;
            \item $\sin{x},\tan{x},\csc{x},\cot{x}$均爲奇函數;
            \item $\cos{x},\sec{x}$均爲偶函數。
        \end{enumerate}
    \end{example}

    \begin{proposition}
        設$f_1,f_2$為奇函數,$g_1,g_2$為偶函數。則以下成立:\begin{itemize}
            \item $f_1\circ f_2$是奇函數;
            \item $f_1\circ g_1$是偶函數;
            \item $g_1\circ f_2$是偶函數;
            \item $g_1\circ g_2$是偶函數。
        \end{itemize}
    \end{proposition}

    \begin{proof}
        設$f_1,f_2$為奇函數,$g_1,g_2$為偶函數。則\begin{itemize}
            \item $f_1(-x)=-f_1(x)$;
            \item $f_2(-x)=-f_2(x)$;
            \item $g_1(-x)=g_1(x)$;
            \item $g_2(-x)=g_2(x)$。
        \end{itemize}
        由此可得\begin{itemize}
            \item $(f_1\circ f_2)(-x)=f_1(f_2(-x))=f_1(-f_2(x))=-f_1(f_2(x))=-(f_1\circ f_2)(x)$;
            \item $(f_1\circ g_2)(-x)=f_1(g_2(-x))=f_1(g_2(x))=(f_1\circ g_2)(x)$;
            \item $(g_1\circ f_2)(-x)=g_1(f_2(-x))=g_1(-f_2(x))=g_1(f_2(x))=(g_1\circ f_2)(x)$;
            \item $(g_1\circ g_2)(-x)=g_1(g_2(-x))=g_1(g_2(x))=(g_1\circ g_2)(x)$。
        \end{itemize}
    \end{proof}

\end{document}