\documentclass[12pt]{article}
\usepackage{ctex}
\usepackage[english]{babel}
\usepackage{blindtext}
\usepackage{nameref}
\usepackage{fancyhdr}
\usepackage{amsmath,amssymb,amsthm}
\usepackage{graphicx,float}
\usepackage{physics}
\usepackage{pgfplots}
\usepackage[a4paper, total={6in, 9in}]{geometry}

\graphicspath{{../image/}}

\pagestyle{fancy}
\fancyhf{}
\fancyhf[HL]{引導練習題解}
\fancyhf[CF]{\thepage}

\newcommand{\innerprod}[2]{\langle{#1},{#2}\rangle}
\newcommand{\id}{\mathtt{id}}

\newtheorem{definition}{定義}
\newtheorem*{theorem}{定理}
\newtheorem*{corollary}{衍理}
\newtheorem*{lemma}{引理}
\newtheorem*{proposition}{命題}
\newtheorem*{remark}{小記}
\newtheorem*{claim}{主張}
\newtheorem*{example}{示例}
\newtheorem*{axiom}{公設}
\renewenvironment*{proof}{\textit{證明.}}{\hfill$\qed$}

\newenvironment*{sol}{\par \textbf{解}.}{\hfill$\blacksquare$}

\begin{document}
    \begin{enumerate}
        \item \begin{enumerate}
            \item \begin{align*}
                \begin{bmatrix}
                    y_1\\y_2\\\vdots\\y_n
                \end{bmatrix}=\begin{bmatrix}
                    a_{11}&a_{12}&\cdots&a_{1n}\\
                    a_{21}&a_{22}&\cdots&a_{2n}\\
                    \vdots&\vdots&\ddots&\vdots\\
                    a_{n1}&a_{n2}&\cdots&a_{nn}
                \end{bmatrix}\begin{bmatrix}
                    x_1\\x_2\\\vdots\\x_n
                \end{bmatrix}
            \end{align*}
            \item 欲證明$\mathbf{y}_k=A\mathbf{x}_k$,則證明\begin{align*}
                \begin{bmatrix}
                    y_{1k}\\y_{2k}\\\vdots\\y_{nk}
                \end{bmatrix}=\begin{bmatrix}
                    a_{11}&a_{12}&\cdots&a_{1n}\\
                    a_{21}&a_{22}&\cdots&a_{2n}\\
                    \vdots&\vdots&\ddots&\vdots\\
                    a_{n1}&a_{n2}&\cdots&a_{nn}
                \end{bmatrix}\begin{bmatrix}
                    x_{1k}\\x_{2k}\\\vdots\\x_{nk}
                \end{bmatrix}
            \end{align*}
            將之簡化,則需證明$\forall i, \forall k$,$$y_{ik}=\sum_{j=1}^{n}a_{ij}x_{jk}$$
            此為矩陣乘法定義,因此證畢。
            \item 此矩陣按固定變換方式將$\mathbf{x}_k$映射至$\mathbf{y}_k$。視$\mathbf{x}_k$與$\mathbf{y_k}$為$n$維空間内的兩組坐標,則是$n$組坐標按同一方式進行坐標變換,可視作幾何圖形的綫性變換。
        \end{enumerate}
        \item \begin{enumerate}
            \item $f(x)=g^{-1}(d(g(x)))=P^{-1}DPx=Ax$。$f$為以矩陣$A$對坐標$x$進行綫性變換。變換過程:先從$X$的基底變換至$Y$的基底,按$Y$的基底利用矩陣$D$放大縮小,再變換回$X$基底。
            \item 若$X$的維數為$m$,$Y$的維數為$n$,設$x\in X, y\in Y$,則$$\begin{bmatrix}
                y_1\\y_2\\\vdots\\y_n
            \end{bmatrix}=\begin{bmatrix}
                b_{11}&b_{12}&\cdots&b_{1m}\\
                b_{21}&b_{22}&\cdots&b_{2m}\\
                \vdots&\vdots&\ddots&\vdots\\
                b_{n1}&b_{n2}&\cdots&b_{nm}
            \end{bmatrix}\begin{bmatrix}
                x_1\\x_2\\\vdots\\x_m
            \end{bmatrix}$$
            定$y=Bx$。若$m<n$,則$B=\begin{bmatrix}
                b_{11}&b_{12}&\cdots&b_{1m}&0&\cdots&b_{1n}=0\\
                b_{21}&b_{22}&\cdots&b_{2m}&0&\cdots&b_{2n}=0\\
                \vdots&\vdots&\ddots&\vdots&\vdots&\cdots&\vdots\\
                b_{n1}&b_{n2}&\cdots&b_{nm}&0&\cdots&b_{nn}=0
            \end{bmatrix}$使得$|B|=0$,則$B$不可逆;若$m>n$,則$B=\begin{bmatrix}
                b_{11}&b_{12}&\cdots&b_{1m}\\
                b_{21}&b_{22}&\cdots&b_{2m}\\
                \vdots&\vdots&\ddots&\vdots\\
                b_{n1}&b_{n2}&\cdots&b_{nm}\\
                0&0&\cdots&0\\
                \vdots&\vdots&\ddots&\vdots\\
                b_{m1}=0&b_{m2}=0&\cdots&b_{mm}=0
            \end{bmatrix}$使得$|B|=0$,則$B$不可逆。
            \item 不可被對角化的幾何含義為綫性變換的維數不同。
        \end{enumerate}
        \item \begin{enumerate}
            \item \begin{enumerate}
                \item \begin{align*}
                    \derivative{A}{t}&=\lim_{h\to 0}\frac{A-A}{h}=0
                \end{align*}
                \item \begin{align*}
                    \derivative{(At^n)}{t}&=\lim_{h\to 0}\frac{A(t+h)^n-At^n}{h}\\
                    &=A\lim_{h\to 0}\frac{(t+h)^n-t^n}{h}\\
                    &=A\derivative{t^n}{t}\\
                    &=A(nt^{n-1})\\
                    &=nAt^{n-1}
                \end{align*}
                \item \begin{align*}
                    \derivative{e^{At}}{t}&=\lim_{h\to 0}\frac{e^{A(t+h)}-e^{At}}{h}\\
                    &=e^{At}\lim_{h\to 0}\frac{e^{Ah}-I}{h}\\
                \end{align*}
                考慮\begin{align*}
                    \frac{e^{Ah}-I}{h}&=\frac{1}{h}(\sum_{n=0}^{\infty}\frac{(Ah)^n}{n!}-I)\\
                    &=\frac{1}{h}(\sum_{n=1}^{\infty}\frac{(Ah)^n}{n!})\\
                    &=\sum_{n=1}^{\infty}\frac{A^nh^{n-1}}{n!}\\
                    &=\sum_{n=0}^{\infty}\frac{A^{n+1}h^n}{(n+1)!}\\
                    &=A+\sum_{n=1}^{\infty}\frac{A^{n+1}h^n}{(n+1)!}
                \end{align*}
                因此\begin{align*}
                    \lim_{h\to 0}\frac{e^{Ah}-I}{h}&=A\\
                    \derivative{e^{At}}{t}&=e^{At}A\\
                    &=\sum_{n=0}^{\infty}\frac{A^nt^n}{n!}A\\
                    &=\sum_{n=0}^{\infty}\frac{A^{n+1}t^n}{n!}\\
                    &=A\sum_{n=0}^{\infty}\frac{A^nt^n}{n!}\\
                    &=Ae^{At}
                \end{align*}
            \end{enumerate}
            \item \begin{enumerate}
                \item 設$F(t):=\begin{bmatrix}
                    f_{11}(t)&f_{12}(t)&\cdots&f_{1n}(t)\\
                    f_{21}(t)&f_{22}(t)&\cdots&f_{2n}(t)\\
                    \vdots&\vdots&\ddots&\vdots\\
                    f_{m1}(t)&f_{m2}(t)&\cdots&f_{mn}(t)
                \end{bmatrix}$,$F'(t)=\begin{bmatrix}
                    f_{11}'(t)&f_{12}'(t)&\cdots&f_{1n}'(t)\\
                    f_{21}'(t)&f_{22}'(t)&\cdots&f_{2n}'(t)\\
                    \vdots&\vdots&\ddots&\vdots\\
                    f_{m1}'(t)&f_{m2}'(t)&\cdots&f_{mn}'(t)
                \end{bmatrix}$,$G(t):=\begin{bmatrix}
                    g_{11}(t)&g_{12}(t)&\cdots&g_{1\ell}(t)\\
                    g_{21}(t)&g_{22}(t)&\cdots&g_{2\ell}(t)\\
                    \vdots&\vdots&\ddots&\vdots\\
                    g_{n1}(t)&g_{n2}(t)&\cdots&g_{n\ell}(t)
                \end{bmatrix}$,$G'(t)=\begin{bmatrix}
                    g_{11}'(t)&g_{12}'(t)&\cdots&g_{1\ell}'(t)\\
                    g_{21}'(t)&g_{22}'(t)&\cdots&g_{2\ell}'(t)\\
                    \vdots&\vdots&\ddots&\vdots\\
                    g_{n1}'(t)&g_{n2}'(t)&\cdots&g_{n\ell}'(t)
                \end{bmatrix}$
                則\begin{align*}
                    (FG)_{ij}&=\sum_{k=1}^{n}f_{ik}(t)g_{kj}(t)\\
                    (FG)_{ij}'&=\sum_{k=1}^{n}[f_{ik}'(t)g_{kj}(t)+f_{ik}(t)g_{kj}'(t)]\\
                    &=\sum_{k=1}^{n}[f_{ik}'(t)g_{kj}(t)]+\sum_{k=1}^{n}[f_{ik}(t)g_{kj}'(t)]\\
                    &=(F'G+FG')_{ij}
                \end{align*}
                \item 不適用。
            \end{enumerate}
            \item $At^{A-I}$
        \end{enumerate}
        \item \begin{enumerate}
            \item 出錯。
            \item 設$X=e^A=P^{-1}e^KP$,則$A=P^{-1}KP=P^{-1}\log{D}P$。
            \item 設$A(t)=\log{Xt}$,則$e^{A(t)}A'(t)=X\implies A'(t)=(Xt)^{-1}X=t^{-1}I$
        \end{enumerate}
    \end{enumerate}
\end{document}