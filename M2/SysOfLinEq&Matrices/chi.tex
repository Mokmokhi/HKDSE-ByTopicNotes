\documentclass[12pt]{article}
\usepackage{ctex}
\usepackage[english]{babel}
\usepackage{blindtext}
\usepackage{nameref}
\usepackage{fancyhdr}
\usepackage{amsmath,amssymb,amsthm}
\usepackage{graphicx,float}
\usepackage{physics}
\usepackage{pgfplots}
\usepackage[a4paper, total={6in, 9in}]{geometry}

\graphicspath{{../image/}}

\pagestyle{fancy}
\fancyhf{}
\fancyhf[HL]{矩陣}
\fancyhf[CF]{\thepage}

\newcommand{\innerprod}[2]{\langle{#1},{#2}\rangle}
\newcommand{\id}{\mathtt{id}}
\newcommand{\rref}{\mathrm{RREF}}

\newtheorem{definition}{定義}
\newtheorem*{theorem}{定理}
\newtheorem*{corollary}{衍理}
\newtheorem*{lemma}{引理}
\newtheorem*{proposition}{命題}
\newtheorem*{remark}{小記}
\newtheorem*{claim}{主張}
\newtheorem*{example}{示例}
\newtheorem*{axiom}{公設}
\renewenvironment*{proof}{\textit{證明.}}{\hfill$\qed$}

\newenvironment*{sol}{\par \textbf{解}.}{\hfill$\blacksquare$}

\begin{document}
    \section*{矩陣的定義}

    \begin{definition}[矩陣]
        \textbf{矩陣}是代數學中一種特殊的表達形式,用以同時表達多位元的表格:$$\mathbf{A}=(a_{ij})_{m\times n}=\begin{pmatrix}
            a_{11}&a_{12}&\cdots&a_{1n}\\
            a_{21}&a_{22}&\cdots&a_{2n}\\
            \vdots&\vdots&\ddots&\vdots\\
            a_{m1}&a_{m2}&\cdots&a_{mn}
        \end{pmatrix}$$
        若\textbf{元素}$a_{ij}\in S$則稱爲\textbf{$S$上的$m\times n$矩陣},記$\mathbf{A}\in \mathbb{M}^{m\times n}(S)=\mathbb{M}_S^{m\times n}$。
    \end{definition}

    \begin{example}
        以下爲矩陣的例子:
        \begin{enumerate}
            \item $1=:[1]\in\mathbb{M}^{1\times 1}(\mathbb{R})$。
            \item $\begin{bmatrix}
                1&0\\0&1
            \end{bmatrix}\in\mathbb{M}^{2\times 2}(\mathbb{R})$。
            \item $\begin{bmatrix}
                1+2i&3-4i&i\\-i&1&2-i
            \end{bmatrix}\in\mathbb{M}^{2\times 3}(\mathbb{C})$。
        \end{enumerate}
        
    \end{example}

    \begin{definition}[矩陣的加法與乘法定義]
        給定$\mathbf{A},\mathbf{B}\in\mathbb{M}^{m\times n}$,$\mathbf{C}\in\mathbb{M}^{n\times t}$,則\begin{enumerate}
            \item 加法:$\mathbf{A}+\mathbf{B}=(a_{ij}+b_{ij})$。
            \item 數乘:$k\mathbf{A}=(ka_{ij})$。
            \item 矩陣乘法:$\mathbf{A}\times\mathbf{B}=(\sum_{k=1}^{n}a_{ik}b_{kj})$。
        \end{enumerate}
    \end{definition}

    \begin{example}
        設$\mathbf{A}=\begin{pmatrix}
            1&1&0\\1&0&1
        \end{pmatrix},\mathbf{B}=\begin{pmatrix}
            2&1&0\\1&2&1
        \end{pmatrix},\mathbf{C}=\begin{pmatrix}
            2&0\\2&1\\3&3
        \end{pmatrix}$,則\begin{enumerate}
            \item $\mathbf{A}+\mathbf{B}=\begin{pmatrix}
                1+2&1+1&0+0\\1+1&0+2&1+1
            \end{pmatrix}=\begin{pmatrix}
                3&2&0\\2&2&2
            \end{pmatrix}$。
            \item $4\mathbf{A}=\begin{pmatrix}
                4&4&0\\4&0&4
            \end{pmatrix}$。
            \item $\mathbf{A}\times\mathbf{C}=\begin{pmatrix}
                1\times2+1\times2+0\times3&1\times0+1\times1+0\times3\\1\times2+0\times2+1\times3&1\times0+0\times1+1\times3
            \end{pmatrix}=\begin{pmatrix}
                4&1\\5&3
            \end{pmatrix}$。
        \end{enumerate}
    \end{example}

    \begin{definition}[特殊矩陣]
        下列為一些矩陣寫法的共識:\begin{itemize}
            \item 零矩陣:$\mathbf{0}_{m\times n}=\mathbf{O}_{m\times n}=(0)_{m\times n}$。
            \item 一矩陣:$\mathbf{1}_{m\times n}=(1)_{m\times n}$。
            \item 單位矩陣:$\mathbf{I}_{m\times n}=(\delta_{ij})_{m\times n}$,其中$\delta_{ij}=\begin{cases}
                1 & i=j\\ 0 & i\neq j
            \end{cases}$。
            \item 向量:\begin{itemize}
                \item 行向量:$\mathbf{x}^T=\vec{x}^T=\begin{bmatrix}
                    x_1&x_2&\cdots&x_n
                \end{bmatrix}=(x_1,x_2,\dots,x_n)^T\in\mathbb{M}^{1\times n}$。
                \item 列向量:$\mathbf{y}=\vec{y}=\begin{bmatrix}
                    y_1\\y_2\\\vdots\\y_m
                \end{bmatrix}=(y_1,y_2,\dots,y_m)\in\mathbb{M}^{m\times 1}$
            \end{itemize}
        \end{itemize}
    \end{definition}

    \begin{proposition}[加法與數乘定則]
        設$\mathbf{A},\mathbf{B},\mathbf{C}\in\mathbb{M}^{m\times n}$,則\begin{enumerate}
            \item 加法結合律:$(\mathbf{A}+\mathbf{B})+\mathbf{C}=\mathbf{A}+(\mathbf{B}+\mathbf{C})$。
            \item 加法交換律:$\mathbf{A}+\mathbf{B}=\mathbf{B}+\mathbf{A}$。
            \item $\mathbf{A}+\mathbf{0}=\mathbf{A}$。
            \item $\mathbf{A}+(-\mathbf{A})=\mathbf{0}$。
            \item $a(\mathbf{A}+\mathbf{B})=a\mathbf{A}+\mathbf{B}$。
            \item $(a+b)\mathbf{A}=a\mathbf{A}+b\mathbf{A}$。
            \item $(ab)\mathbf{A}=a(b\mathbf{A})$
        \end{enumerate}
    \end{proposition}

    \begin{proposition}[矩陣乘法定則]
        設$\mathbf{A},\mathbf{B},\mathbf{C},\mathbf{D}$為矩陣,則\begin{enumerate}
            \item 乘法結合律:$(\mathbf{A}\mathbf{B})\mathbf{C}=\mathbf{A}(\mathbf{B}\mathbf{C})$。
            \item 分配律I:$(\mathbf{A}+\mathbf{B})\mathbf{C}=\mathbf{A}\mathbf{C}+\mathbf{B}\mathbf{C}$。
            \item 分配律II:$\mathbf{D}(\mathbf{A}+\mathbf{B})=\mathbf{D}\mathbf{A}+\mathbf{D}\mathbf{B}$。
            \item $\alpha\mathbf{A}\mathbf{B}=(\alpha\mathbf{A})\mathbf{B}=\mathbf{A}(\alpha\mathbf{B})$。
            \item $\mathbf{A}\mathbf{I}=\mathbf{I}\mathbf{A}=\mathbf{A}$。
        \end{enumerate}
    \end{proposition}
    \begin{definition}[矩陣的轉置]
        設$\mathbf{A}=(a_{ij})_{m\times n}=\begin{pmatrix}
            a_{11}&a_{12}&\cdots&a_{1n}\\
            a_{21}&a_{22}&\cdots&a_{2n}\\
            \vdots&\vdots&\ddots&\vdots\\
            a_{m1}&a_{m2}&\cdots&a_{mn}
        \end{pmatrix}$,則定義\textbf{$\mathbf{A}$的轉置}為$$\mathbf{A}^T=(a_{ji})_{n\times m}=\begin{pmatrix}
            a_{11}&a_{21}&\cdots&a_{m1}\\
            a_{12}&a_{22}&\cdots&a_{m2}\\
            \vdots&\vdots&\ddots&\vdots\\
            a_{1n}&a_{2n}&\cdots&a_{mn}
        \end{pmatrix}$$
    \end{definition}

    \begin{example}
        設$\mathbf{A}=\begin{pmatrix}
            1&2&3\\4&5&6\\7&8&9
        \end{pmatrix}$,則$\mathbf{A}^T=\begin{pmatrix}
            1&4&7\\2&5&8\\3&6&9
        \end{pmatrix}$。
    \end{example}

    \begin{definition}[矩陣的跡]
        設$\mathbf{A}=\begin{pmatrix}
            a_{11}&a_{12}&\cdots&a_{1n}\\
            a_{21}&a_{22}&\cdots&a_{2n}\\
            \vdots&\vdots&\ddots&\vdots\\
            a_{m1}&a_{m2}&\cdots&a_{mn}
        \end{pmatrix}$,則定義\textbf{$\mathbf{A}$的跡}為$$\tr(\mathbf{A})=\sum_{k}^{\min(m,n)}a_{kk}=a_{11}+a_{22}+\cdots+a_{ll}$$其中$l=\min(m,n)$。
    \end{definition}

    \begin{proposition}
        $\tr(\mathbf{A})=\tr(\mathbf{A}^T)$。
    \end{proposition}
    \begin{proof}
        留做習題。
    \end{proof}
    
    \begin{definition}[綫性組合]
        設$c_1,c_2,\dots,c_\ell$為常數,$\mathbf{A}_1,\mathbf{A}_2,\dots,\mathbf{A}_\ell\in\mathbb{M}_{m\times n}$,則稱$$\sum_{k=1}^{\ell}c_k\mathbf{A}_k=c_1\mathbf{A}_1+c_2\mathbf{A}_2+\cdots+c_\ell\mathbf{A}_\ell$$為\textbf{綫性組合}。
    \end{definition}

    \begin{example}
        設$e_1=\begin{pmatrix}
            1\\0\\0
        \end{pmatrix},e_2=\begin{pmatrix}
            0\\1\\0
        \end{pmatrix},e_3=\begin{pmatrix}
            0\\0\\1
        \end{pmatrix}$,則可寫
            $$\begin{pmatrix}
                3\\2\\1
            \end{pmatrix}=3e_1+2e_2+e_1$$
    \end{example}

    \begin{example}
        設$E_n\in\mathbb{M}^{3\times 3}$,對於$1\leq n\leq 9$,$E_n=(e_{ij})_{3\times 3}$而且$$e_{ij}=\begin{cases}
            1 & n=3(i-1)+j\\ 0 & otherwise
        \end{cases}$$,則可寫
            $$\begin{pmatrix}
                1&2&3\\4&5&6\\7&8&9
            \end{pmatrix}=\sum_{k=1}^{9}kE_k$$
    \end{example}

    \begin{remark}
        以上例子便是矩陣與綫性方程組的關係。
    \end{remark}

    \begin{proposition}
        設$c_1,c_2,\dots,c_\ell$為常數,$\mathbf{A}_1,\mathbf{A}_2,\dots,\mathbf{A}_\ell\in\mathbb{M}_{m\times n}$,則\begin{itemize}
            \item $\tr(\sum_{k=1}^{\ell}c_k\mathbf{A}_k)=\sum_{k=1}^{\ell}c_k\tr(\mathbf{A}_k)$;
            \item $\left(\sum_{k=1}^{\ell}c_k\mathbf{A}_k\right)^T=\sum_{k=1}^{\ell}c_k\mathbf{A}_k^T$;
            \item $\tr(\sum_{k=1}^{\ell}c_k\mathbf{A}_k^T)=\sum_{k=1}^{\ell}c_k\tr(\mathbf{A}_k)$。
        \end{itemize}
    \end{proposition}
    
    \begin{proof}
        留作習題。
    \end{proof}
    
    \newpage

    \section*{綫性方程組}

    綫性方程組主要用以表達兩條或多條同時成立的綫性方程,最具代表性的可數初中的聯立方程:$$\begin{cases}
        ax+by&=k_1\\cx+dy&=k_2
    \end{cases}$$

    當然,綫性方程組的含義不僅是二元一次方程,更可以拓展到多元一次(綫性)方程:
    $$\begin{cases}
        a_{11}x_1+a_{12}x_2+\cdots+a_{1n}x_{n}&=b_1\\
        a_{21}x_1+a_{22}x_2+\cdots+a_{2n}x_{n}&=b_2\\
        \vdots&\vdots\\
        a_{m1}x_1+a_{m2}x_2+\cdots+a_{mn}x_{n}&=b_m
    \end{cases}$$

    \begin{example}
        以下是綫性方程組的例子:
        \begin{enumerate}
            \item $\begin{cases}
                x+y&=0\\
                2x-3y&=0
            \end{cases}$
            \item $\begin{cases}
                x+y+z&=0\\
                2x+3y+4z&=0
            \end{cases}$
            \item $\begin{cases}
                x-y+z&=0\\
                2x+4z&=0\\
                -2x-4y&=0
            \end{cases}$
        \end{enumerate}
    \end{example}

    \subsection*{解綫性方程組}

    欲求綫性方程組的解,我們可使\textbf{代入法}或\textbf{消元法},其中消元法比代入法的效率更高,因此普遍數學家都會使用消元法解方程。同時此辦法也衍生出矩陣的各項命題。

    \begin{example}
        求解綫性方程組$\begin{cases}
            2x+3y&=8\\
            6x-2y&=9
        \end{cases}$
        \begin{enumerate}
            \item 運用代入法求解:從$2x+3y=8$可得$x=\dfrac{8-3y}{2}$,代入$6x-2y=9$得\begin{align*}
                6(\frac{8-3y}{2})-2y&=9\\
                3(8-3y)-2y&=9\\
                24-9y-2y&=9\\
                11y&=15\\
                y&=\frac{15}{11}
            \end{align*}
            再代$y=\dfrac{15}{11}$入$x=\dfrac{8-3y}{2}$得\begin{align*}
                x&=\frac{8-3(\frac{15}{11})}{2}\\
                &=\frac{43}{22}
            \end{align*}
            因此綫性方程組的解為$(\dfrac{43}{22},\dfrac{15}{11})$。
            \item 運用消元法求解:從$2x+3y=8$三倍後可得$6x+9y=24$,則上式減去下式可得\begin{align*}
                &(1)\times 3:&6x+9y&=24\\
                -)&(2):&6x-2y&=9\\
                \hline
                &(1)\times 3 - (2):&11y&=15\\
                &&y&=\frac{15}{11}\\
                &&6x-2(\frac{15}{11})&=9\\
                &&x&=\frac{43}{22}
            \end{align*}
        \end{enumerate}
    \end{example}

    \begin{example}
        求解綫性方程組$\begin{cases}
            2x+3y+4z&=1\\
            3x-y+3z&=0\\
            x+y+z&=1
        \end{cases}$\begin{enumerate}
            \item 運用代入法求解:先從$3x-y+3z=0$得出$y=3x+3z$,代入其餘兩式可得$$\begin{cases}
                2x+3(3x+3z)+4z&=1\\
                x+(3x+3z)+z&=1
            \end{cases}\implies\begin{cases}
                11x+13z&=1\\
                4x+4z&=1
            \end{cases}$$
            再從$4x+4z=1$得$z=\dfrac{1-4x}{4}$,代入$11x+13z=1$得\begin{align*}
                11x+13(\frac{1-4x}{4})&=1\\
                44x+13-52x&=4\\
                -8x&=-9\\
                x&=\frac{9}{8}
            \end{align*}
            由此可得:\begin{align*}
                z&=\frac{1-4(\frac{9}{8})}{4}\\ &=-\frac{7}{8}\\
                y&=3(\frac{11}{4})+3(-\frac{5}{2})\\ &=\frac{3}{4}
            \end{align*}
            \item 運用消元法求解:\begin{align*}
                &(1):&2x+3y+4z&=1\\
                &(2):&3x-y+3z&=0\\
                &(3):&x+y+z&=1\\
                \hline
                &(1)-(3)\times 2:&y+2z&=-1\\
                &(2)-(3)\times 3:&-4y&=-3\\
                \hline
                &&y&=\frac{3}{4}\\
                &&z&=-\frac{7}{8}\\
                &&x&=\frac{9}{8}
            \end{align*}
        \end{enumerate}
    \end{example}

    我們亦容許綫性方程組\textbf{沒有解}或\textbf{有無限解}。若要理解沒有解或有無限多個解,可參考直綫的特性:

    \begin{definition}[平行綫]
        設$L_1$和$L_2$為兩條直綫,並分別以$m_1$和$m_2$代表其斜率。若$L_1//L_2$,則$m_1=m_2$。
    \end{definition}

    \begin{theorem}[平行綫的交點數]
        設$L_1$和$L_2$為一對(歐幾里得幾何)平行綫,則$L_1$和$L_2$的交點數只能為$0$或$\infty$。
    \end{theorem}
    
    \begin{proof}
        設$m$爲$L_1$及$L_2$的斜率,及$c_1$和$c_2$分別爲$L_1$及$L_2$的縱軸截距,則$$L_1:y=mx+c_1,\,\,L_2:y=mx+c_2$$

        若$c_1=c_2$,則$\forall x$, 若 $(x,y_1)\in L_1, (x,y_2)\in L_2$, $$y_1=mx+c_1=mx+c_2=y_2$$
        由於對任意$x$均成立,此爲無限多交點。

        若$c_1\neq c_2 \implies c_1-c_2\neq 0$,則$\forall x$, 若 $(x,y_1)\in L_1, (x,y_2)\in L_2$, $$y_2-y_1=(mx+c_1)-(mx+c_2)=c_1-c_2\neq 0 \implies y_1\neq y_2$$
        因此沒有交點。
    \end{proof}

    \begin{corollary}
        任何一對歐幾里得二維直綫可有$0$,$1$或$\infty$交點。
    \end{corollary}

    因此,綫性方程組可有$0$,$1$或無限多解。同時,綫性方程組若有兩個解,則會有無限多解。

    \begin{theorem}
        設$A\in\mathbb{M}^{m\times n}, \mathbf{b}\in\mathbb{M}^{m\times 1}$使得$(S):[\mathbf{A}|\mathbf{b}]$為一綫性方程組,並設$\mathbf{x},\mathbf{y}\in\mathbb{M}^{m\times 1}$為相異解,則$$I:=\{t\mathbf{x}+(1-t)\mathbf{y}:t\in\mathbb{R}\}$$為方程組的解集。
    \end{theorem}
    
    \begin{proof}
        對於任意$1\leq k\leq m$,\begin{align*}
            &a_{k1}[tx_1+(1-t)y_1]+\cdots+a_{kn}[tx_n+(1-t)y_n]\\
            &=t[a_{k1}x_1+\cdots+a_{kn}x_n]+(1-t)[a_{k1}y_1+\cdots+a_{kn}y_n]\\
            &=tb_k+(1-t)b_k\\
            &=b_k
        \end{align*}
    \end{proof}

    我們稱擁有無限多解的方程組的解集為\textbf{通解}。

    \begin{example}

        以下爲綫性方程組無限多解的例子:
        \begin{enumerate}
            \item 考慮$\begin{cases}
                3x+y=2\\-3x-y=-2
            \end{cases}$,由於任何符合$3x+y=2$的點均符合$-3x-y=-2$,因此此方程組有無限多解。解集為$$\{(t,2-3t):t\in\mathbb{R}\}$$
            \item 考慮$\begin{cases}
                3x+y+2z=0\\x+y+7z=0\\2x-5z=0
            \end{cases}$,由於上式減中式等於下式,因此此方程組無法解下去,有無限多解。解集為$$\{(5t,-19t,2t):t\in\mathbb{R}\}$$
        \end{enumerate}

        
        以下爲綫性方程組沒有解的例子:\begin{enumerate}
            \item 考慮$\begin{cases}
                3x+y=2\\-3x-y=0
            \end{cases}$,由於任何符合$3x+y=2$的點均不符合$-3x-y=0$,因此此方程組沒有解。解集為$$\emptyset$$
        \end{enumerate}
    \end{example}

    \subsection*{綫性方程組的可解性I}

    現記綫性方程組的通常式為
    $$\begin{cases}
        a_{11}x_1+a_{12}x_2+\cdots+a_{1n}x_{n}&=b_1\\
        a_{21}x_1+a_{22}x_2+\cdots+a_{2n}x_{n}&=b_2\\
        \vdots&\vdots\\
        a_{m1}x_1+a_{m2}x_2+\cdots+a_{mn}x_{n}&=b_m
    \end{cases}$$
    
    \begin{theorem}[綫性方程組的可解性I]
        對於$m$行$n$列綫性方程組$S$,若$m\leq n$,則$S$有解。
    \end{theorem}
    
    \begin{proof}
        當m=1時,$a_{11}x_1+a_{12}x_2+\cdots+x_{1n}x_{n}=b_1$,則存在綫性函數$f$使得$x_1=f(x_2,x_3,\dots,x_n)$令$a_{11}x_1+a_{12}x_2+\cdots+x_{1n}x_{n}=b_1$成立。

        當$1<m<n$時,已知有函數$f_1,f_2,f_3,\dots,f_{m-1}$使得\begin{align*}
            x_1&=f_1(x_{m+1}, \dots, x_n)\\
            x_2&=f_2(x_{m+1}, \dots, x_n)\\
            \vdots&=\vdots\\
            x_{m}&=f_{m}(x_{m+1}, \dots, x_n)\\
        \end{align*}令$\begin{cases}
            a_{11}x_1+a_{12}x_2+\cdots+a_{1n}x_{n}&=b_1\\
            a_{21}x_1+a_{22}x_2+\cdots+a_{2n}x_{n}&=b_2\\
            \vdots&\vdots\\
            a_{m1}x_1+a_{m2}x_2+\cdots+a_{mn}x_{n}&=b_m
        \end{cases}$成立。

        則$m+1\leq n$時,存在綫性函數$f_{m+1}$使得\begin{align*}
            x_{m+1}&=f_{m+1}(x_1,x_2,x_3,\dots,x_{m},x_{m+2},\dots,x_n)\\
            &=f_{m+1}(f_1(x_{m+1}, \dots, x_n),\dots,f_{m}(x_{m+1}, \dots, x_n),x_{m+2},\dots,x_n)
        \end{align*}
        則存在綫性函數$F$和$g_{m+1}$使得\begin{align*}
            x_{m+1}&=F(x_{m+1},x_{m+2},\dots,x_n)\\
            &=g_{m+1}(x_{m+2},\dots,x_n)
        \end{align*}
        因此可見存在綫性函數$g_1,g_2,\dots,g_m$(若$m+2>n$視所有$g_k$為常函數)\begin{align*}
            x_1&=f_1(g_{m+1}(x_{m+2},\dots,x_n),x_{m+2}, \dots, x_n)\\
            &=g_1(x_{m+2}, \dots, x_n)\\
            x_2&=f_2(g_{m+1}(x_{m+2},\dots,x_n),x_{m+2}, \dots, x_n)\\
            &=g_2(x_{m+2}, \dots, x_n)\\
            \vdots&=\vdots\\
            \vdots&=\vdots\\
            x_{m}&=f_{m}(g_{m+1}(x_{m+2},\dots,x_n),x_{m+2}, \dots, x_n)\\&=g_m(x_{m+2}, \dots, x_n)\\
        \end{align*}
        此證畢方程解的存在性。
    \end{proof}

    目前只有存在性能證明,因爲:\begin{enumerate}
        \item 未能確定是否有多於一個解;
        \item 當$m>n$時,未知是否有解。
    \end{enumerate}

    下一節會解決上述問題,判斷解的唯一性。

    \newpage

    \section*{增廣矩陣}

    爲了使方程組的觀察更方便及易於處理,在此階段引入\textbf{增廣矩陣}的概念。

    考慮以下綫性方程組:$$(S):\begin{cases}
        a_{11}x_1+a_{12}x_2+\cdots+a_{1n}x_{n}&=b_1\\
        a_{21}x_1+a_{22}x_2+\cdots+a_{2n}x_{n}&=b_2\\
        \vdots&\vdots\\
        a_{m1}x_1+a_{m2}x_2+\cdots+a_{mn}x_{n}&=b_m
    \end{cases}$$
    考慮$\mathbf{A}:=\begin{bmatrix}
        a_{11}&a_{12}&\cdots&a_{1n}\\
        a_{21}&a_{22}&\cdots&a_{2n}\\
        \vdots&\vdots&\ddots&\vdots\\
        a_{m1}&a_{m2}&\cdots&a_{mn}
    \end{bmatrix}\in\mathbb{M}^{m\times n}$為\textbf{係數矩陣}及$\mathbf{b}=\begin{bmatrix}
        b_1\\b_2\\\vdots\\b_m
    \end{bmatrix}$為\textbf{水平向量},則可將$(S)$作以下簡化表達:$$(S):[\mathbf{A}\vert \mathbf{b}]=\left[
        \begin{matrix}
            a_{11}&a_{12}&\cdots&a_{1n}\\
            a_{21}&a_{22}&\cdots&a_{2n}\\
            \vdots&\vdots&\ddots&\vdots\\
            a_{m1}&a_{m2}&\cdots&a_{mn}
        \end{matrix}
        \left|
          \,
          \begin{matrix}
            b_1\\b_2\\\vdots\\b_m
          \end{matrix}
        \right.
      \right]$$

    以上$[\mathbf{A}|\mathbf{b}]$稱爲\textbf{增廣矩陣}。增廣矩陣的概念在於簡化方程組的表達,因此方程組的運作均可在擴增矩陣中實現。如此引入以下概念:

    \begin{definition}[列階梯形矩陣]
        設$A\in\mathbb{M}^{m\times n}$,若$A$符合以下條件,則稱$A$為\textbf{列階梯形矩陣}:\begin{itemize}
            \item 若某列有非零元素,則必在所有全零列之上;
            \item 某列最左邊的非零元素稱爲\textbf{首項係數}。某列的首項係數必定比上一列的首項係數更靠右。
        \end{itemize}
    \end{definition}

    \begin{example}
        列階梯形矩陣的例子:
        $$\left[
        \begin{matrix}
            1&a_{12}&\cdots&a_{1(n-1)}&a_{1n}\\
            0&2&\cdots&a_{2(n-1)}&a_{2n}\\
            0&0&\cdots&0&a_{3n}\\
            0&0&\cdots&0&0
        \end{matrix}
        \left|
          \,
          \begin{matrix}
            b_1\\b_2\\b_3\\0
          \end{matrix}
        \right.
      \right]$$
    \end{example}

    \begin{definition}[還原列梯矩陣]
        設$A\in\mathbb{M}^{m\times n}$,若$A$符合以下條件,則稱$A$為\textbf{還原列梯矩陣}:\begin{itemize}
            \item $\mathbf{A}$為列階梯形矩陣,并且;
            \item $\mathbf{A}$的所有首項係數均爲1;
            \item $\mathbf{A}$中所有首項係數的對應行僅有一個非零數字。
        \end{itemize}
    \end{definition}

    \begin{example}
        還原列梯矩陣的例子:
        $$\left[
        \begin{matrix}
            1&0&\cdots&a_{1(n-1)}&0\\
            0&1&\cdots&a_{2(n-1)}&0\\
            0&0&\cdots&0&1\\
            0&0&\cdots&0&0
        \end{matrix}
        \left|
          \,
          \begin{matrix}
            b_1\\b_2\\b_3\\0
          \end{matrix}
        \right.
      \right]$$
    \end{example}

    已知綫性方程組能通過消元法求解,由此可見還原列梯矩陣正是消元法所形成的結果。我們給出以下定義:

    \begin{definition}[初等行(列)變換]
        下列三種對矩陣的變換稱爲矩陣的\textbf{初等行(列)變換}:\begin{enumerate}
            \item 把第$i$行(列)和第$j$行(列)互換位置,記爲$r_i\leftrightarrow r_j$($c_i\leftrightarrow c_j$);
            \item 用非零數$\alpha$乘以第$i$行(列),記爲$\alpha r_i$($\alpha c_i$);
            \item 把第$i$行(列)的$\alpha$倍加到第$j$行(列),記爲$\alpha r_i + r_j$($\alpha c_i + c_j$)。
        \end{enumerate}
    \end{definition}

    \begin{definition}
        設$\mathbf{A},\mathbf{A}'\in\mathbb{M}^{m\times n}$。
        
        我們稱$\mathbf{A}$和$\mathbf{A}'$是\textbf{等價}的若$\mathbf{A}$可通過列運算達成$\mathbf{A}'$,記$\mathbf{A}\sim\mathbf{A}'$;
        若$\mathbf{A}'$為還原列階矩陣而且$\mathbf{A}\sim\mathbf{A}'$,則稱$\mathbf{A}'$為$\mathbf{A}$的還原式,記$\mathbf{A}'=\rref(\mathbf{A})$。
    \end{definition}

    \begin{example}
        設$\mathbf{A}=\left[\begin{matrix}
            6& 9& 3& 3& 9\\
            1& 5& 4& 0& 1\\
            0& 7& 3& 2& 6\\
            8& 1& 1& 2& 3\\
            0& 2& 7& 3& 9\\
            7& 5& 6& 0& 0
        \end{matrix}
        \left|
            \,
            \begin{matrix}
            3\\9\\6\\4\\2\\2
            \end{matrix}
        \right.
        \right]$,求$\rref(\mathbf{A})$。

        先將第一行的首項係數求出,由於第二行的首項係數為1,因此我們可以通過簡單的行變換造出適當的等價矩陣。

        \begin{align*}
            \left[\begin{matrix}
                6& 9& 3& 3& 9\\
                1& 5& 4& 0& 1\\
                0& 7& 3& 2& 6\\
                8& 1& 1& 2& 3\\
                0& 2& 7& 3& 9\\
                7& 5& 6& 0& 0
            \end{matrix}
            \left|
                \,
                \begin{matrix}
                3\\9\\6\\4\\2\\2
                \end{matrix}
            \right.
            \right]&\overset{\begin{matrix}
                \frac{1}{3}r_1\\
                r_1\leftrightarrow r_2
            \end{matrix}}{\sim}\left[\begin{matrix}
                1& 5& 4& 0& 1\\
                2& 3& 1& 1& 3\\
                0& 7& 3& 2& 6\\
                8& 1& 1& 2& 3\\
                0& 2& 7& 3& 9\\
                7& 5& 6& 0& 0
            \end{matrix}
            \left|
                \,
                \begin{matrix}
                9\\1\\6\\4\\2\\2
                \end{matrix}
            \right.
            \right]\overset{\begin{matrix}
                -2r_1 + r_2\\ -8r_1 + r_4\\ -7r_1+r_6
            \end{matrix}}{\sim}\left[\begin{matrix}
                1& 5& 4& 0& 1\\
                0& -7& -7& 1& 1\\
                0& 7& 3& 2& 6\\
                0& -39& -31& 2& -5\\
                0& 2& 7& 3& 9\\
                0& -30& -22& 0& -7
            \end{matrix}
            \left|
                \,
                \begin{matrix}
                9\\-17\\6\\-68\\2\\-61
                \end{matrix}
            \right.
            \right]
        \end{align*}
        隨後將第二及其餘行的首項係數求出:
        \begin{align*}
            \overset{\begin{matrix}
                r_3+r_2\\20r_5+r_4\\15r_5+r_6
            \end{matrix}}{\sim}\left[\begin{matrix}
                1& 5& 4& 0& 1\\
                0& 0& -4& 3& 7\\
                0& 7& 3& 2& 6\\
                0& 1& 109& 62& 175\\
                0& 2& 7& 3& 9\\
                0& 0& 83& 45& 128
            \end{matrix}
            \left|
                \,
                \begin{matrix}
                9\\-11\\6\\-28\\2\\-31
                \end{matrix}
            \right.
            \right]
            &\overset{\begin{matrix}
                -5r_4+r_1\\-7r_4 + r_3\\ -2r_4 + r_5
            \end{matrix}}{\sim}\left[\begin{matrix}
                1& 0& -541& -310& -874\\
                0& 0& -4& 3& 7\\
                0& 0& -760& -432& -1219\\
                0& 1& 109& 62& 175\\
                0& 0& -211& -121& -341\\
                0& 0& 83& 45& 128
            \end{matrix}
            \left|
                \,
                \begin{matrix}
                149\\-11\\174\\-28\\58\\-31
                \end{matrix}
            \right.
            \right]\\
            \overset{\begin{matrix}
                -r_2\\r_2\leftrightarrow r_4
            \end{matrix}}{\sim}\left[\begin{matrix}
                1& 0& -541& -310& -874\\
                0& 1& 109& 62& 175\\
                0& 0& -760& -432& -1219\\
                0& 0& 4& -3& -7\\
                0& 0& -211& -121& -341\\
                0& 0& 83& 45& 128
            \end{matrix}
            \left|
                \,
                \begin{matrix}
                149\\-28\\174\\11\\58\\-31
                \end{matrix}
            \right.
            \right]
            &\overset{\begin{matrix}
                135r_4+r_1\\-27r_4+r_2\\190r_4+r_2\\53r_4+r_5\\-20r_4+r_6
            \end{matrix}}{\sim}\left[\begin{matrix}
                1& 0& -1& -715& -1819\\
                0& 1& 1& 143& 364\\
                0& 0& 0& -1002& -2549\\
                0& 0& 4& -3& -7\\
                0& 0& 1& -280& -712\\
                0& 0& 3& 105& 268
            \end{matrix}
            \left|
                \,
                \begin{matrix}
                1634\\-325\\2264\\11\\641\\-251
                \end{matrix}
            \right.
            \right]\\
            \sim\cdots
            &\overset{\begin{matrix}
                r_5+r_1\\-r_5+r_2\\-4r_5+r_4\\-3r_5+r_6
            \end{matrix}}{\sim}\left[\begin{matrix}
                1& 0& 0& 0& 0\\
                0& 1& 0& 0& 0\\
                0& 0& 1& 0& 0\\
                0& 0& 0& 1& 0\\
                0& 0& 0& 0& 1\\
                0& 0& 0& 0& 0
            \end{matrix}
            \left|
                \,
                \begin{matrix}
                -\frac{148}{551}\\\frac{717}{551}\\\frac{895}{551}\\\frac{3975}{551}\\-\frac{2058}{551}\\-\frac{6817}{551}
                \end{matrix}
            \right.
            \right]
        \end{align*}
        詳細步驟參考RREF calculator: https://www.emathhelp.net/en/calculators/linear-algebra/reduced-row-echelon-form-rref-calculator/
    \end{example}

    \begin{remark}
        上例可見最後一行的左方為零行,但右方為非零數字,得到$0=1$的結論。此情形我們稱方程組無解。
    \end{remark}

    重新考慮綫性方程組的樣式,因應矩陣乘法的特性,我們可以將$S=[\mathbf{A}|\mathbf{b}]$寫作$$\mathbf{A}\mathbf{x}=\mathbf{b}$$若存在$\mathbf{x_0}\in\mathbb{M}^{n\times 1}$使得$\mathbf{Ax_0}=\mathbf{b}$成立,則稱$\mathbf{x_0}$為$\mathbf{A}\mathbf{x}=\mathbf{b}$的一個解。

    \begin{theorem}
        綫性方程組$S=[\mathbf{A}|\mathbf{b}]$有解當且僅當$\rref(S)=[\mathbf{A}'|\mathbf{b}']$有解;並且$S$與$\rref(S)$同解。
    \end{theorem}

    \begin{proof}
        由於$S$與$\rref(S)$等價,故存在若干行變換及對應的變換矩陣$\mathbf{P}$使得$$\mathbf{Ax_0}=\mathbf{b}\iff \mathbf{PAx_0}=\mathbf{Pb}\iff \mathbf{A'x_0}=\mathbf{b'}$$
    \end{proof}

    故本節以下列定理總結:

    \begin{theorem}[綫性方程組的可解性II]
        對於$m$行$n$列綫性方程組$S$,若$\rref(S)$中係數矩陣的非零行數$\leq n$同時大於或等於水平向量的非零行數,則$S$有解。
    \end{theorem}
    \newpage

    \section*{行列式}

    行列式為矩陣運算中一種重要的工具,其概念起源於解二元一次方程組:\begin{align*}
        \begin{cases}
            ax+by=\alpha\\cx+dy=\beta
        \end{cases}&\implies\begin{cases}
            x=\frac{\alpha d-\beta b}{ad-bc}\\y=\frac{\alpha c-\beta a}{bc-ad}
        \end{cases}
    \end{align*}

    從上可見$ad-bc$的值會完全影響方程的可解性,由此可定義係數矩陣$\mathbf{A}=\begin{bmatrix}
        a&b\\c&d
    \end{bmatrix}\in\mathbb{M}^{2\times 2}$時,$\mathbf{A}$的行列式$\det\mathbf{A}=|\mathbf{A}|=ad-bc$。

    \begin{proposition}
        設$\mathbf{A}=\begin{bmatrix}
            a&b&c\\d&e&f\\g&h&i
        \end{bmatrix}\in\mathbb{M}^{3\times 3}$,則$|A|=aei+bfg+cdh-ceg-bdi-afh$.
    \end{proposition}

    \begin{definition}[分塊矩陣]
        任何矩陣$\mathbf{A}\in\mathbb{M}^{m\times n}$,都有$\mathbf{A}_1\in\mathbb{M}^{s\times t},\mathbf{A}_2\in\mathbb{M}^{s\times (n-t)},\mathbf{A}_3\in\mathbb{M}^{(m-s)\times t}, \mathbf{A}_4\in\mathbb{M}^{(m-s)\times (n-t)}$使得$$\mathbf{A}=\begin{bmatrix}
            \mathbf{A}_1&\mathbf{A}_2\\\mathbf{A}_3&\mathbf{A}_4
        \end{bmatrix}$$
    \end{definition}
    
    \begin{theorem}
        
    \end{theorem}

    \section*{逆矩陣}

    設$\mathbf{A}\in\mathbb{M}^{m\times n}, \mathbf{x},\mathbf{b}\in\mathbb{M}^{n\times 1}$,設$$\mathbf{A}\mathbf{x}=\mathbf{b}$$若$\rref(\mathbf{A})=\mathbf{I}$,則存在$\mathbf{B}\in\mathbb{M}^{m\times n}$令$$[\mathbf{A}|\mathbf{I}]\sim[\mathbf{I}|\mathbf{B}]$$對符合條件的矩陣$\mathbf{B}$,我們稱其爲\textbf{$\mathbf{A}$的逆矩陣},記$\mathbf{B}=\mathbf{A}^{-1}$。

    \section*{矩陣的幾何含義}

    \section*{博弈論簡介}
\end{document}