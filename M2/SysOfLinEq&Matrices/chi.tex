\documentclass[12pt]{article}
\usepackage{ctex}
\usepackage[english]{babel}
\usepackage{blindtext}
\usepackage{nameref}
\usepackage{fancyhdr}
\usepackage{amsmath,amssymb,amsthm}
\usepackage{graphicx,float}
\usepackage{physics}
\usepackage{pgfplots}
\usepackage[a4paper, total={6in, 9in}]{geometry}

\graphicspath{{../image/}}

\pagestyle{fancy}
\fancyhf{}
\fancyhf[HL]{矩陣}
\fancyhf[CF]{\thepage}

\newcommand{\innerprod}[2]{\langle{#1},{#2}\rangle}
\newcommand{\id}{\mathtt{id}}

\newtheorem{definition}{定義}
\newtheorem*{theorem}{定理}
\newtheorem*{corollary}{衍理}
\newtheorem*{lemma}{引理}
\newtheorem*{proposition}{設理}
\newtheorem*{remark}{小記}
\newtheorem*{claim}{主張}
\newtheorem*{example}{例子}
\newtheorem*{axiom}{公設}
\renewenvironment*{proof}{\textit{證明.}}{\hfill$\qed$}

\newenvironment*{sol}{\par \textbf{解}.}{\hfill$\blacksquare$}

\begin{document}
    \section*{矩陣的定義}

    \begin{definition}[矩陣]
        \textbf{矩陣}是代數學中一種特殊的表達形式,用以同時表達多位元的表格:$$\mathbf{A}=(a_{ij})_{m\times n}=\begin{pmatrix}
            a_{11}&a_{12}&\cdots&a_{1n}\\
            a_{21}&a_{22}&\cdots&a_{2n}\\
            \vdots&\vdots&\ddots&\vdots\\
            a_{m1}&a_{m2}&\cdots&a_{mn}
        \end{pmatrix}$$
        若\textbf{元素}$a_{ij}\in S$則稱爲\textbf{$S$上的$m\times n$矩陣},記$\mathbf{A}\in \mathbb{M}^{m\times n}$。
    \end{definition}

    \begin{definition}[矩陣的加法與乘法定義]
        給定$\mathbf{A},\mathbf{B}\in\mathbb{M}^{m\times n}$,$\mathbf{C}\in\mathbb{M}^{n\times t}$,則\begin{enumerate}
            \item 加法:$\mathbf{A}+\mathbf{B}=(a_{ij}+b_{ij})$。
            \item 數乘:$k\mathbf{A}=(ka_{ij})$。
            \item 矩陣乘法:$\mathbf{A}\times\mathbf{B}=(\sum_{k=1}^{n}a_{ik}b_{kj})$。
        \end{enumerate}
    \end{definition}

    \begin{example}
        設$\mathbf{A}=\begin{pmatrix}
            1&1&0\\1&0&1
        \end{pmatrix},\mathbf{B}=\begin{pmatrix}
            2&1&0\\1&2&1
        \end{pmatrix},\mathbf{C}=\begin{pmatrix}
            2&0\\2&1\\3&3
        \end{pmatrix}$,則\begin{enumerate}
            \item $\mathbf{A}+\mathbf{B}=\begin{pmatrix}
                1+2&1+1&0+0\\1+1&0+2&1+1
            \end{pmatrix}=\begin{pmatrix}
                3&2&0\\2&2&2
            \end{pmatrix}$。
            \item $4\mathbf{A}=\begin{pmatrix}
                4&4&0\\4&0&4
            \end{pmatrix}$。
            \item $\mathbf{A}\times\mathbf{C}=\begin{pmatrix}
                1\times2+1\times2+0\times3&1\times0+1\times1+0\times3\\1\times2+0\times2+1\times3&1\times0+0\times1+1\times3
            \end{pmatrix}=\begin{pmatrix}
                4&1\\5&3
            \end{pmatrix}$。
        \end{enumerate}
    \end{example}

    \begin{definition}[特殊矩陣]
        下列為一些矩陣寫法的共識:\begin{itemize}
            \item 零矩陣:$\mathbf{0}_{m\times n}=\mathbf{O}_{m\times n}=(0)_{m\times n}$。
            \item 一矩陣:$\mathbf{1}_{m\times n}=(1)_{m\times n}$。
            \item 單位矩陣:$\mathbf{I}_{m\times n}=(\delta_{ij})_{m\times n}$,其中$\delta_{ij}=\begin{cases}
                1 & i=j\\ 0 & i\neq j
            \end{cases}$。
        \end{itemize}
    \end{definition}

    \begin{proposition}[加法與數乘定則]
        設$\mathbf{A},\mathbf{B},\mathbf{C}\in\mathbb{M}^{m\times n}$,則\begin{enumerate}
            \item 加法結合律:$(\mathbf{A}+\mathbf{B})+\mathbf{C}=\mathbf{A}+(\mathbf{B}+\mathbf{C})$。
            \item 加法交換律:$\mathbf{A}+\mathbf{B}=\mathbf{B}+\mathbf{A}$。
            \item $\mathbf{A}+\mathbf{0}=\mathbf{A}$。
            \item $\mathbf{A}+(-\mathbf{A})=\mathbf{0}$。
            \item $a(\mathbf{A}+\mathbf{B})=a\mathbf{A}+\mathbf{B}$。
            \item $(a+b)\mathbf{A}=a\mathbf{A}+b\mathbf{A}$。
            \item $(ab)\mathbf{A}=a(b\mathbf{A})$
        \end{enumerate}
    \end{proposition}

    \begin{proposition}[矩陣乘法定則]
        設$\mathbf{A},\mathbf{B},\mathbf{C},\mathbf{D}$為矩陣,則\begin{enumerate}
            \item 乘法結合律:$(\mathbf{A}\mathbf{B})\mathbf{C}=\mathbf{A}(\mathbf{B}\mathbf{C})$。
            \item 分配律1:$(\mathbf{A}+\mathbf{B})\mathbf{C}=\mathbf{A}\mathbf{C}+\mathbf{B}\mathbf{C}$。
            \item 分配律2:$\mathbf{D}(\mathbf{A}+\mathbf{B})=\mathbf{D}\mathbf{A}+\mathbf{D}\mathbf{B}$。
            \item $\alpha\mathbf{A}\mathbf{B}=(\alpha\mathbf{A})\mathbf{B}=\mathbf{A}(\alpha\mathbf{B})$。
            \item $\mathbf{A}\mathbf{I}=\mathbf{I}\mathbf{A}=\mathbf{A}$。
        \end{enumerate}
    \end{proposition}
    
    \begin{definition}[綫性組合]
        設$c_1,c_2,\dots,c_\ell$為常數,$\mathbf{A}_1,\mathbf{A}_2,\dots,\mathbf{A}_\ell\in\mathbb{M}_{m\times n}$,則稱$$\sum_{k=1}^{\ell}c_k\mathbf{A}_k=c_1\mathbf{A}_1+c_2\mathbf{A}_2+\cdots+c_\ell\mathbf{A}_\ell$$為\textbf{綫性組合}。
    \end{definition}

    \section*{綫性方程組}

    \section*{擴增矩陣}

    \section*{逆矩陣}

    \section*{行列式}

    \section*{矩陣的幾何含義}
\end{document}