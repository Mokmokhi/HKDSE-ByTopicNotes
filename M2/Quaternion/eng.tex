\documentclass[12pt]{article}
\usepackage{ctex}
\usepackage[english]{babel}
\usepackage{blindtext}
\usepackage{nameref}
\usepackage{fancyhdr}
\usepackage{amsmath,amssymb,amsthm}
\usepackage{graphicx,float}
\usepackage{physics}
\usepackage{pgfplots}
\usepackage[a4paper, total={6in, 9in}]{geometry}

\graphicspath{{../images/}}

\pagestyle{fancy}
\fancyhf{}
\fancyhf[LH]{Quaternion and its application}
\fancyhf[CF]{\thepage}

\newcommand{\id}{\mathtt{id}}

\newtheorem*{definition}{Definition}
\newtheorem*{theorem}{Theorem}
\newtheorem*{corollary}{Corollary}
\newtheorem*{lemma}{Lemma}
\newtheorem*{proposition}{Proposition}
\newtheorem*{remark}{Remark}
\newtheorem*{claim}{Claim}
\newtheorem*{example}{Example}
\newtheorem*{axiom}{Axiom}

\begin{document}
    \section{Quaternion}

    \begin{definition}[Quaternion]
        Let $a,b,c,d\in\mathbb{R}$ and unit vectors $\mathbf{i},\mathbf{j},\mathbf{k}$ be pointing positively along 3 spatial axes such that a \textbf{quaternion} $q$ can be written in the form of $$a+b\mathbf{i}+c\mathbf{j}+d\mathbf{k}$$

        The scalar part of $q$ is denoted by $\Re{q}:=a$ and the vector part of $q$ is denoted by $\Im{q}:=b\mathbf{i}+c\mathbf{j}+d\mathbf{k}$. The space of quaternion is denoted by $\mathbb{H}$, called the \textbf{Hamiltonian}.
    \end{definition}

    To type it simple, I may sometimes denote $a+b\mathbf{i}+c\mathbf{j}+d\mathbf{k}$ as $(a,b,c,d)$. For readability, it may sometimes be a column vector $\begin{pmatrix}
        a\\b\\c\\d
    \end{pmatrix}$.

    \begin{definition}[Quaternion Arithmetic]
        Let $p:=(p_0,p_1,p_2,p_3),q:=(q_0,q_1,q_2,q_3)\in\mathbb{H}$. Then the arithmetic of quaternion is defined by\begin{itemize}
            \item Addition: $p+q:=(p_0+q_0,p_1+q_1,p_2+q_2,p_3+q_3)$;
            \item Scalar multiplication: $\lambda p:=(\lambda p_0,\lambda p_1, \lambda p_2, \lambda p_3)$ for $\lambda\in\mathbb{R}$.
            \item $\mathbf{i}\mathbf{j}=-\mathbf{j}\mathbf{i}=\mathbf{k}$, $\mathbf{j}\mathbf{k}=-\mathbf{k}\mathbf{j}=\mathbf{i}$, $\mathbf{k}\mathbf{i}=-\mathbf{i}\mathbf{k}=\mathbf{j}$, $\mathbf{i}\mathbf{j}\mathbf{k}=-1$.
        \end{itemize}
    \end{definition}

    \begin{proposition}[Identity element]
        $1\in\mathbb{H}$ is the only identity element.
    \end{proposition}

    \begin{proof}
        Suppose the identity element $e\in\mathbb{H}$ is in the form of $(e_0,e_1,e_2,e_3)$. Then \begin{align*}
            (q_0,q_1,q_2,q_3)(e_0,e_1,e_2,e_3)&=\begin{pmatrix}
                q_0e_0-q_1e_1-q_2e_2-q_3e_3\\
                q_0e_1+q_1e_0+q_2e_3-q_3e_2\\
                q_0e_2+q_2e_0+q_3e_1-q_1e_3\\
                q_0e_3+q_3e_0+q_1e_2-q_2e_1
            \end{pmatrix}
            =\begin{pmatrix}
                q_0\\q_1\\q_2\\q_3
            \end{pmatrix}
        \end{align*}

        Solving equation yields $e_0=1$ and $e_1=e_2=e_3=0$.
    \end{proof}

    \begin{definition}[Modulus]
        The modulus of a quaternion $q=(q_0,q_1,q_2,q_3)\in\mathbb{H}$ is defined as $$|q|:=\sqrt{q_0^2+q_1^2+q_2^2+q_3^2}$$
    \end{definition}

    \begin{proposition}[Conjugation]
        For $q=(q_0,q_1,q_2,q_3)$, the algebraic conjugation $\bar{q}$ is $$(q_0,-q_1,-q_2,-q_3)$$
    \end{proposition}

    \begin{proof}
        Consider $\bar{q}:=(\bar{q}_0,\bar{q}_1,\bar{q}_2,\bar{q}_3)$, we have to have \begin{align*}
            (q_0,q_1,q_2,q_3)(\bar{q}_0,\bar{q}_1,\bar{q}_2,\bar{q}_3)&=\begin{pmatrix}
                q_0\bar{q}_0-q_1\bar{q}_1-q_2\bar{q}_2-q_3\bar{q}_3\\
                q_0\bar{q}_1+q_1\bar{q}_0+q_2\bar{q}_3-q_3\bar{q}_2\\
                q_0\bar{q}_2+q_2\bar{q}_0+q_3\bar{q}_1-q_1\bar{q}_3\\
                q_0\bar{q}_3+q_3\bar{q}_0+q_1\bar{q}_2-q_2\bar{q}_1
            \end{pmatrix}
            =\begin{pmatrix}
                q_0^2+q_1^2+q_2^2+q_3^2\\0\\0\\0
            \end{pmatrix}
        \end{align*}
        which yields $\bar{q_0}=q_0$, $\bar{q_1}=-q_1$, $\bar{q_2}=-q_2$, $\bar{q_3}=-q_3$.
    \end{proof}

    \section{Rubik's cube with quaternion}
\end{document}