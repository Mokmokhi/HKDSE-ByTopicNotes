\documentclass[12pt]{article}
\usepackage{ctex}
\usepackage[english]{babel}
\usepackage{blindtext}
\usepackage{nameref}
\usepackage{fancyhdr}
\usepackage{amsmath,amssymb,amsthm}
\usepackage{graphicx,float}
\usepackage{physics}
\usepackage{pgfplots}
\usepackage[a4paper, total={6in, 9in}]{geometry}

\graphicspath{{../image/}}

\pagestyle{fancy}
\fancyhf{}
\fancyhf[HL]{微分}
\fancyhf[CF]{\thepage}

\newcommand{\innerprod}[2]{\langle{#1},{#2}\rangle}
\newcommand{\id}{\mathtt{id}}

\newtheorem{definition}{定義}
\newtheorem*{theorem}{定理}
\newtheorem*{corollary}{衍理}
\newtheorem*{lemma}{引理}
\newtheorem*{proposition}{設理}
\newtheorem*{remark}{小記}
\newtheorem*{claim}{主張}
\newtheorem*{example}{例子}
\newtheorem*{axiom}{公設}
\renewenvironment*{proof}{\textit{證明.}}{\hfill$\qed$}

\newenvironment*{sol}{\par \textbf{解}.}{\hfill$\blacksquare$}

\begin{document}
    \section*{單元導數定義}

    由牛頓發揚光大的流數法,今時今日變成了以極限定義的導數。

    \begin{definition}[導數定義]
        若$f$在$c$可導,則其導數$f'(c)$為$$f'(c)=\lim_{x\to c}\frac{f(x)-f(c)}{x-c}$$
    \end{definition}

    \begin{definition}[現代導數嚴謹定義]
        $L$為$f$在$c$的導數當:對於任意$\epsilon>0$,若存在$\delta(\epsilon)>0$使得對於$0<|x-c|<\delta(\epsilon)$,$$|\frac{f(x)-f(c)}{x-c}-L|<\epsilon$$
        則寫$f'(c)=L$。
    \end{definition}
\end{document}