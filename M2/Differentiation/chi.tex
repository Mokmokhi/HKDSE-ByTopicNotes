\documentclass[12pt]{article}
\usepackage{ctex}
\usepackage[english]{babel}
\usepackage{blindtext}
\usepackage{nameref}
\usepackage{fancyhdr}
\usepackage{amsmath,amssymb,amsthm}
\usepackage{graphicx,float}
\usepackage{physics}
\usepackage{pgfplots}
\usepackage[a4paper, total={6in, 9in}]{geometry}

\graphicspath{{../image/}}

\pagestyle{fancy}
\fancyhf{}
\fancyhf[HL]{微分}
\fancyhf[CF]{\thepage}

\newcommand{\innerprod}[2]{\langle{#1},{#2}\rangle}
\newcommand{\id}{\mathtt{id}}

\newtheorem{definition}{定義}
\newtheorem*{theorem}{定理}
\newtheorem*{corollary}{衍理}
\newtheorem*{lemma}{引理}
\newtheorem*{proposition}{設理}
\newtheorem*{remark}{小記}
\newtheorem*{claim}{主張}
\newtheorem*{example}{例子}
\newtheorem*{axiom}{公設}
\renewenvironment*{proof}{\textit{證明.}}{\hfill$\qed$}

\newenvironment*{sol}{\par \textbf{解}.}{\hfill$\blacksquare$}

\begin{document}
    \section*{單元導數定義}

    由牛頓發揚光大的流數法,今時今日變成了以極限定義的導數。

    \begin{definition}[導數定義]
        若$f$在$c$可導,則其導數$f'(c)$為$$f'(c)=\lim_{x\to c}\frac{f(x)-f(c)}{x-c}$$
    \end{definition}

    \begin{definition}[現代導數嚴謹定義]
        $L$為$f$在$c$的導數當:對於任意$\epsilon>0$,若存在$\delta(\epsilon)>0$使得對於$0<|x-c|<\delta(\epsilon)$,$$|\frac{f(x)-f(c)}{x-c}-L|<\epsilon$$
        則寫$f'(c)=L$。
    \end{definition}

    \begin{theorem}
        若$f:I\to\mathbb{R}$在$c\in I$可微,則$f$在$c\in I$連續。
    \end{theorem}

    \begin{proof}
        導數存在,則\begin{align*}
            \lim_{x\to c}(f(x)-f(c))&=\lim_{x\to c}\frac{f(x)-f(c)}{x-c}\cdot \lim_{x\to c}(x-c)\\
            &=f'(c)\cdot 0\\
            &=0
        \end{align*}
        因此$\displaystyle\lim_{x\to c}f(x)=f(c)$使得$f$在$c\in I$連續。
    \end{proof}

    導數的目的在於處理函數的變化:從算式可見,導數取自$f$的變化除以變量$x$的變化,即可理解爲變化的比例,等於變化比率。簡而言之,導數為函數的斜率。

    \begin{theorem}[導數的性質]
        導數擁有以下性質:\begin{enumerate}
            \item 綫性性:對於連續函數$f,g$,常數$\alpha,\beta$,$(\alpha f+\beta g)'=\alpha f' + \beta g'$。
            \item 乘積法則:對於連續函數$f,g$,$(f\cdot g)'=f'\cdot g + f\cdot g'$。
            \item 除法定則:對於連續函數$f,g$,$(\frac{f}{g})'=\frac{f'\cdot g-f\cdot g'}{g^2}$。
        \end{enumerate}
    \end{theorem}

    \begin{theorem}[連鎖律]
        對於複合函數$h:=(g\circ f)$,$$h'=(g'\circ f)\cdot(f')$$
    \end{theorem}

    \begin{theorem}[逆函數]
        對於逆函數$g:=f^{-1}$,$$g'=\frac{1}{f'\circ g}$$
    \end{theorem}
\end{document}