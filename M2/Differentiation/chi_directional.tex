\documentclass[12pt]{article}
\usepackage{ctex}
\usepackage[english]{babel}
\usepackage{blindtext}
\usepackage{nameref}
\usepackage{fancyhdr}
\usepackage{amsmath,amssymb,amsthm}
\usepackage{graphicx,float}
\usepackage{physics}
\usepackage{pgfplots}
\usepackage[a4paper, total={6in, 9in}]{geometry}

\graphicspath{{../image/}}

\pagestyle{fancy}
\fancyhf{}
\fancyhf[HL]{微分幾何1:向量與導數}
\fancyhf[CF]{\thepage}

\newcommand{\innerprod}[2]{\langle{#1},{#2}\rangle}
\newcommand{\id}{\mathtt{id}}

\newtheorem{definition}{定義}
\newtheorem*{theorem}{定理}
\newtheorem*{corollary}{衍理}
\newtheorem*{lemma}{引理}
\newtheorem*{proposition}{命題}
\newtheorem*{remark}{小記}
\newtheorem*{claim}{主張}
\newtheorem*{example}{示例}
\newtheorem*{axiom}{公設}
\newtheorem*{exercise}{即時練習}
\renewenvironment*{proof}{\textit{證明.}}{\hfill$\qed$}

\newenvironment*{sol}{\par \textbf{解}.}{\hfill$\blacksquare$}

\begin{document}
    References: Introduction to Real Analysis (Bartle \& Sherbert), Thomas Calculus 12th Edition
    \section*{向量}

    向量屬於一種特殊的矩陣,通常用以表達\textbf{多維坐標}。

    \begin{definition}[向量]
        一個\textbf{$n$-維向量}包含$n$個元素,可視之爲\textbf{$n$-維空間}中的坐標,同時代表從原點指向該坐標的箭頭。
    \end{definition}
    
    爲方便描述,記$V_S$為帶有$S$域的元素的向量集合。

    \begin{definition}[向量加法]
        在向量集合$V_S$中,若$\vec{x}=(x_i)_i=(x_1,x_2,\dots),\vec{y}=(y_i)_i=(y_1,y_2,\dots)\in V_S$,則$$\vec{x}+\vec{y}:=(x_1+y_1,x_2+y_2,\dots)=(x_i+y_i)_i$$
    \end{definition}

    \begin{exercise}
        求以下向量和:\begin{enumerate}
            \item $(1,2)+(3,4)$;
            \item $(1,2,3,4)+(1,3,4,5)$;
            \item $(1,-1,-1,1,1,1)+(-1,1,1,-1,-1,-1)$.
        \end{enumerate}
    \end{exercise}

    \begin{definition}[標量乘法]
        在向量集合$V_S$中,若$\vec{x}=(x_i)_i=(x_1,x_2,\dots)\in V_S$,$\alpha \in S$,則$$\alpha \vec{x}:=(\alpha x_1,\alpha x_2,\dots)=(\alpha x_i)_i$$
    \end{definition}

    \begin{exercise}
        求以下標量積:\begin{enumerate}
            \item $3(1,2)$;
            \item $-7(1,3,4,5)$;
            \item $-1(-1,1,1,-1,-1,-1)$.
        \end{enumerate}
    \end{exercise}

    \begin{definition}[向量的量值]
        對於任意向量$\vec{v}$,其量值定義爲$|\vec{v}|$,代表其長度。
    \end{definition}
    \section*{向量空間}

    \begin{definition}[向量空間]
        設$V_S$為向量集合,且$\vec{x},\vec{y},\vec{z}\in V_S$, $\alpha, \beta \in S$。若$V_S$符合以下定理:\begin{itemize}
            \item 加法結合律:$(\vec{x}+\vec{y})+\vec{z}=\vec{x}+(\vec{y}+\vec{z})$。
            \item 加法交換律:$\vec{x}+\vec{y}=\vec{y}+\vec{z}$。
            \item 加法單位元:$\vec{0}\in V_S$使得$\vec{x}+\vec{0}=\vec{0}+\vec{x}=\vec{x}$。
            \item 加法逆:$\forall \vec{x}\in V_S$, 存在$\vec{y}\in V_S$使得$\vec{x}+\vec{y}=\vec{y}+\vec{x}=\vec{0}$。
            \item 標乘結合律:$\alpha(\beta \vec{x})=(\alpha\beta)\vec{x}$。
            \item 標乘單位元:$1\in S$使得$1\vec{x}=\vec{x}$。
            \item 分配律1:$\alpha(\vec{x}+\vec{y})=\alpha\vec{x}+\alpha\vec{y}$。
            \item 分配律2:$(\alpha+\beta)\vec{x}=\alpha\vec{x}+\beta\vec{x}$。
        \end{itemize}
    \end{definition}

    \begin{example}
        $\mathbb{R}$是一個向量空間。而且任何域也是向量空間。
    \end{example}

    \begin{example}
        在牛頓力學中討論力時,我們會以向量表示力的大小與方向。假設目前的討論僅限於平面(二維空間),並記施力點為原點$O$。
        \begin{figure}[H]
            \centering
            \includegraphics[scale=0.6]{Force.png}
        \end{figure}
        在上圖中可通過改變力量發生的先後次序來實現向量的平移,從而得出$$\vec{F_0}=\vec{F_1}+\vec{F_2}$$的關係式。
        又因二維向量可拆分爲水平向量及鉛垂向量兩個分量,故$$\vec{F_0}=(|\vec{F_1}|\cos{\theta}+|\vec{F_2}|\cos{\phi})\hat{i}+(|\vec{F_1}|\sin{\theta}+|\vec{F_2}|\sin{\phi})\hat{j}$$其中$\hat{i}$和$\hat{j}$分別代表水平單位向量及鉛垂單位向量。
    \end{example}

    欲考慮作功問題,我們定義以下計算方式

    \begin{definition}[點積/内積]
        兩向量$\vec{a},\vec{b}$的内積可定義爲$$\innerprod{\vec{a}}{\vec{b}}\equiv \vec{a}\cdot\vec{b}:=|\vec{a}||\vec{b}|\cos{\theta}$$其中$\theta$為$\vec{a},\vec{b}$之間的夾角。
    \end{definition}

    \begin{example}
        根據經典力學定義,作功方程爲$$W=\vec{F}\cdot\vec{s}$$其中$W$為作功純量,$\vec{F}$為施力向量,$\vec{s}$為位移向量。考慮内積定義,作功方程可寫成$$W=|\vec{F}||\vec{s}|\cos{\theta}$$其中$\theta$為向量之間的夾角。
    \end{example}

    \begin{theorem}[内積的性質]
        對於$\vec{a},\vec{b},\vec{c}$,\begin{enumerate}
            \item $\innerprod{0}{\vec{a}}=\innerprod{\vec{a}}{0}=0$;
            \item $\innerprod{\vec{a}}{\vec{a}}\geq 0$,$\innerprod{\vec{a}}{\vec{a}}=0$當且僅當$\vec{a}\equiv 0$;
            \item $\innerprod{\vec{a}}{x\vec{b}+z\vec{c}}=x\innerprod{\vec{a}}{\vec{b}}+z\innerprod{\vec{a}}{\vec{c}}$;
            \item 若$\vec{a},\vec{b}\in\mathbb{R}^n$,則$\innerprod{\vec{a}}{\vec{b}}=\innerprod{\vec{b}}{\vec{a}}$。
        \end{enumerate}
    \end{theorem}

    \begin{proposition}
        向量$\vec{a},\vec{b}$之間的夾角為$$\theta=\arccos{\frac{\vec{a}\cdot\vec{b}}{|\vec{a}||\vec{b}|}}$$
    \end{proposition}

    另外,若$\vec{a}$正交於$\vec{b}$(在$\mathbb{R}^2$為互相垂直),$\vec{a},\vec{c}$平行,則根據定義$$\vec{a}\cdot\vec{b}=|\vec{a}||\vec{b}|\cos{90^\circ}=0$$及$$\vec{a}\cdot\vec{c}=|\vec{a}||\vec{c}|\cos{0^\circ}=|\vec{a}||\vec{c}|$$
    因此$$|\vec{a}|=\sqrt{|\vec{a}|^2}=\sqrt{\innerprod{\vec{a}}{\vec{a}}}$$
    \begin{definition}[基與正交基與標準正交基]
        對於向量空間$V_S$,若$\{\vec{b_1},\vec{b_2},\dots,\vec{b_n}\}\subset V_S$為互不平行,即綫性獨立,同時對任意$\vec{v}\in V_S$,都存在$\alpha_1,\alpha_2,\dots,\alpha_n\in S$使得$$\vec{v}=\sum_{k=1}^{n}\alpha_k \vec{b_k}$$則稱$\{\vec{b_1},\vec{b_2},\dots,\vec{b_n}\}$為$V_S$的\textbf{基}。

        若$\{\vec{\beta_1},\vec{\beta_2},\dots,\vec{\beta_n}\}\subset V_S$為$V_S$的基而且對所有$i\neq j$,均有$$\innerprod{\vec{\beta_i}}{\vec{\beta_j}}=0$$則稱$\{\vec{\beta_1},\vec{\beta_2},\dots,\vec{\beta_n}\}$為$V_S$的\textbf{正交基}。

        若$\{\vec{\gamma_1},\vec{\gamma_2},\dots,\vec{\gamma_n}\}\subset V_S$為$V_S$的正交基而且對所有$i$,均有$$|\vec{\gamma_i}|=1$$則稱$\{\vec{\gamma_1},\vec{\gamma_2},\dots,\vec{\gamma_n}\}$為$V_S$的\textbf{標準正交基}。
    \end{definition}
    
    \begin{remark}
        對於任何向量空間,標準正交基的構成并非唯一。舉例$\mathbb{R}$作爲$\mathbb{R}$的向量空間,$1$和$-1$均可作爲$\mathbb{R}$的標準正交基;$\mathbb{R}^2$作爲$\mathbb{R}$的向量空間,則$$\{\begin{bmatrix}
            \cos{\theta}\\\sin{\theta}
        \end{bmatrix},\begin{bmatrix}
            -\sin{\theta}\\\cos{\theta}
        \end{bmatrix}:\theta\in\mathbb{R}\}$$均爲$\mathbb{R}^2$的標準正交基。
    \end{remark}

    對於有標準正交基的向量空間,我們的討論會比較簡單:

    \begin{theorem}
        設$V_S$為向量空間,$\{e_1,e_2,\dots,e_n\}$為$V_S$的標準正交基。若$v,u\in V_S$可寫作$v=\sum_{i=1}^{n}v_i e_i$及$u=\sum_{i=1}^{n}u_i e_i$,則$$\innerprod{u}{v}=\sum_{i=1}^{n}u_iv_i$$
    \end{theorem}

    \begin{proof}
        \begin{align*}
            \innerprod{u}{v}
            &=\innerprod{\sum_{i=1}^{n}u_i e_i}{\sum_{j=1}^{n}v_j e_j}
            =\sum_{i=1}^{n}u_i\innerprod{e_i}{\sum_{j=1}^{n}v_j e_j}
            =\sum_{i=1}^{n}\sum_{j=1}^{n}u_iv_j\innerprod{e_i}{e_j}\\
            &=\sum_{i=1}^{n}\sum_{j=1}^{n}u_iv_j\delta_{ij}
            =\sum_{\substack{
                i=j\\1\leq i,j\leq n
            }}u_iv_j
            =\sum_{i=1}^{n}u_iv_i
        \end{align*}
    \end{proof}

    \begin{corollary}
        設$V_S$為向量空間,$\{e_1,e_2,\dots,e_n\}$為$V_S$的標準正交基。若$v\in V_S$可寫作$v=\sum_{i=1}^{n}v_i e_i$,則$$|v|=\sqrt{\sum_{i=1}^{n}v_i^2}$$而且$$v_i=\innerprod{v}{e_i}$$
    \end{corollary}

    爲方便描述,接下來會稱$V_S$的標準正交基為$\mathcal{O}(V_S):= \{e_i\}_{i\in I}$,$I$為索引集。

    \begin{theorem}[餘弦定理]
        設$u,v\in V_S$,則$$|u-v|^2=|u|^2+|v|^2-2|u||v|\cos{\theta}$$其中$\theta$是$u,v$的夾角。
    \end{theorem}

    \begin{proof}
        設$u=\sum_{i\in I}u_ie_i, v=\sum_{i\in I}v_ie_i$,則\begin{align*}
            |u-v|^2&=\sum_{i\in I}(u_i-v_i)^2\\
            &=\sum_{i\in I}(u_i^2+v_i^2-2u_iv_i)\\
            &=\sum_{i\in I}u_i^2+\sum_{i\in I}v_i^2-2\sum_{i\in I}u_iv_i\\
            &=|u|^2+|v|^2-2\innerprod{u}{v}\\
            &=|u|^2+|v|^2-2|u||v|\cos{\theta}
        \end{align*}
    \end{proof}

    在三維空間中,存在外積:

    \begin{definition}[外積]
        假設$\{\hat{i},\hat{j},\hat{k}\}\subset \mathbb{R}^3$為$\mathbb{R}^3$的標準正交基,則定義$\times:\mathbb{R}^3\to\mathbb{R}^3$為$$\hat{i}\times \hat{j}=\hat{k},\hat{j}\times \hat{k}=\hat{i}, \hat{k}\times \hat{i}=\hat{j}$$
    \end{definition}

    外積的定義可考慮面積與體積的計算原理:考慮三個單位向量$\hat{i},\hat{j},\hat{k}$的乘積為體積及任意兩個向量的乘積為面積,基於$\hat{i}\times \hat{j}$為$ij$平面的面積單位,而面積乘以高等於體積,故$V(\hat{i},\hat{j},\hat{k})=(\hat{i}\times \hat{j})\cdot \hat{k}$為標量,使得$\hat{i}\times\hat{j}=\hat{k}$。

    因此,我們可定義

    \begin{definition}[面積與體積]
        設$u,v,w\in\mathbb{R}^3$,則\begin{align*}
            A(u,v)&:=|u\times v|=\left|\left|\begin{matrix}
                \hat{i}&\hat{j}&\hat{k}\\
                u_1&u_2&u_3\\
                v_1&v_2&v_3
            \end{matrix}\right|\right|\\
            V(u,v,w)&:=(u\times v)\cdot w=\left|\begin{matrix}
                u_1&u_2&u_3\\
                v_1&v_2&v_3\\
                w_1&w_2&w_3
            \end{matrix}\right|\\
        \end{align*}
    \end{definition}

    事實上,外積的計算無法以向量簡單作結。Kronecker就發現基本的向量無法解釋$i\times j=k$的情況(例如,爲何面積是向量而體積不是?),因此,他提出以\textbf{雙向量}(bi-vector)為$\hat{i},\hat{j},\hat{k}$下定義:

    \begin{definition}
        定義$\hat{i}=e_1e_2,\hat{j}=e_2e_3,\hat{k}=e_1e_3$,且$e_ie_j=-e_je_i$,$e_i^2=1$由此符合\begin{align*}
            \hat{i}\hat{j}&=e_1e_2e_2e_3=e_1e_3=\hat{k}\\
            \hat{j}\hat{k}&=e_2e_3e_1e_3=-e_2e_3e_3e_1=-e_2e_1=e_1e_2=\hat{i}\\
            \hat{k}\hat{i}&=e_1e_3e_1e_2=-e_3e_1e_1e_2=-e_3e_2=e_2e_3=\hat{j}
        \end{align*}
    \end{definition}

    上述定義可引申至對軸心的旋轉:$\hat{i}$為沿$z$軸逆時針旋轉90度;$\hat{j}$為沿$x$軸逆時針旋轉90度;$\hat{k}$為沿$y$軸逆時針旋轉90度。

    事實上,在更高維的空間裏,向量的外積有以下定義

    \begin{definition}
        設$u=(u_i)_{i\in I},v=(v_i)_{i\in I}$,則外積為$$u\wedge v =\begin{bmatrix}
            u_1v_1&u_1v_2&\cdots&u_1v_n\\
            u_2v_1&u_2v_2&\cdots&u_2v_n\\
            \vdots&\vdots&\ddots&\vdots\\
            u_n v_1&u_n v_2&\cdots&u_n v_n
        \end{bmatrix}$$
    \end{definition}

    \begin{theorem}
        設$u,v\in \mathbb{R}^n$,則平行四邊形面積爲$$A(u,v)=|u||v|\sin{\theta}$$其中$\theta$為$u$和$v$的夾角。因此$$|u\wedge v|=|u||v|\sin{\theta}$$
    \end{theorem}

    \begin{corollary}
        設$u,v\in \mathbb{R}^n$,則$$|u|^2|v|^2=\innerprod{u}{v}^2+|u\wedge v|^2$$
    \end{corollary}

    外積的應用并不廣汎,因爲在高維空間其實很難定義内外,因此我們通常更集中於内積的運算。其中對於互補空間的探索可謂是將内積運用至極緻。
    
    \begin{definition}[維度]
        若某向量空間$V$的標準正交基有$n$項綫性獨立元素,則稱$V$是一個$n$維空間。記$\dim{V}=n$。
    \end{definition}

    \begin{definition}[補空間]
        設$V \subset \mathbb{R}^n$同時$\dim{V}=k< n$。若存在$V^{\perp}\subset \mathbb{R}^n$使得\begin{enumerate}
            \item $\dim{V^{\perp}}=n-k$及;
            \item $\forall v\in V, \forall v^{\perp}\in V^{\perp}$使得$$\innerprod{v}{v^{\perp}}=0$$
        \end{enumerate}
        則稱$V^{\perp}$為$V$的補空間。
    \end{definition}

    因此,$V^{\perp}$通常被定義爲$$V^{\perp}:=\{w\in\mathbb{R}^n:\innerprod{w}{v}=0, \forall v\in V\}$$

    \begin{example}
        在$\mathbb{R}^2$上,下列情況均為互補空間:\begin{enumerate}
            \item $\mathbf{X}:=\{(x,0)\in\mathbb{R}^2\}$及$\mathbf{Y}:=\{(0,y)\in\mathbb{R}^2\}$;
            \item $\forall u,v \in\mathbb{R}^2$,若$\innerprod{u}{v}=0$,則$\mathbf{U}:=\{ku:k\in\mathbb{R}\}$及$\mathbf{V}:=\{hv:h\in\mathbb{R}\}$為互補空間。
        \end{enumerate}
    \end{example}

    考慮$u,v\in\mathbb{R}^n$使得$u\wedge v\neq 0$,令$w:=u-\frac{\innerprod{u}{v}}{|v|^2}v$則$$\innerprod{u}{w}=0$$由此,我們可得以下命題:

    \begin{proposition}[正交化]
        設$u,v\in \mathbb{R}^n$且$u\in U\subset \mathbb{R}^n$,則$u-\frac{\innerprod{u}{v}}{|v|^2}v\in U^{\perp}$。
    \end{proposition}

    \begin{proof}
        留做習題。
    \end{proof}

    \begin{theorem}[格林-史密特正交化]
        若$B_V=\{b_1,b_2,\dots,b_n\}$為向量空間$V$的基,并且$b_i\wedge b_j\neq 0$。令$\{\beta_1,\beta_2,\dots,\beta_n\}$為另一基集使得$$\beta_n:=b_n-\sum_{k=1}^{n-1}\frac{\innerprod{\beta_k}{b_n}}{|\beta_k|^2}\beta_k$$則$\{\beta_1,\beta_2,\dots,\beta_n\}$為$V$的正交基。
    \end{theorem}

    \begin{proof}
        利用命題,可證明$\{\beta_1,\beta_2,\dots,\beta_n\}$為$V$的正交集;又因$\{\beta_1,\beta_2,\dots,\beta_n\}$的元素為$B_V$的綫性組合,$B_V$亦可寫成$\{\beta_1,\beta_2,\dots,\beta_n\}$的綫性組合,因此$\{\beta_1,\beta_2,\dots,\beta_n\}$為$V$的基。
    \end{proof}

    \begin{example}
        求以$\{(1,2,3),(1,3,5),(1,0,1)\}$為基的向量空間的一組標準正交基。

        \begin{align*}
            \beta_1'&=(1,0,1)\\
            \beta_1&=\frac{\beta_1'}{|\beta_1'|}=\frac{1}{\sqrt{2}}(1,0,1)\\
            \beta_2'&=(1,2,3)-\frac{4}{2}(1,0,1)\\
            &=(-1,2,1)\\
            \beta_2&=\frac{\beta_2'}{|\beta_2'|}=\frac{1}{\sqrt{6}}(-1,2,1)\\
            \beta_3'&=(1,3,5)-\frac{6}{2}(1,0,1)-\frac{10}{6}(-1,2,1)\\
            &=(-\frac{1}{3},-\frac{1}{3},\frac{1}{3})\\
            \beta_3&=\frac{\beta_3'}{|\beta_3'|}=\frac{1}{\sqrt{3}}(-1,-1,1)
        \end{align*}
        因此,$V$的其中一組標準正交基為$\{\frac{1}{\sqrt{2}}(1,0,1),\frac{1}{\sqrt{6}}(-1,2,1),\frac{1}{\sqrt{3}}(-1,-1,1)\}$
    \end{example}

    \section*{向量函數}

    向量函數顧名思義在於將函數從一個向量空間映射向另一個向量空間,其形式可以看成

    \begin{definition}[向量函數]
        給定$f:\mathbb{R}^m\to\mathbb{R}^n$,設$x=(x_1,x_2,\dots,x_m)\in\mathbb{R}^m$及$y=(y_1,y_2,\dots,y_n)\in\mathbb{R}^n$,若$$y=f(x)$$則對於所有$1\leq k\leq n$,存在獨特的$f_k:\mathbb{R}^m\to\mathbb{R}$使得$$y_k=f_k(x)$$
    \end{definition}

    \begin{example}
        以下爲向量函數的例子:\begin{enumerate}
            \item $f(x):=x^2+3x+1$;
            \item $f(x,y,z):=xy+xz+3y-2z$;
            \item $f(x):=(x,x^2,x^2+x)$;
            \item $f(x,y):=(x^2y,xy^2,x^3-y^3)$.
        \end{enumerate}
    \end{example}

    由於向量函數相較實函數有更多偏差發生,例如$$f(x,y):=(\frac{1}{x-y},x^2+y^2)$$并不能在$x=y$的情況下連續,因此我們又需要調出那個萬用的極限。

    \begin{definition}[向量函數的極限]
        設$y=f(x)$,其中$y\in\mathbb{R}^n$。若存在$y_0\in\mathbb{R}^n$使得對於任意$\epsilon>0$,都有$\delta>0$使得$|x-x_0|<\delta$時$|f(x)-y_0|<\epsilon$,則稱$y_0$為$f(x)$在$x_0$的極限,記$\displaystyle y_0=\lim_{x\to x_0}f(x)$。
    \end{definition}

    \begin{definition}[向量函數的連續性]
        若$f(x_0)=\displaystyle \lim_{x\to x_0}f(x)$,則稱$f$在$x_0$上連續。
    \end{definition}

    \section*{偏導數與全導數}

    此後的函數均預設為連續函數。

    \begin{definition}[偏導數]
        設$f:\mathbb{R}^m\to\mathbb{R}^n$,而$(x_1,x_2,\dots,x_m)\overset{f}{\mapsto}(y_1,y_2,\dots,y_n)$。若只求$f$對於單一變量$x_k$的變化,則$$\partialderivative{f}{x_k}=\lim_{h\to 0}\frac{f(x_1,x_2,\dots,x_k+h,\dots,x_n)-f(x_1,x_2,\dots,x_k,\dots,x_n)}{h}$$

        考慮$f=(f^1,f^2,\dots,f^n)$的記法,可將之寫成$$(\partialderivative{f^1}{x_k},\partialderivative{f^2}{x_k},\dots,\partialderivative{f^n}{x_k})$$
    \end{definition}

    \begin{remark}
        偏導數的記法通常以$\displaystyle\partialderivative{f}{x}$表示,也有寫$\partial_x f$或$f_x$簡便記之。
    \end{remark}

    \begin{example}
        求$f(x,y,z):=xyz$的所有偏導數。

        \begin{align*}
            \partial_x f&=\partial_x(x)yz+x\partial_x(y)z+xy\partial_x(z)=yz\\
            \partial_y f&=\partial_y(x)yz+x\partial_y(y)z+xy\partial_y(z)=xz\\
            \partial_z f&=\partial_z(x)yz+x\partial_z(y)z+xy\partial_z(z)=xy
        \end{align*}
    \end{example}

    \begin{example}
        求$f(x,y):=(x^2,xy,xy^2)$的所有偏導數。

        \begin{align*}
            \partial_x f&=(\partial_x(x^2),\partial_x(xy),\partial_x(xy^2))=(2x,y,y^2)\\
            \partial_y f&=(\partial_y(x^2),\partial_y(xy),\partial_y(xy^2))=(0,x,2xy)
        \end{align*}
    \end{example}

    \begin{definition}[導數法]
        設$f:\mathbb{R}^m\to\mathbb{R}^n$,而$(x_1,x_2,\dots,x_m)\overset{f}{\mapsto}(y_1,y_2,\dots,y_n)$。則\begin{align*}
            \derivative{f}{x}:=\lim_{\vec{h}\to\vec{0}}\frac{f(x_1+h_1,x_2+h_2,\dots,x_m+h_m)-f(x_1,x_2,\dots,x_m)}{\norm*{\vec{h}}}
        \end{align*}
    \end{definition}

    \begin{theorem}
        設$f:\mathbb{R}^m\to\mathbb{R}^n$,而$(x_1,x_2,\dots,x_m)\overset{f}{\mapsto}(y_1,y_2,\dots,y_n)$。則$f$的全導數為$$df:=\sum_{k=1}^{n}\partial_{x_k}f dx_k=\begin{bmatrix}
            f_{x_1}&f_{x_2}&\cdots&f_{x_m}
        \end{bmatrix}d(x_1,x_2,\dots,x_m)$$
    \end{theorem}

    \begin{proof}
        考慮基本原理,若$f$為連續函數,則\begin{align*}
            \derivative{f}{x}:&=\lim_{\vec{h}\to\vec{0}}\frac{f(x_1+h_1,x_2+h_2,\dots,x_m+h_m)-f(x_1,x_2,\dots,x_m)}{\norm*{\vec{h}}}\\
            &=\sum_{i=1}^{m}\lim_{h_i\to 0}\frac{f(x_1,x_2,\dots,x_i+h_i,\dots,x_m)-f(x_1,x_2,\dots,x_i,\dots,x_m)}{h_i}\\
            &=\sum_{i=1}^{m}f_{x_i}
        \end{align*}
        并且,因$f_{x_i} dx_j=df\frac{dx_j}{dx_i}=0$,\begin{align*}
            (\sum_{i=1}^{m}f_{x_i})dx&=\sum_{i=1}^{m}(f_{x_i} dx_i)\\
            &=\begin{bmatrix}
                f_{x_1}&f_{x_2}&\cdots&f_{x_m}
            \end{bmatrix}\begin{bmatrix}
                dx_1\\dx_2\\\vdots\\dx_m
            \end{bmatrix}\\
            &=\begin{bmatrix}
                f_{x_1}&f_{x_2}&\cdots&f_{x_m}
            \end{bmatrix}d(x_1,x_2,\dots,x_m)
        \end{align*}
    \end{proof}

    \subsection*{情況一:$\mathbb{R}^3\to\mathbb{R}$}

    \begin{example}
        設$\vec{v}=(x,y,z)$並考慮$f(x,y,z)=xyz$,\begin{align*}
            df&=\sum_{k=1}^{3}\partial_{x_k}f dx_k\\
            &=\partial_x f dx + \partial_y f dy + \partial_z f dz\\
            &=yz dx+ xz dy + xy dz\\
            &=\begin{bmatrix}
                yz&xz&xy
            \end{bmatrix}\begin{bmatrix}
                dx\\dy\\dz
            \end{bmatrix}\\
            &=\begin{bmatrix}
                f_x&f_y&f_z
            \end{bmatrix}d\vec{v}
        \end{align*}
    \end{example}

    \subsection*{情況二:$\mathbb{R}\to\mathbb{R}^3$}

    \begin{example}
        考慮$\vec{f}(t)=(t,t^2,t^3)$,\begin{align*}
            d\vec{f}&=\sum_{k=1}^{3}\partial_{x_k}f dx_k\\
            &=\partial_x f dt\\
            &=(1,2t,3t^2)dt\\
            &=\vec{f}_t dt
        \end{align*}
    \end{example}

    \subsection*{情況三:$\mathbb{R}^3\to\mathbb{R}^3$}

    \begin{example}
        設$\vec{v}=(x,y,z)$並考慮$f(x,y,z)=(x+y^2+z^3, x^3+y+z^2, x^2+y^3+z)$,\begin{align*}
            df&=\sum_{k=1}^{3}\partial_{x_k}f dx_k\\
            &=\partial_x f dx + \partial_y f dy + \partial_z f dz\\
            &=(1,3x^2,2x) dx+ (2y,1,3y^2) dy + (3z^2,2z,1) dz\\
            &=\begin{bmatrix}
                1&2y&3z^2\\3x^2&1&2z\\2x&3y^2&1
            \end{bmatrix}\begin{bmatrix}
                dx\\dy\\dz
            \end{bmatrix}\\
            &=\begin{bmatrix}
                \vec{f}_x&\vec{f}_y&\vec{f}_z
            \end{bmatrix}d\vec{v}
        \end{align*}
    \end{example}

    \begin{definition}[全導數]
        設$(x_1,x_2,\dots,x_m)\overset{f}{\mapsto}(y_1,y_2,\dots,y_n)$,記$$\nabla f=\begin{bmatrix}
            \partialderivative{y_1}{x_1}&\partialderivative{y_1}{x_2}&\cdots&\partialderivative{y_1}{x_m}\\
            \partialderivative{y_2}{x_1}&\partialderivative{y_2}{x_2}&\cdots&\partialderivative{y_2}{x_m}\\
            \vdots&\vdots&\ddots&\vdots\\
            \partialderivative{y_n}{x_1}&\partialderivative{y_n}{x_2}&\cdots&\partialderivative{y_n}{x_m}
        \end{bmatrix}$$
        則$df=\nabla f dx$。我們稱$\nabla f$為$f$的\textbf{第一階全導數}。
    \end{definition}

    \begin{example}
        估算$(3.001)^{5.001}$的值。

        \begin{sol}
            設$f(x,y)=x^y$,則\begin{align*}
                df&=\nabla f dx\\
                &=\begin{bmatrix}
                    yx^{y-1}&x^y\ln{x}
                \end{bmatrix}\begin{bmatrix}
                    dx\\dy
                \end{bmatrix}\\
                &=yx^{y-1}dx+x^y\ln{x}dy
            \end{align*}

            因此對$(3.001)^{5.001}$作估算:\begin{align*}
                (3.001)^{5.001}&\approx 3^5+5\cdot 3^4 (0.001)+3^5\ln{3} (0.001)\\
                &=243.405+0.243\ln{3}\\
                &=243.672
            \end{align*}
        \end{sol}
    \end{example}

    更進一步,我們可以討論二階導數:

    \begin{theorem}[二階導數]
        設$(x_1,x_2,\dots,x_m)\overset{f}{\mapsto}(y_1,y_2,\dots,y_n)$,記$$\nabla^2 f=\begin{bmatrix}
            \frac{\partial^2 y_1}{\partial^2 x_1}&\frac{\partial^2 y_1}{\partial x_1 \partial x_2}&\cdots&\frac{\partial^2 y_1}{\partial x_1 \partial x_m}\\
            \frac{\partial^2 y_2}{\partial x_1 \partial x_2}&\frac{\partial^2 y_2}{\partial^2 x_2}&\cdots&\frac{\partial^2 y_2}{\partial x_2 \partial x_m}\\
            \vdots&\vdots&\ddots&\vdots\\
            \frac{\partial^2 y_n}{\partial x_1 \partial x_m}&\frac{\partial^2 y_n}{\partial x_2 \partial x_m}&\cdots&\frac{\partial^2 y_n}{\partial^2 x_m}
        \end{bmatrix}$$
        則$d^2f=\nabla^2 f d^2x$。我們稱$\nabla f$為$f$的\textbf{第二階全導數}。
    \end{theorem}

    \section*{方向導數}

    \section*{切面與法綫}

    \section*{二維極值與鞍點}
\end{document}