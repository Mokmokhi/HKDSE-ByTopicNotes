\documentclass[12pt]{article}
\usepackage{ctex}
\usepackage[english]{babel}
\usepackage{blindtext}
\usepackage{nameref}
\usepackage{fancyhdr}
\usepackage{amsmath,amssymb,amsthm}
\usepackage{graphicx,float}
\usepackage{physics}
\usepackage{pgfplots}
\usepackage[a4paper, total={6in, 9in}]{geometry}

\graphicspath{{../image/}}

\pagestyle{fancy}
\fancyhf{}
\fancyhf[HL]{微分}
\fancyhf[CF]{\thepage}

\newcommand{\innerprod}[2]{\langle{#1},{#2}\rangle}
\newcommand{\id}{\mathtt{id}}

\newtheorem{definition}{定義}
\newtheorem*{theorem}{定理}
\newtheorem*{corollary}{衍理}
\newtheorem*{lemma}{引理}
\newtheorem*{proposition}{設理}
\newtheorem*{remark}{小記}
\newtheorem*{claim}{主張}
\newtheorem*{example}{例子}
\newtheorem*{axiom}{公設}
\renewenvironment*{proof}{\textit{證明.}}{\hfill$\qed$}

\newenvironment*{sol}{\par \textbf{解}.}{\hfill$\blacksquare$}

\begin{document}
    References: Introduction to Real Analysis (Bartle \& Sherbert), Thomas Calculus 12th Edition
    \section*{向量}

    向量屬於一種特殊的矩陣,通常用以表達\textbf{多維坐標}。

    \begin{definition}[向量]
        一個\textbf{$n$-維向量}包含$n$個元素,可視之爲\textbf{$n$-維空間}中的坐標,同時代表從原點指向該坐標的箭頭。
    \end{definition}

    \begin{definition}[二元算子]
        設$S$為集合,且$x,y\in S$。定義$\circ$為$S$上的算符使得$$x\circ y := \circ(x,y)$$則稱$\circ$為$S$上的算子。
    \end{definition}

    \begin{example}[實數加法]
        在實域$\mathbb{R}$中,若$x,y\in \mathbb{R}$,則$x+y=+(x,y)$
    \end{example}

    \begin{example}[向量加法]
        在向量集合$V$中,若$\vec{x},\vec{y}\in V$,則$\vec{x}+\vec{y}=+(\vec{x},\vec{y})$
    \end{example}
    \section*{向量空間}

    \section*{向量函數}

    \section*{偏導數與全導數}

    \section*{方向導數}

    \section*{切面與法綫}

    \section*{二維極值與鞍點}
\end{document}