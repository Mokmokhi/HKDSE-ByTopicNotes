\documentclass[12pt]{article}
\usepackage{ctex}
\usepackage[english]{babel}
\usepackage{blindtext}
\usepackage{nameref}
\usepackage{fancyhdr}
\usepackage{amsmath,amssymb,amsthm}
\usepackage{graphicx,float}
\usepackage{physics}
\usepackage{pgfplots}
\usepackage[a4paper, total={6in, 9in}]{geometry}

\pagestyle{fancy}
\fancyhf{}
\fancyhf[HL]{Differentiation and its applications}
\fancyhf[CF]{\thepage}

\newcommand{\innerprod}[2]{\langle{#1},{#2}\rangle}
\newcommand{\id}{\mathtt{id}}

\newtheorem*{definition}{Definition}
\newtheorem*{theorem}{Theorem}
\newtheorem*{corollary}{Corollary}
\newtheorem*{lemma}{Lemma}
\newtheorem*{proposition}{Proposition}
\newtheorem*{remark}{Remark}
\newtheorem*{claim}{Claim}
\newtheorem*{example}{Example}
\newtheorem*{axiom}{Axiom}

\begin{document}
\section*{Intuition - What is differentiation}
    One would say that differentiation is quite difficult to learn, but we as a Math learner should see it as a simple tool that helps Understand many complicated concepts. We shall start with the intuition of differentiation.

    Imagine we are having a slide. Place a ball on the top of the slide and let it roll down. We will always want to ask: What is the instantaneous speed of the ball at every moment? This is the origin of differentiation, and in fact we shall refer this question to some general statement: What is the \textit{rate of change} of the distance travel at every moment?

    If we further convert the question to a mathematical statement, we will see that the formulation of the question be like: $$v(t)=\frac{\Delta s(t)}{\Delta t}=\frac{s(t)-s(t_0)}{t-t_0}$$ which comes from Newtonian physics with $v$ denoting the average speed of the ball at some time interval, and $s$ denoting the displacement of the ball at some moment.
\end{document}