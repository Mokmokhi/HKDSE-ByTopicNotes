\documentclass[12pt]{article}
\usepackage{ctex}
\usepackage[english]{babel}
\usepackage{blindtext}
\usepackage{nameref}
\usepackage{fancyhdr}
\usepackage{amsmath,amssymb,amsthm}
\usepackage{graphicx,float}
\usepackage{physics}
\usepackage{pgfplots}
\usepackage[a4paper, total={6in, 9in}]{geometry}

\graphicspath{{../image/}}

\pagestyle{fancy}
\fancyhf{}
\fancyhf[HL]{測驗3}
\fancyhf[CF]{\thepage}

\newcommand{\innerprod}[2]{\langle{#1},{#2}\rangle}
\newcommand{\id}{\mathtt{id}}

\newtheorem{definition}{定義}
\newtheorem*{theorem}{定理}
\newtheorem*{corollary}{衍理}
\newtheorem*{lemma}{引理}
\newtheorem*{proposition}{設理}
\newtheorem*{remark}{小記}
\newtheorem*{claim}{主張}
\newtheorem*{example}{例子}
\newtheorem*{axiom}{公設}
\renewenvironment*{proof}{\textit{證明.}}{\hfill$\qed$}

\newenvironment*{sol}{\par \textbf{解}.}{\hfill$\blacksquare$}

\begin{document}
    \begin{enumerate}
        \item 利用$\epsilon-\delta$定義,證明微分的乘積法則:對於可微實函數$f,g$,$$(fg)'=f'g+fg'$$即$$\lim_{x\to x_0}\frac{f(x)g(x)-f(x_0)g(x_0)}{x-x_0}=f'(x_0)g(x_0)+f(x_0)g'(x_0)$$
        \item 設$f(x)=\sin{\pi x}$。設$(g_n)$為函數數列使得$g_1(x)=f(x)$及$g_n(x)=f(g_{n-1}(x))$。證明對於任意正整數$n$,$g_n'(0)=\pi^n$。
        \item 設$E(x)$為可微實函數,且$\forall x\in \mathbb{R}, E'(x)>0$。此稱爲單調遞升函數。\begin{enumerate}
            \item 證明$E(x)$最多只有一個實根。
            \item 定義單調遞降可微實函數$D$為$\forall x\in \mathbb{R}, D'(x)<0$。證明單調遞升實函數與單調遞降實函數只能有最多一個相交點。
        \end{enumerate}
        \item 設$f(x)=e^{-x^2}$。求$f(x)$及$-f(x)$所包裹的區域内最大圓形與最大正方形的面積之比。
        \item 運用數學歸納法,證明$P_n(x):=\dfrac{1}{2^n n!}\dfrac{d^n}{dx^n}((x^2-1)^n)$的所有根均爲實根,而且均存在於$-1$與$1$之間。
        \item 已知對於任意$\delta>0$, $\forall x\in (-\delta,\delta)$,$-kx+c\leq f(x)\leq kx+c$。證明$$\lim_{x\to 0}x^{f(x)-c}=1$$
        \item 求下列函數的泰勒展開式至$x^5$:\begin{enumerate}
            \item $\sin{x}$
            \item $\cos{x}$
            \item $e^x$
        \end{enumerate}
        \item 利用牛頓分割法,解$xe^x=1$准確至五位小數。
    \end{enumerate}
\end{document}