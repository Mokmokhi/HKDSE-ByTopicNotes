\documentclass[12pt]{article}
\usepackage{ctex}
\usepackage[english]{babel}
\usepackage{blindtext}
\usepackage{nameref}
\usepackage{fancyhdr}
\usepackage{amsmath,amssymb,amsthm}
\usepackage{graphicx,float}
\usepackage{physics}
\usepackage{pgfplots}
\usepackage[a4paper, total={6in, 9in}]{geometry}

\graphicspath{{../image/}}

\pagestyle{fancy}
\fancyhf{}
\fancyhf[HL]{矩陣}
\fancyhf[CF]{\thepage}

\newcommand{\innerprod}[2]{\langle{#1},{#2}\rangle}
\newcommand{\id}{\mathtt{id}}

\newtheorem{definition}{定義}
\newtheorem*{theorem}{定理}
\newtheorem*{corollary}{衍理}
\newtheorem*{lemma}{引理}
\newtheorem*{proposition}{命題}
\newtheorem*{remark}{小記}
\newtheorem*{claim}{主張}
\newtheorem*{example}{示例}
\newtheorem*{axiom}{公設}
\renewenvironment*{proof}{\textit{證明.}}{\hfill$\qed$}

\newenvironment*{sol}{\par \textbf{解}.}{\hfill$\blacksquare$}

\begin{document}
    \begin{center}
        參考答案/提示
    \end{center}
    \begin{enumerate}
        \item 從定義,考慮$f(x)g(x)-f(x_0)g(x_0)\equiv f(x)g(x)-f(x)g(x_0)+f(x)g(x_0)-f(x_0)g(x_0)$并運用三角不等式及考慮$|g(x)|<|g(x_0)|+\epsilon_g$。
        \item 運用數學歸納法。
        \item (a)設$E(x)$有兩個相異實根$\alpha<\beta$,則$E(\alpha)=E(\beta)$。但此違反$E$作爲單調遞升函數的定義(均值定理)。因此只能擁有不多於一個實根。(b)設$F,G$分別爲單調遞升及單調遞降函數,設$H:=F-G\implies H'=F'-G'>0$。
        \item $\max r = \min \sqrt{x^2+[f(x)]^2}$。
        \item 
    \end{enumerate}
\end{document}