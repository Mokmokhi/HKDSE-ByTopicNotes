\documentclass[12pt]{article}
\usepackage{ctex}
\usepackage[english]{babel}
\usepackage{blindtext}
\usepackage{nameref}
\usepackage{fancyhdr}
\usepackage{amsmath,amssymb,amsthm}
\usepackage{graphicx,float}
\usepackage{physics}
\usepackage{pgfplots}
\usepackage[a4paper, total={6in, 9in}]{geometry}

\pagestyle{fancy}
\fancyhf{}
\fancyhf[HL]{F4 Level 2 M2 mock paper}
\fancyhf[HR]{Time limit: 1 hr 30 mins}
\fancyhf[CF]{\thepage}

\newcommand{\innerprod}[2]{\langle{#1},{#2}\rangle}
\newcommand{\id}{\mathtt{id}}

\newtheorem*{definition}{Definition}
\newtheorem*{theorem}{Theorem}
\newtheorem*{corollary}{Corollary}
\newtheorem*{lemma}{Lemma}
\newtheorem*{proposition}{Proposition}
\newtheorem*{remark}{Remark}
\newtheorem*{claim}{Claim}
\newtheorem*{example}{Example}
\newtheorem*{axiom}{Axiom}

\begin{document}
    \thispagestyle{plain}

    \centering 

    \section*{PRACTICE PAPER\\MATHEMTICS Extended Part\\Module 2 (Algebra and Calculus)\\Question-Answer Book}

    Time allowed: 1.5 hours

    Name:\hrulefill \hfill Marks:\hrulefill/100

    School:\hrulefill

    \raggedright

    \subsection*{Instructions}

    \begin{enumerate}
        \item This paper must be answered in English.
        \item Unless otherwise specified, all working must be clearly shown.
        \item Unless otherwise specified, numerical answers must be exact.
        \item This paper is for \textbf{internal use} only.
        \item All questions are collected from AL/CE/DSE past papers, reference site: https://www.dse.life/ppindex/m2/
    \end{enumerate}

    \newpage
    \begin{enumerate}
        \item Prove, by mathematical induction, that for any positive integer $n$, $$1+2+3+\cdots+n=\frac{n(n+1)}{2}.$$\hfill(10 marks)
        
            \hrulefill
            
            \hrulefill
            
            \hrulefill
            
            \hrulefill
            
            \hrulefill
            
            \hrulefill
            
            \hrulefill
            
            \hrulefill
            
            \hrulefill
            
            \hrulefill
            
            \hrulefill
            
            \hrulefill
            
            \hrulefill
            
            \hrulefill
            
            \hrulefill
            
            \hrulefill
            
            \hrulefill
            
            \hrulefill
            
            \hrulefill
            
            \hrulefill
            
            \hrulefill
            
            \hrulefill
            
            \hrulefill
            
            \hrulefill
            
            \hrulefill

        \pagebreak
        \item (2012-DSE-MATH-EP(M2) \#03)Prove, by mathematical induction, that $$1\times2+2\times5+\cdots+n(3n-1)=n^2(n+1)$$ for all positive integers $n$.\hfill(10 marks)
        
        \hrulefill
            
        \hrulefill
        
        \hrulefill
        
        \hrulefill
        
        \hrulefill
        
        \hrulefill
        
        \hrulefill
        
        \hrulefill
        
        \hrulefill
        
        \hrulefill
        
        \hrulefill
        
        \hrulefill
        
        \hrulefill
        
        \hrulefill
        
        \hrulefill
        
        \hrulefill
        
        \hrulefill
        
        \hrulefill
        
        \hrulefill
        
        \hrulefill
        
        \hrulefill
        
        \hrulefill
        
        \hrulefill
        
        \hrulefill
        
        \hrulefill

    \pagebreak
        \item (2004-CE-A MATH \#02)\begin{enumerate}
            \item Expand $(1+2x)^n$ in ascending powers of $x$ up to the term $x^3$, where $n$ is a positive integer.
            \item In the expansion of $(x-\frac{3}{x})^2(1+2x)^n$, the constant term is 210. Find the value of $n$.
        \end{enumerate}\hfill(12 marks)

            \hrulefill
            
            \hrulefill
            
            \hrulefill
            
            \hrulefill
            
            \hrulefill
            
            \hrulefill
            
            \hrulefill
            
            \hrulefill
            
            \hrulefill
            
            \hrulefill
            
            \hrulefill
            
            \hrulefill
            
            \hrulefill
            
            \hrulefill
            
            \hrulefill
            
            \hrulefill
            
            \hrulefill
            
            \hrulefill
            
            \hrulefill
            
            \hrulefill
            
            \hrulefill
            
            \hrulefill

        \pagebreak
        \item (1990-CE-A MATH 2 \#06(a)) If $\cos{\theta}+\sqrt{3}\sin{\theta}=r\cos(\theta-\alpha)$, where $r>0$ and $0^\circ\leq \alpha\leq 90^\circ$. Find $r$ and $\alpha$.\hfill(8 marks)
        
        \hrulefill
            
            \hrulefill
            
            \hrulefill
            
            \hrulefill
            
            \hrulefill
            
            \hrulefill
            
            \hrulefill
            
            \hrulefill
            
            \hrulefill
            
            \hrulefill
            
            \hrulefill
            
            \hrulefill
            
            \hrulefill
            
            \hrulefill
            
            \hrulefill
            
            \hrulefill
            
            \hrulefill
            
            \hrulefill
            
            \hrulefill
            
            \hrulefill
            
            \hrulefill
            
            \hrulefill
            
            \hrulefill
            
            \hrulefill
            
            \hrulefill
            
            \hrulefill
            
            \hrulefill

        \pagebreak
        \item (2003-CE-A MATH \#10) Given two acute angles $\alpha$ and $\beta$. Show that $$\frac{\sin{\alpha}+\sin{\beta}}{\cos{\alpha}+\cos{\beta}}=\tan{\frac{\alpha+\beta}{2}}$$ If $3\sin{\alpha}-4\cos{\alpha}=4\cos{\beta}-3\sin{\beta}$, find the value of $\tan{\alpha+\beta}$.\hfill(16 marks)
        
        \hrulefill
            
            \hrulefill
            
            \hrulefill
            
            \hrulefill
            
            \hrulefill
            
            \hrulefill
            
            \hrulefill
            
            \hrulefill
            
            \hrulefill
            
            \hrulefill
            
            \hrulefill
            
            \hrulefill
            
            \hrulefill
            
            \hrulefill
            
            \hrulefill
            
            \hrulefill
            
            \hrulefill
            
            \hrulefill
            
            \hrulefill
            
            \hrulefill
            
            \hrulefill
            
            \hrulefill
            
            \hrulefill
            
            \hrulefill
            
            \hrulefill

        \pagebreak
        \item (1996-CE-A MATH 1 \#02) Find $\dfrac{d}{dx}(x^2)$ from first principles.\hfill(8 marks)
        
        \hrulefill
            
            \hrulefill
            
            \hrulefill
            
            \hrulefill
            
            \hrulefill
            
            \hrulefill
            
            \hrulefill
            
            \hrulefill
            
            \hrulefill
            
            \hrulefill
            
            \hrulefill
            
            \hrulefill
            
            \hrulefill
            
            \hrulefill
        \item (2002-CE-A MATH \#03) Let $x\sin{y}=2002$. Find $\dfrac{dy}{dx}$.\hfill(8 marks)
        
            \hrulefill
            
            \hrulefill
            
            \hrulefill
            
            \hrulefill
            
            \hrulefill
            
            \hrulefill
            
            \hrulefill
            
            \hrulefill
            
            \hrulefill
            
            \hrulefill
            
            \hrulefill

        \pagebreak
        \item (2015-DSE-MATH-EP(M2) \#02) Let $y=x\sin{x}+\cos{x}$. \begin{enumerate}
            \item Find $\dfrac{dy}{dx}$ and $\dfrac{d^2y}{dx^2}$.
            \item Let $k$ be a constant such that $x\dfrac{d^2y}{dx^2}+k\dfrac{dy}{dx}+xy=0$ for all real values of $x$. Find the value of $k$.
        \end{enumerate}\hfill(14 marks)

        \hrulefill
            
            \hrulefill
            
            \hrulefill
            
            \hrulefill
            
            \hrulefill
            
            \hrulefill
            
            \hrulefill
            
            \hrulefill
            
            \hrulefill
            
            \hrulefill
            
            \hrulefill
            
            \hrulefill
            
            \hrulefill
            
            \hrulefill
            
            \hrulefill
            
            \hrulefill
            
            \hrulefill
            
            \hrulefill
            
            \hrulefill
            
            \hrulefill
            
            \hrulefill
            
            \hrulefill
            
            \hrulefill

        \pagebreak
        \item (2010-CE-A MATH \#10) It is given that $P$ is a point on the curve $C:y=x^3$. If the y-intercept of the tangent line $L$ to $C$ at $P$ is $-16$, find the equation of $L$.\hfill(14 marks)
        
        \hrulefill
            
            \hrulefill
            
            \hrulefill
            
            \hrulefill
            
            \hrulefill
            
            \hrulefill
            
            \hrulefill
            
            \hrulefill
            
            \hrulefill
            
            \hrulefill
            
            \hrulefill
            
            \hrulefill
            
            \hrulefill
            
            \hrulefill
            
            \hrulefill
            
            \hrulefill
            
            \hrulefill
            
            \hrulefill
            
            \hrulefill
            
            \hrulefill
            
            \hrulefill
            
            \hrulefill
            
            \hrulefill
            
            \hrulefill
            
            \hrulefill

        \pagebreak
    \end{enumerate}
\end{document}