\documentclass[12pt]{article}
\usepackage{ctex}
\usepackage[english]{babel}
\usepackage{blindtext}
\usepackage{nameref}
\usepackage{fancyhdr}
\usepackage{amsmath,amssymb,amsthm}
\usepackage{graphicx,float}
\usepackage{physics}
\usepackage{pgfplots}
\usepackage[a4paper, total={6in, 9in}]{geometry}

\pagestyle{fancy}
\fancyhf{}
\fancyhf[HL]{F4 第二級 M2 模擬試題}
\fancyhf[HR]{限時:1小時30分鐘}
\fancyhf[CF]{\thepage}

\newcommand{\innerprod}[2]{\langle{#1},{#2}\rangle}
\newcommand{\id}{\mathtt{id}}

\newtheorem*{definition}{Definition}
\newtheorem*{theorem}{Theorem}
\newtheorem*{corollary}{Corollary}
\newtheorem*{lemma}{Lemma}
\newtheorem*{proposition}{Proposition}
\newtheorem*{remark}{Remark}
\newtheorem*{claim}{Claim}
\newtheorem*{example}{Example}
\newtheorem*{axiom}{Axiom}

\begin{document}
    \thispagestyle{plain}

    \centering 

    \section*{練習試卷\\數學延伸單元\\單元2 (代數與微積分)\\試題-答題簿}

    限時: 1.5 小時

    姓名:\hrulefill \hfill 得分:\hrulefill/100

    學校:\hrulefill

    \raggedright

    \subsection*{規則}

    \begin{enumerate}
        \item 此試卷必須使用中文回答。
        \item 除特別指明外,需詳細列出所有算式。
        \item 除特別指明外,數值答案必須用真確值表示。
        \item 本試卷只作\textbf{内部使用}。
        \item 所有試題取自AL/CE/DSE歷届試題,來源: https://www.dse.life/ppindex/m2/
    \end{enumerate}
    \newpage
    \begin{enumerate}
        \item 運用數學歸納法,證明對於所有正整數$n$,$$1+2+3+\cdots+n=\frac{n(n+1)}{2}.$$\hfill(10分)
        
            \hrulefill
            
            \hrulefill
            
            \hrulefill
            
            \hrulefill
            
            \hrulefill
            
            \hrulefill
            
            \hrulefill
            
            \hrulefill
            
            \hrulefill
            
            \hrulefill
            
            \hrulefill
            
            \hrulefill
            
            \hrulefill
            
            \hrulefill
            
            \hrulefill
            
            \hrulefill
            
            \hrulefill
            
            \hrulefill
            
            \hrulefill
            
            \hrulefill
            
            \hrulefill
            
            \hrulefill
            
            \hrulefill
            
            \hrulefill
            
            \hrulefill

        \pagebreak
        \item (2012-DSE-MATH-EP(M2) \#03) 運用數學歸納法,證明對於所有正整數$n$,$$1\times2+2\times5+\cdots+n(3n-1)=n^2(n+1)$$\hfill(10分)
        
        \hrulefill
            
        \hrulefill
        
        \hrulefill
        
        \hrulefill
        
        \hrulefill
        
        \hrulefill
        
        \hrulefill
        
        \hrulefill
        
        \hrulefill
        
        \hrulefill
        
        \hrulefill
        
        \hrulefill
        
        \hrulefill
        
        \hrulefill
        
        \hrulefill
        
        \hrulefill
        
        \hrulefill
        
        \hrulefill
        
        \hrulefill
        
        \hrulefill
        
        \hrulefill
        
        \hrulefill
        
        \hrulefill
        
        \hrulefill
        
        \hrulefill

    \pagebreak
        \item (2004-CE-A MATH \#02)\begin{enumerate}
            \item 按$x$的升冪展開$(1+2x)^n$至$x^3$項,其中$n$為正整數。
            \item 若展開式$(x-\frac{3}{x})^2(1+2x)^n$的常數項為 210,求$n$的值。
        \end{enumerate}\hfill(12分)

            \hrulefill
            
            \hrulefill
            
            \hrulefill
            
            \hrulefill
            
            \hrulefill
            
            \hrulefill
            
            \hrulefill
            
            \hrulefill
            
            \hrulefill
            
            \hrulefill
            
            \hrulefill
            
            \hrulefill
            
            \hrulefill
            
            \hrulefill
            
            \hrulefill
            
            \hrulefill
            
            \hrulefill
            
            \hrulefill
            
            \hrulefill
            
            \hrulefill
            
            \hrulefill
            
            \hrulefill

        \pagebreak
        \item (1990-CE-A MATH 2 \#06(a)) 若$\cos{\theta}+\sqrt{3}\sin{\theta}=r\cos(\theta-\alpha)$,其中$r>0$及$0^\circ\leq \alpha\leq 90^\circ$。求$r$和$\alpha$的值\hfill(8分)
        
        \hrulefill
            
            \hrulefill
            
            \hrulefill
            
            \hrulefill
            
            \hrulefill
            
            \hrulefill
            
            \hrulefill
            
            \hrulefill
            
            \hrulefill
            
            \hrulefill
            
            \hrulefill
            
            \hrulefill
            
            \hrulefill
            
            \hrulefill
            
            \hrulefill
            
            \hrulefill
            
            \hrulefill
            
            \hrulefill
            
            \hrulefill
            
            \hrulefill
            
            \hrulefill
            
            \hrulefill
            
            \hrulefill
            
            \hrulefill
            
            \hrulefill
            
            \hrulefill
            
            \hrulefill

        \pagebreak
        \item (2003-CE-A MATH \#10) 已知$\alpha$及$\beta$為銳角。證明 $$\frac{\sin{\alpha}+\sin{\beta}}{\cos{\alpha}+\cos{\beta}}=\tan{\frac{\alpha+\beta}{2}}$$若$3\sin{\alpha}-4\cos{\alpha}=4\cos{\beta}-3\sin{\beta}$求$\tan{\alpha+\beta}$的值\hfill(16分)
        
        \hrulefill
            
            \hrulefill
            
            \hrulefill
            
            \hrulefill
            
            \hrulefill
            
            \hrulefill
            
            \hrulefill
            
            \hrulefill
            
            \hrulefill
            
            \hrulefill
            
            \hrulefill
            
            \hrulefill
            
            \hrulefill
            
            \hrulefill
            
            \hrulefill
            
            \hrulefill
            
            \hrulefill
            
            \hrulefill
            
            \hrulefill
            
            \hrulefill
            
            \hrulefill
            
            \hrulefill
            
            \hrulefill
            
            \hrulefill
            
            \hrulefill

        \pagebreak
        \item (1996-CE-A MATH 1 \#02) 從基本原理求導$\dfrac{d}{dx}(x^2)$。\hfill(8 分)
        
        \hrulefill
            
            \hrulefill
            
            \hrulefill
            
            \hrulefill
            
            \hrulefill
            
            \hrulefill
            
            \hrulefill
            
            \hrulefill
            
            \hrulefill
            
            \hrulefill
            
            \hrulefill
            
            \hrulefill
            
            \hrulefill
            
            \hrulefill
        \item (2002-CE-A MATH \#03) 設$x\sin{y}=2002$,求$\dfrac{dy}{dx}$。\hfill(8分)
        
            \hrulefill
            
            \hrulefill
            
            \hrulefill
            
            \hrulefill
            
            \hrulefill
            
            \hrulefill
            
            \hrulefill
            
            \hrulefill
            
            \hrulefill
            
            \hrulefill
            
            \hrulefill

        \pagebreak
        \item (2015-DSE-MATH-EP(M2) \#02) 設$y=x\sin{x}+\cos{x}$。 \begin{enumerate}
            \item 求$\dfrac{dy}{dx}$ 及$\dfrac{d^2y}{dx^2}$.
            \item 設$k$為常數使得對於所有實數$x$,均有$x\dfrac{d^2y}{dx^2}+k\dfrac{dy}{dx}+xy=0$。求$k$的值。
        \end{enumerate}\hfill(14分)

        \hrulefill
            
            \hrulefill
            
            \hrulefill
            
            \hrulefill
            
            \hrulefill
            
            \hrulefill
            
            \hrulefill
            
            \hrulefill
            
            \hrulefill
            
            \hrulefill
            
            \hrulefill
            
            \hrulefill
            
            \hrulefill
            
            \hrulefill
            
            \hrulefill
            
            \hrulefill
            
            \hrulefill
            
            \hrulefill
            
            \hrulefill
            
            \hrulefill
            
            \hrulefill
            
            \hrulefill
            
            \hrulefill

        \pagebreak
        \item (2010-CE-A MATH \#10) 已知 $P$ 為曲綫 $C:y=x^3$上的一點。 若$C$在$P$點上的切綫$L$的y軸截距為$-16$,求$L$的方程。\hfill(14分)
        
        \hrulefill
            
            \hrulefill
            
            \hrulefill
            
            \hrulefill
            
            \hrulefill
            
            \hrulefill
            
            \hrulefill
            
            \hrulefill
            
            \hrulefill
            
            \hrulefill
            
            \hrulefill
            
            \hrulefill
            
            \hrulefill
            
            \hrulefill
            
            \hrulefill
            
            \hrulefill
            
            \hrulefill
            
            \hrulefill
            
            \hrulefill
            
            \hrulefill
            
            \hrulefill
            
            \hrulefill
            
            \hrulefill
            
            \hrulefill
            
            \hrulefill

        \pagebreak
    \end{enumerate}
\end{document}