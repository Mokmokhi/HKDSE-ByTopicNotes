\documentclass[12pt]{article}
\usepackage{ctex}
\usepackage[english]{babel}
\usepackage{blindtext}
\usepackage{nameref}
\usepackage{fancyhdr}
\usepackage{amsmath,amssymb,amsthm}
\usepackage{graphicx,float}
\usepackage{physics}
\usepackage{pgfplots}
\usepackage[a4paper, total={6in, 9in}]{geometry}

\pagestyle{fancy}
\fancyhf{}
\fancyhf[HL]{F4 Level 2 M2 mock paper}
\fancyhf[CF]{\thepage}

\newcommand{\innerprod}[2]{\langle{#1},{#2}\rangle}
\newcommand{\id}{\mathtt{id}}

\newtheorem{definition}{定義}
\newtheorem*{theorem}{定理}
\newtheorem*{corollary}{衍理}
\newtheorem*{lemma}{引理}
\newtheorem*{proposition}{設理}
\newtheorem*{remark}{小記}
\newtheorem*{claim}{主張}
\newtheorem*{example}{例子}
\newtheorem*{axiom}{公設}
\renewenvironment*{proof}{\textit{證明.}}{\hfill$\qed$}

\newenvironment*{sol}{\par \textbf{解}.}{\hfill$\blacksquare$}

\begin{document}
    \begin{enumerate}
        \item \begin{align*}
            P(1): 1=\frac{1(1+1)}{2}&\\
            P(n)\implies P(n+1):&\\
            1+2+3+\cdots+n+(n+1)&=\frac{n(n+1)}{2}+(n+1)\\
            &=(n+1)(\frac{n}{2}+1)\\
            &=\frac{(n+1)(n+2)}{2}
        \end{align*}
        \item \begin{align*}
            P(1): 1\times 2=1^2(1+1)\\
            P(n)\implies P(n+1):&\\
            1\times 2+2\times 5+\cdots+n(3n-1)+(n+1)(3n+2)&=n^2(n+1)+(n+1)(3n+2)\\
            &=(n+1)(n^2+3n+2)\\
            &=(n+1)(n+1)(n+2)\\
            &=(n+1)^2(n+2)
        \end{align*}
        \item \begin{enumerate}
            \item $(1+2x)^n=1+2nx+4C_2^nx^2+8C_3^n+\cdots$.
            \item \begin{align*}
                (x-\dfrac{3}{x})^2(1+2x)^n&=(x^2-6+\frac{9}{x^2})(1+2nx+4C_2^nx^2+8C_3^n+\cdots)\\
                -6+36C_2^n&=210\\
                C_2^n&=6\\
                n&=4
            \end{align*}
        \end{enumerate}
        \item \begin{align*}
            r\cos(\theta-\alpha)=r\cos{\theta}\cos{\theta}+r\sin{\theta}\sin{\theta}&=\cos{\theta}+\sqrt{3}\sin{\theta}\\
            \implies&\begin{cases}
                r\cos{\alpha}&=1\\
                r\sin{\alpha}&=\sqrt{3}
            \end{cases}
            \implies&\begin{cases}
                r&=\sqrt{1^2+\sqrt{3}^2}=2\\
                \alpha&=\arctan{\sqrt{3}}=\frac{\pi}{3}
            \end{cases}
        \end{align*}
        \item \begin{align*}
            \frac{\sin{\alpha}+\sin{\beta}}{\cos{\alpha}+\cos{\beta}}&=\frac{2\sin{\frac{\alpha+\beta}{2}}\cos{\frac{\alpha+\beta}{2}}}{2\cos{\frac{\alpha+\beta}{2}}\cos{\frac{\alpha+\beta}{2}}}\\
            &=\tan{\frac{\alpha+\beta}{2}}
        \end{align*}
        Note that \begin{align*}
            3\sin{\alpha}-4\cos{\alpha}&=4\cos{\beta}-3\cos{\beta}\\
            3(\sin{\alpha}+\cos{\beta})&=4(\cos{\alpha}+\cos{\beta})\\
            \tan(\frac{\alpha+\beta}{2})&=\frac{4}{3}\\
            \tan(\alpha+\beta)&=\frac{2\tan(\frac{\alpha+\beta}{2})}{1-\tan^2(\frac{\alpha+\beta}{2})}\\
            &=\frac{8/3}{1-16/9}\\
            &=-\frac{24}{7}
        \end{align*}
        \item \begin{align*}
            \dfrac{d}{dx}(x^2)&=\lim_{h\to 0}\frac{(x+h)^2-x^2}{h}\\
            &=\lim_{h\to 0}\frac{h(2x+h)}{h}\\
            &=\lim_{h\to 0}(2x+h)\\
            &=2x
        \end{align*}
        \item Differentiation\begin{align*}
            \sin{y}+x\cos{y}\dfrac{dy}{dx}&=0\\
            \dfrac{dy}{dx}&=-\frac{\tan{x}}{x}=-\frac{\sin{x}}{x\cos{x}}
        \end{align*}
        \item \begin{enumerate}
            \item $\dfrac{dy}{dx}=x\cos{x}$, $\dfrac{d^2y}{dx^2}=\cos{x}-x\sin{x}$.
            \item \begin{align*}
                x\cos{x}-x^2\sin{x}+kx\cos{x}+x(x\sin{x}+\cos{x})&=0\\
                k&=-2
            \end{align*}
        \end{enumerate}
        \item \begin{align*}
            y'&=3x^2\\
            x^3+16&=3x^2(x-0)\\
            x^3&=8\\
            x&=2\\
            L:&y=12x-16
        \end{align*}
    \end{enumerate}
\end{document}