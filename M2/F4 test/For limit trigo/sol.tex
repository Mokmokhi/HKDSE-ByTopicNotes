\documentclass[12pt]{article}
\usepackage{ctex}
\usepackage[english]{babel}
\usepackage{blindtext}
\usepackage{nameref}
\usepackage{fancyhdr}
\usepackage{amsmath,amssymb,amsthm}
\usepackage{graphicx,float}
\usepackage{physics}
\usepackage{pgfplots}
\usepackage[a4paper, total={6in, 9in}]{geometry}

\graphicspath{{../image/}}

\pagestyle{fancy}
\fancyhf{}
\fancyhf[HL]{測驗2參考答案}
\fancyhf[CF]{\thepage}

\newcommand{\innerprod}[2]{\langle{#1},{#2}\rangle}
\newcommand{\id}{\mathtt{id}}

\newtheorem{definition}{定義}
\newtheorem*{theorem}{定理}
\newtheorem*{corollary}{衍理}
\newtheorem*{lemma}{引理}
\newtheorem*{proposition}{設理}
\newtheorem*{remark}{小記}
\newtheorem*{claim}{主張}
\newtheorem*{example}{例子}
\newtheorem*{axiom}{公設}
\renewenvironment*{proof}{\textit{證明.}}{\hfill$\qed$}

\newenvironment*{sol}{\par \textbf{解}.}{\hfill$\blacksquare$}

\begin{document}
    \begin{enumerate}
        \item \begin{enumerate}
            \item \begin{flalign*}
                e^{i2x}&=\cos^2{x}-\sin^2{x}+i2\sin{x}\cos{x}\\
                e^{i3x}&=4\cos^3{x}-3\cos{x}+i(3\sin{x}-4\sin^3{x})
            \end{flalign*}
            \item Remember to do both directions.
            \item \begin{enumerate}
                \item \begin{flalign*}
                \sum_{k=0}^{n}\cos{kx}&=\Re{\sum_{k=0}^{n}e^{ikx}}\\
                &=\Re{\frac{1-e^{i(n+1)x}}{1-e^{ix}}}
            \end{flalign*}
            \item \begin{flalign*}
                \sum_{k=0}^{n}\sin{kx}&=\Im{\sum_{k=0}^{n}e^{ikx}}\\
                &=\Im{\frac{1-e^{i(n+1)x}}{1-e^{ix}}}
            \end{flalign*}
            \item \begin{flalign*}
                \sum_{k=0}^{n}r^k\cos{kx}&=\Re{\sum_{k=0}^{n}r^ke^{ikx}}\\
                &=\Re{\frac{1-r^{n+1}e^{i(n+1)x}}{1-re^{ix}}}
            \end{flalign*}
            \item \begin{flalign*}
                \sum_{k=0}^{n}r^k\sin{kx}&=\Im{\sum_{k=0}^{n}r^ke^{ikx}}\\
                &=\Im{\frac{1-r^{n+1}e^{i(n+1)x}}{1-re^{ix}}}
            \end{flalign*}
            \end{enumerate}
            \item \begin{enumerate}
                \item \begin{flalign*}
                    \sum_{k=0}^{\infty}r^k\cos{kx}&=\Re{\sum_{k=0}^{\infty}r^ke^{ikx}}\\
                    &=\Re{\frac{1}{1-re^{ix}}}
                \end{flalign*}
                \item \begin{flalign*}
                    \sum_{k=0}^{\infty}r^k\sin{kx}&=\Im{\sum_{k=0}^{n}r^ke^{ikx}}\\
                    &=\Im{\frac{1}{1-re^{ix}}}
                \end{flalign*}
            \end{enumerate}
        \end{enumerate}
        \item \begin{enumerate}
            \item \begin{enumerate}
                \item Taylor
                \item Taylor
                \item Taylor
            \end{enumerate}
            \item 弧長等於弧度角,并且弧長永遠大於垂直高度。
            \item 利用極限。
        \end{enumerate}
        \item \begin{enumerate}
            \item $\lim_{x\to a}(f(x)-f(a))=\lim_{x\to a}D_af(x)\cdot \lim_{x\to a}(x-a)=0$.
            \item 沒有一點有導數,則為離散函數。
        \end{enumerate}
        \item \begin{enumerate}
            \item 0
            \item 代$x=r\cos{\theta}, y=r\sin{\theta}$.
        \end{enumerate}
    \end{enumerate}
\end{document}