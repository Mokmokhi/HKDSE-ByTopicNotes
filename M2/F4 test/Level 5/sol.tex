\documentclass[12pt]{article}
\usepackage{ctex}
\usepackage[english]{babel}
\usepackage{blindtext}
\usepackage{nameref}
\usepackage{fancyhdr}
\usepackage{amsmath,amssymb,amsthm}
\usepackage{graphicx,float}
\usepackage{physics}
\usepackage{pgfplots}
\usepackage[a4paper, total={6in, 9in}]{geometry}

\pagestyle{fancy}
\fancyhf{}
\fancyhf[HL]{F4 M2 Level 5 solution}
\fancyhf[CF]{\thepage}

\newcommand{\innerprod}[2]{\langle{#1},{#2}\rangle}
\newcommand{\id}{\mathtt{id}}

\newtheorem*{definition}{Definition}
\newtheorem*{theorem}{Theorem}
\newtheorem*{corollary}{Corollary}
\newtheorem*{lemma}{Lemma}
\newtheorem*{proposition}{Proposition}
\newtheorem*{remark}{Remark}
\newtheorem*{claim}{Claim}
\newtheorem*{example}{Example}
\newtheorem*{axiom}{Axiom}

\newenvironment*{sol}{\par \textbf{Solution}.}{\hfill$\blacksquare$}

\begin{document}
    \begin{enumerate}
        \item \begin{sol}
            Let $P(n)$ be the proposition $$\sum_{k=1}^{n}T_k=n[(n+1)!]$$ where $T_n=(n^2+1)(n!)$ for any positive integer $n$.

            For $P(1)$:\begin{align*}
                \textrm{L.H.S.}&=\sum_{k=1}^{1}T_k\\
                &=T_1\\
                &=(1^2+1)(1!)\\
                &=2\\
                \textrm{R.H.S.}&=1[(1+1)!]\\
                &=2
            \end{align*}
            $\therefore$ L.H.S=R.H.S.

            $\therefore P(1)$ is true.
            
            Assume $P(m)$ is true for some positive integer $m$, i.e. $$\sum_{k=1}^{m}T_k=m[(m+1)!]$$

            Then for $P(m+1)$:\begin{align*}
                \sum_{k=1}^{m+1}T_k&=m[(m+1)!]+[(m+1)^2+1][(m+1)!]\\
                &=[m^2+2m+2+m][(m+1)!]\\
                &=(m+1)(m+2)[(m+1)!]\\
                &=(m+1)[(m+2)!]
            \end{align*}

            $\therefore P(m+1)$ is true when $P(m)$ is true.

            Thus, by the principle of Mathematical Induction, $P(n)$ is true for all positive integer $n$.
        \end{sol}
        \item \begin{enumerate}
            \item \begin{sol}
                Let $P(n)$ be the proposition $$A_n=(-1)^{n-1}B_n$$ where $A_n=\sum_{k=1}^{n}(-1)^{k-1}k^2$ and $B_n=\dfrac{n(n+1)}{2}$ for any positive integer $n$.
        
                For $P(1)$:\begin{align*}
                    \textrm{L.H.S.}&=A_1\\
                    &=\sum_{k=1}^{1}(-1)^{k-1}k^2\\
                    &=(-1)^0 1^2\\
                    &=(-1)^0 1\\
                    &=(-1)^{1-1}\dfrac{1(1+1)}{2}\\
                    &=\textrm{R.H.S.}
                \end{align*}
                $\therefore$ L.H.S=R.H.S.
        
                $\therefore P(1)$ is true.
                
                Assume $P(m)$ is true for some positive integer $m$, i.e. $$A_m=(-1)^{m-1}B_m$$
        
                Then for $P(m+1)$:\begin{align*}
                    A_{m+1}&=\sum_{k=1}^{m+1}(-1)^{k-1}k^2\\
                    &=(-1)^{m-1}\dfrac{m(m+1)}{2}+(-1)^m(m+1)^2\\
                    &=(-1)^m(m+1)[(m+1)-\frac{m}{2}]\\
                    &=(-1)^m(m+1)\frac{m+2}{2}\\
                    &=(-1)^m B_{m+1}
                \end{align*}
        
                $\therefore P(m+1)$ is true when $P(m)$ is true.
        
                Thus, by the principle of Mathematical Induction, $P(n)$ is true for all positive integer $n$.
            \end{sol}
            \item \begin{sol}
                \begin{align*}
                    \sum_{n=1}^{2m}A_n&=\sum_{n=1}^{2m}(-1)^{n-1}B_n\\
                    &=\sum_{n=1}^{m}(B_{2n-1}-B_{2n})\\
                    &=\sum_{n=1}^{m}(-2n)\\
                    &=-2\sum_{n=1}^{m}n\\
                    &=-2 B_m\\
                    &=-m(m+1)\\
                    \sum_{n=1}^{2m+1}A_n&=-m(m+1)+A_{2m+1}\\
                    &=-m(m+1)+B_{2m+1}\\
                    &=-m(m+1)+(m+1)(2m+1)\\
                    &=(m+1)^2
                \end{align*}
            \end{sol}
        \end{enumerate}
        \item \begin{sol}
            By considering the corresponding coefficients, we have
            \begin{align*}
                \lambda_1&=C_1^8a=8a\\
                \lambda_2&=C_2^8a^2=28a^2\\
                \mu_7&=C_7^9b^2=36b^2\\
                \mu_8&=C_8^9b=9b
            \end{align*}
            That means we have to solve the folloing system:\begin{align*}
                \begin{cases}
                    \frac{28a^2}{36b^2}=\frac{7}{4}\implies 4a^2=9b^2&\implies 2a=\pm 3b\\
                    8a+9b+6=0
                \end{cases}
            \end{align*}
            If $2a=3b$, we have \begin{align*}
                12b+9b+6&=0\\
                b=-\frac{2}{7} &,a=-\frac{3}{7}
            \end{align*}
            If $2a=-3b$, we have \begin{align*}
                -12b+9b+6&=0\\
                b=-2 &,a=3
            \end{align*}
            Then $a=-\dfrac{3}{7}$ or $a=3$.
        \end{sol}
        \item \begin{enumerate}
            \item \begin{sol}
                We have the fundamental identities:\begin{align*}
                    \sin{2x}&=2\sin{x}\cos{x}\\
                    \sin{3x}&=3\sin{x}-4\sin^3{x}\\
                \end{align*}
                Then \begin{align*}
                    \sin{x}-\sin{2x}+\sin{3x}&=0\\
                    \sin{x}(1-2\cos{x}+3-4\sin^2{x})&=0\\
                \end{align*}
                One solution is $\sin{x}=0$ which means $x=\pi$. Otherwise,\begin{align*}
                    1-2\cos{x}+3-4\sin^2{x}&=0\\
                    -2\cos{x}+4\cos^2{x}&=0\\
                    -2\cos{x}(1-2\cos{x})&=0\\
                \end{align*}
                Then, either $\cos{x}=0\implies x=\pi/2,3\pi/2$ or $\cos{x}=1/2\implies x=\pi/3,5\pi/3$.

                In conclusion, the set of solution to the equation is $\{\pi/3,\pi/2,\pi,3\pi/2,5\pi/3\}$.
            \end{sol}
            \item \begin{enumerate}
                \item \begin{sol}
                    \begin{align*}
                        f(\theta)&=\sin{2\theta}+\sin{\theta}+\cos{\theta}\\
                        &=2\sin{\theta}\cos{\theta}+\sin{\theta}+\cos{\theta}\\
                        &=\sin^2{\theta}+2\sin{\theta}\cos{\theta}+\cos^2{\theta}+\sin{\theta}+\cos{\theta}-1\\
                        &=p^2+p-1
                    \end{align*}
                \end{sol}
                \item \begin{sol}
                    $f(\theta)=(p+1/2)^2-5/4$. So $f(\theta)$ has the minimum value $-5/4$. Such $\theta$ can be computed as follows:\begin{align*}
                        p+\frac{1}{2}&=0\\
                        p&=-\frac{1}{2}\\
                        \sin{2\theta}-\frac{1}{2}&=-\frac{5}{4}\\
                        \sin{2\theta}&=-\frac{3}{4}\\
                    \end{align*}

                    Case 1:
                    \begin{align*}
                        \tan{2\theta}&=\frac{3}{\sqrt{7}}\\
                        \frac{2\tan{\theta}}{1-\tan^2{\theta}}&=\frac{3}{\sqrt{7}}\\
                        2\sqrt{7}\tan{\theta}&=3-3\tan^2{\theta}\\
                        3\tan^2{\theta}+2\sqrt{7}\tan{\theta}-3&=0\\
                        \tan{\theta}&=\frac{-2\sqrt{7}\pm\sqrt{28+36}}{6}\\
                        &=\frac{-\sqrt{7}\pm 4}{3}
                    \end{align*}
                    By $\tan{\theta}<0$, $\theta=\tan^{-1}(-\frac{4+\sqrt{7}}{3})$.

                    Case 2:
                    \begin{align*}
                        \tan{2\theta}&=-\frac{3}{\sqrt{7}}\\
                        \frac{2\tan{\theta}}{1-\tan^2{\theta}}&=-\frac{3}{\sqrt{7}}\\
                        -2\sqrt{7}\tan{\theta}&=3-3\tan^2{\theta}\\
                        3\tan^2{\theta}-2\sqrt{7}\tan{\theta}-3&=0\\
                        \tan{\theta}&=\frac{2\sqrt{7}\pm\sqrt{28+36}}{6}\\
                        &=\frac{\sqrt{7}\pm 4}{3}
                    \end{align*}
                    By $\tan{\theta}<0$, $\theta=\tan^{-1}(\frac{\sqrt{7}-4}{3})$.

                    Hence the conclusion is $f$ attains minimum at $\theta=\tan^{-1}(\frac{-4\pm\sqrt{7}}{3})$.
                \end{sol}
            \end{enumerate}
        \end{enumerate}
        \item \begin{enumerate}
            \item \begin{sol}
                \begin{align*}
                    \tan{4\theta}&=\frac{2\tan{2\theta}}{1-\tan^2{2\theta}}\\
                    &=\frac{2\frac{2\tan{\theta}}{1-\tan^2{\theta}}}{1-(\frac{2\tan{\theta}}{1-\tan^2{\theta}})^2}\\
                    &=\frac{4\tan{\theta}(1-\tan^2{\theta})}{(1-\tan^2{\theta})^2-4\tan^2{\theta}}\\
                    &=\frac{4\tan{\theta}-4\tan^3{\theta}}{1-6\tan^2{\theta}+\tan^4{\theta}}
                \end{align*}

                Then \begin{align*}
                    \cot{4\theta}&=\frac{1}{\tan{4\theta}}\\
                    &=\frac{1-6\tan^2{\theta}+\tan^4{\theta}}{4\tan{\theta}-4\tan^3{\theta}}\\
                    &=\frac{\cot^4{\theta}-6\cot^2{\theta}+1}{4\cot^3{\theta}-4\cot{\theta}}
                \end{align*}

                Hence $\cot{\theta}$ solve the equation $x^4-4x^3-6x^2+4x+1=0$ when $\cot{4\theta}=1$, which is \begin{align*}
                    4\theta&=n\pi+\pi/4\\
                    \theta&=n\pi/4+\pi/16\\
                    x&=\cot(\pi/16),\cot(5\pi/16),\cot(9\pi/16),\cot(13\pi/16)
                \end{align*}
                restricting to $[0,2\pi]$.
            \end{sol}
            \item \begin{enumerate}
                \item \begin{sol}
                    \begin{align*}
                        \frac{1}{2}(2-a^2-b^2)&=\frac{1}{2}(2-\cos^2{\theta}+2\cos{\theta}\cos{\phi}-\cos^2{\phi}-\sin^2{\theta}+2\sin{\theta}\sin{\phi}-\sin^2{\phi})\\
                        &=\cos{\theta}\cos{\phi}+\sin{\theta}\sin{\phi}\\
                        &=\cos{(\theta-\phi)}
                    \end{align*}
                    and \begin{align*}
                        \frac{-a}{b}&=\frac{\cos{\phi}-\cos{\theta}}{\sin{\theta}-\sin{\phi}}\\
                        &=\frac{-2\sin{\frac{\phi+\theta}{2}}\sin{\frac{\phi-\theta}{2}}}{2\cos{\frac{\theta+\phi}{2}}\sin{\frac{\theta-\phi}{2}}}\\
                        &=\tan{\frac{\theta+\phi}{2}}
                    \end{align*}
                \end{sol}
                \item \begin{sol}
                    By i.\begin{align*}
                        a=1&,b=\sqrt{3}\\
                        \implies&\begin{cases}
                            \cos{(\theta-\phi)}=-1\\
                            \tan{\frac{\theta+\phi}{2}}=-\frac{1}{\sqrt{3}}
                        \end{cases}\\
                        \implies&\begin{cases}
                            \theta-\phi=2n\pi-\pi/2\\
                            \theta+\phi=2n\pi-\pi/3
                        \end{cases}\\
                        \implies&\theta=7\pi/12, \phi=11\pi/12
                    \end{align*}
                \end{sol}
            \end{enumerate}
        \end{enumerate}
        \item \begin{sol}
            Let $f(x)=(x^2-1)e^x$. Then \begin{align*}
                f(1+h)&=[(1+h)^2-1]e^{1+h}=h(2+h)e^{1+h}\\
                f'(1)&=\lim_{h\to 0}\frac{f(1+h)-f(1)}{h}\\
                &=\lim_{h\to 0}\frac{h(2+h)e^{1+h}}{h}\\
                &=\lim_{h\to 0}(2+h)e^{1+h}\\
                &=2e
            \end{align*}
        \end{sol}
        \item \begin{enumerate}
            \item \begin{sol}
                \begin{align*}
                    f'(x)&=e^x(\sin{x}+\cos{x})+e^x(\cos{x}-\sin{x})\\
                    &=2e^x\cos{x}\\
                    f''(x)&=2e^x\cos{x}-2e^x\sin{x}\\
                    &=2e^x(\cos{x}-\sin{x})
                \end{align*}
            \end{sol}
            \item \begin{sol}
                \begin{align*}
                    f''(x)-f'(x)+f(x)&=0\\
                    2e^x(\cos{x}-\sin{x})-2e^x\cos{x}+e^x(\sin{x}+\cos{x})&=0\\
                    \cos{x}-\sin{x}&=0\\
                    \tan{x}&=1\\
                    x&=n\pi+\pi/4
                \end{align*}
                where $n\in \mathbb{Z}$ is integer.
            \end{sol}
        \end{enumerate}
        \item \begin{sol}
            \begin{align*}
                \frac{dy}{dx}=\frac{-6}{(x+1)^2}&=-\frac{1}{6}\\
                (x+1)^2&=36\\
                x=5,-7
            \end{align*}

            Then the equation of tangent to $C$ at $(5,1)$ is \begin{align*}
                y-1&=-\frac{1}{6}(x-5)\\
                x+6y-11&=0
            \end{align*}

            Then the equation of tangent to $C$ at $(-7,-1)$ is \begin{align*}
                y+1&=-\frac{1}{6}(x+7)\\
                x+6y+13&=0
            \end{align*}
        \end{sol}
        \item \begin{enumerate}
            \item \begin{enumerate}
                \item \begin{sol}
                    $x=4\sin{\theta}$.
                \end{sol}
                \item \begin{sol}
                    Since $\dfrac{dx}{dt}=\dfrac{1}{2}$, we have \begin{align*}
                        \frac{dx}{dt}&=4\cos{\theta}\frac{d\theta}{dt}\\
                        \frac{d\theta}{dt}&=\frac{1}{8\cos{\theta}}
                    \end{align*}
                \end{sol}
            \end{enumerate}
            \item \begin{enumerate}
                \item \begin{sol}
                    Write $y=4\cos{\theta}$ and $z=\sqrt{5^2-x^2}=\sqrt{25-16\sin^2{\theta}}$. Then \begin{align*}
                        \frac{dy}{dt}&=-4\sin{\theta}\frac{d\theta}{dt}\\
                        &=-4\sin{\theta}\frac{1}{8\cos{\theta}}\\
                        &=-\frac{1}{2}\tan{\theta}\\
                        \frac{dz}{dt}&=\frac{-32\sin{\theta}\cos{\theta}}{2\sqrt{25-16\sin^2{\theta}}}\frac{1}{8\cos{\theta}}\\
                        &=\frac{-2\sin{\theta}}{\sqrt{25-16\sin^2{\theta}}}
                    \end{align*}
                \end{sol}
                \item \begin{sol}
                    \begin{align*}
                        \frac{dPQ}{dt}|_{\theta=\pi/6}&=\frac{dy}{dt}|_{\theta=\pi/6}+\frac{dz}{dt}|_{\theta=\pi/6}\\
                        &=-\frac{1}{2}\tan{\pi/3}-\frac{2\sin{\pi/6}}{\sqrt{25-16\sin^2{\pi/6}}}\\
                        &=-\frac{\sqrt{3}}{2}-\frac{1}{\sqrt{25-4}}\\
                        &=-\frac{\sqrt{3}}{2}-\frac{\sqrt{21}}{21}
                    \end{align*}
                \end{sol}
            \end{enumerate}
            \item \begin{enumerate}
                \item \begin{sol}
                    Recall area of $\triangle OPR$ is $\dfrac{xy}{2}$,\begin{align*}
                        \frac{d}{dt}\frac{xy}{2}&=\frac{1}{2}(x\frac{dy}{dt}+y\frac{dx}{dt})\\
                        &=\frac{1}{2}(-2\sin{\theta}\tan{\theta}+2\cos{\theta})\\
                        &=\cos{\theta}-\sin{\theta}\tan{\theta}
                    \end{align*}

                    Set $\dfrac{d}{dt}\dfrac{xy}{2}=0$, then we have \begin{align*}
                        \tan^2{\theta}&=1\\
                        \tan{\theta}&=\pm1\\
                        \theta&=n\pi/2+\pi/4
                    \end{align*}

                    Since $0<\theta<\pi/2$, $\theta=\pi/4$.
                \end{sol}
                \item \begin{sol}
                    Let $\angle OQR=\phi$, then \begin{align*}
                        4\sin{\theta}&=5\sin{\phi}
                    \end{align*}

                    Similar to i., we have to have $\phi=\pi/4$ so that $\triangle ORQ$ is having maximum area. Then \begin{align*}
                        \theta&=\sin^{-1}(\frac{5\sin{\pi/4}}{4})\\
                        &=\sin^{-1}(\frac{5\sqrt{2}}{8})\\
                        &=1.08 rad
                    \end{align*}
                \end{sol}
            \end{enumerate}
        \end{enumerate}
    \end{enumerate}
\end{document}