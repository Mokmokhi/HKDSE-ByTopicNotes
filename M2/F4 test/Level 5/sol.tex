\documentclass[12pt]{article}
\usepackage{ctex}
\usepackage[english]{babel}
\usepackage{blindtext}
\usepackage{nameref}
\usepackage{fancyhdr}
\usepackage{amsmath,amssymb,amsthm}
\usepackage{graphicx,float}
\usepackage{physics}
\usepackage{pgfplots}
\usepackage[a4paper, total={6in, 9in}]{geometry}

\pagestyle{fancy}
\fancyhf{}
\fancyhf[HL]{F4 M2 Level 5 solution}
\fancyhf[CF]{\thepage}

\newcommand{\innerprod}[2]{\langle{#1},{#2}\rangle}
\newcommand{\id}{\mathtt{id}}

\newtheorem*{definition}{Definition}
\newtheorem*{theorem}{Theorem}
\newtheorem*{corollary}{Corollary}
\newtheorem*{lemma}{Lemma}
\newtheorem*{proposition}{Proposition}
\newtheorem*{remark}{Remark}
\newtheorem*{claim}{Claim}
\newtheorem*{example}{Example}
\newtheorem*{axiom}{Axiom}

\newenvironment*{sol}{\par \textbf{Solution}.}{\hfill$\blacksquare$}

\begin{document}
    \begin{enumerate}
        \item \begin{sol}
            Let $P(n)$ be the proposition $$\sum_{k=1}^{n}T_k=n[(n+1)!]$$ where $T_n=(n^2+1)(n!)$ for any positive integer $n$.

            For $P(1)$:\begin{align*}
                \textrm{L.H.S.}&=\sum_{k=1}^{1}T_k\\
                &=T_1\\
                &=(1^2+1)(1!)\\
                &=2\\
                \textrm{R.H.S.}&=1[(1+1)!]\\
                &=2
            \end{align*}
            $\therefore$ L.H.S=R.H.S.

            $\therefore P(1)$ is true.
            
            Assume $P(m)$ is true for some positive integer $m$, i.e. $$\sum_{k=1}^{m}T_k=m[(m+1)!]$$

            Then for $P(m+1)$:\begin{align*}
                \sum_{k=1}^{m+1}T_k&=m[(m+1)!]+[(m+1)^2+1][(m+1)!]\\
                &=[m^2+2m+2+m][(m+1)!]\\
                &=(m+1)(m+2)[(m+1)!]\\
                &=(m+1)[(m+2)!]
            \end{align*}

            $\therefore P(m+1)$ is true when $P(m)$ is true.

            Thus, by the principle of Mathematical Induction, $P(n)$ is true for all positive integer $n$.
        \end{sol}
        \item \begin{enumerate}
            \item \begin{sol}
                Let $P(n)$ be the proposition $$A_n=(-1)^{n-1}B_n$$ where $A_n=\sum_{k=1}^{n}(-1)^{k-1}k^2$ and $B_n=\dfrac{n(n+1)}{2}$ for any positive integer $n$.
        
                For $P(1)$:\begin{align*}
                    \textrm{L.H.S.}&=A_1\\
                    &=\sum_{k=1}^{1}(-1)^{k-1}k^2\\
                    &=(-1)^0 1^2\\
                    &=(-1)^0 1\\
                    &=(-1)^{1-1}\dfrac{1(1+1)}{2}\\
                    &=\textrm{R.H.S.}
                \end{align*}
                $\therefore$ L.H.S=R.H.S.
        
                $\therefore P(1)$ is true.
                
                Assume $P(m)$ is true for some positive integer $m$, i.e. $$A_m=(-1)^{m-1}B_m$$
        
                Then for $P(m+1)$:\begin{align*}
                    A_{m+1}&=\sum_{k=1}^{m+1}(-1)^{k-1}k^2\\
                    &=(-1)^{m-1}\dfrac{m(m+1)}{2}+(-1)^m(m+1)^2\\
                    &=(-1)^m(m+1)[(m+1)-\frac{m}{2}]\\
                    &=(-1)^m(m+1)\frac{m+2}{2}\\
                    &=(-1)^m B_{m+1}
                \end{align*}
        
                $\therefore P(m+1)$ is true when $P(m)$ is true.
        
                Thus, by the principle of Mathematical Induction, $P(n)$ is true for all positive integer $n$.
            \end{sol}
            \item \begin{sol}
                \begin{align*}
                    \sum_{n=1}^{2m}A_n&=\sum_{n=1}^{2m}(-1)^{n-1}B_n\\
                    &=\sum_{n=1}^{m}(B_{2n-1}-B_{2n})\\
                    &=\sum_{n=1}^{m}(-2n)\\
                    &=-2\sum_{n=1}^{m}n\\
                    &=-2 B_m\\
                    &=-m(m+1)\\
                    \sum_{n=1}^{2m+1}A_n&=-m(m+1)+A_{2m+1}\\
                    &=-m(m+1)+B_{2m+1}\\
                    &=-m(m+1)+(m+1)(2m+1)\\
                    &=(m+1)^2
                \end{align*}
            \end{sol}
        \end{enumerate}
        \item \begin{sol}
            By considering the corresponding coefficients, we have
            \begin{align*}
                \lambda_1&=C_1^8a=8a\\
                \lambda_2&=C_2^8a^2=28a^2\\
                \mu_7&=C_7^9b^2=36b^2\\
                \mu_8&=C_8^9b=9b
            \end{align*}
            That means we have to solve the folloing system:\begin{align*}
                \begin{cases}
                    \frac{28a^2}{36b^2}=\frac{7}{4}\implies 4a^2=9b^2&\implies 2a=\pm 3b\\
                    8a+9b+6=0
                \end{cases}
            \end{align*}
            If $2a=3b$, we have \begin{align*}
                12b+9b+6&=0\\
                b=-\frac{2}{7} &a=-\frac{3}{7}
            \end{align*}
            If $2a=-3b$, we have \begin{align*}
                -12b+9b+6&=0\\
                b=-2 &a=3
            \end{align*}
            Then $a=-\dfrac{3}{7}$ or $a=3$.
        \end{sol}
        \item \begin{enumerate}
            \item \begin{sol}
                We have the fundamental identities:\begin{align*}
                    \sin{2x}&=2\sin{x}\cos{x}\\
                    \sin{3x}&=3\sin{x}-4\sin^3{x}\\
                \end{align*}
                Then \begin{align*}
                    \sin{x}-\sin{2x}+\sin{3x}&=0\\
                    \sin{x}(1-2\cos{x}+3-4\sin^2{x})&=0\\
                \end{align*}
                One solution is $\sin{x}=0$ which means $x=\pi$. Otherwise,\begin{align*}
                    1-2\cos{x}+3-4\sin^2{x}&=0\\
                    -2\cos{x}+4\cos^2{x}&=0\\
                    -2\cos{x}(1-2\cos{x})&=0\\
                \end{align*}
                Then, either $\cos{x}=0\implies x=\pi/2,3\pi/2$ or $\cos{x}=1/2\implies x=\pi/3,5\pi/3$.

                In conclusion, the set of solution to the equation is $\{\pi/3,\pi/2,\pi,3\pi/2,5\pi/3\}$.
            \end{sol}
            \item \begin{enumerate}
                \item \begin{align*}
                    f(\theta)&=\sin{2\theta}+\sin{\theta}+\cos{\theta}\\
                    &=2\sin{\theta}\cos{\theta}+\sin{\theta}+\cos{\theta}\\
                    &=\sin^2{\theta}+2\sin{\theta}\cos{\theta}+\cos^2{\theta}+\sin{\theta}+\cos{\theta}-1\\
                    &=p^2+p-1
                \end{align*}
                \item $f(\theta)=(p+1/2)^2-5/4$. So $f(\theta)$ has the minimum value $-5/4$. Such $\theta$ can be computed as follows:\begin{align*}
                    p+\frac{1}{2}&=0\\
                    \sin{\theta}+\cos{\theta}+\frac{1}{2}&=0\\
                    1+\sin{2\theta}=\frac{1}{4}\\
                    
                \end{align*}
            \end{enumerate}
        \end{enumerate}
    \end{enumerate}
\end{document}