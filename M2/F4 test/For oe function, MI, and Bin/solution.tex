\documentclass[12pt]{article}
\usepackage{ctex}
\usepackage[english]{babel}
\usepackage{blindtext}
\usepackage{nameref}
\usepackage{fancyhdr}
\usepackage{amsmath,amssymb,amsthm}
\usepackage{graphicx,float}
\usepackage{physics}
\usepackage{pgfplots}
\usepackage[a4paper, total={6in, 9in}]{geometry}

\graphicspath{{../image/}}

\pagestyle{fancy}
\fancyhf{}
\fancyhf[HL]{測驗1參考答案}
\fancyhf[CF]{\thepage}

\newcommand{\innerprod}[2]{\langle{#1},{#2}\rangle}
\newcommand{\id}{\mathtt{id}}

\newtheorem{definition}{定義}
\newtheorem*{theorem}{定理}
\newtheorem*{corollary}{衍理}
\newtheorem*{lemma}{引理}
\newtheorem*{proposition}{設理}
\newtheorem*{remark}{小記}
\newtheorem*{claim}{主張}
\newtheorem*{example}{例子}
\newtheorem*{axiom}{公設}
\renewenvironment*{proof}{\textit{證明.}}{\hfill$\qed$}

\newenvironment*{sol}{\par \textbf{解}.}{\hfill$\blacksquare$}

\begin{document}
    \begin{enumerate}
        \item \begin{enumerate}
            \item \begin{align*}
                &O=\frac{f(x)-f(-x)}{2}&&E=\frac{f(x)+f(-x)}{2}
            \end{align*}
            \item \begin{align*}
                &i) \sum_{k\textrm{偶數}}C_k^n O^{n-k}E^k&&ii) \sum_{k\textrm{奇數}}C_k^n O^{n-k}E^k
            \end{align*}
        \end{enumerate}
        \item \begin{align*}
            (1+x)^n(1+x)&\geq (1+nx)(1+x)\\
            &=1+(1+n)x+nx^2\\
            &\geq 1+(n+1)x
        \end{align*}
        \item \begin{enumerate}
            \item \begin{align*}
                S_1&=\sum_{i=1}^{n}i\\
                \begin{pmatrix}
                    n\\n-1,1,0
                \end{pmatrix}a&=\frac{n(n+1)}{2}\\
                na&=\frac{n(n+1)}{2}\\
                a&=\frac{(n+1)}{2}\\
                S_2&=\sum_{i=1}^{n}i^2\\
                \begin{pmatrix}
                    n\\n-2,2,0
                \end{pmatrix}a^2+\begin{pmatrix}
                    n\\n-1,0,1
                \end{pmatrix}b&=\frac{n(n+1)(2n+1)}{6}\\
                \frac{n(n-1)}{2}a^2+nb&=\frac{n(n+1)(2n+1)}{6}\\
                b&=\frac{(n+1)(-3n^2+8n+7)}{24}
            \end{align*}
            \item \begin{align*}
                a:b&=3:1\\
                \frac{-3n^2+8n+7}{12}&=\frac{1}{3}\\
                -3n^2+8n+3&=0\\
                (3n+1)(n-3)&=0\\
                n&=3
            \end{align*}
        \end{enumerate}
    \end{enumerate}
\end{document}