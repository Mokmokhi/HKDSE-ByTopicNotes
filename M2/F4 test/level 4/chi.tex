\documentclass[12pt]{article}
\usepackage{ctex}
\usepackage[english]{babel}
\usepackage{blindtext}
\usepackage{nameref}
\usepackage{fancyhdr}
\usepackage{amsmath,amssymb,amsthm}
\usepackage{graphicx,float}
\usepackage{physics}
\usepackage{pgfplots}
\usepackage[a4paper, total={6in, 9in}]{geometry}

\pagestyle{fancy}
\fancyhf{}
\fancyhf[HL]{F4 第四級 M2 模擬試題}
\fancyhf[HR]{限時:1小時30分鐘}
\fancyhf[CF]{\thepage}

\newcommand{\innerprod}[2]{\langle{#1},{#2}\rangle}
\newcommand{\id}{\mathtt{id}}

\newtheorem*{definition}{Definition}
\newtheorem*{theorem}{Theorem}
\newtheorem*{corollary}{Corollary}
\newtheorem*{lemma}{Lemma}
\newtheorem*{proposition}{Proposition}
\newtheorem*{remark}{Remark}
\newtheorem*{claim}{Claim}
\newtheorem*{example}{Example}
\newtheorem*{axiom}{Axiom}

\begin{document}
    \thispagestyle{plain}

    \centering 

    \section*{練習試卷\\數學延伸單元\\單元2 (代數與微積分)\\試題-答題簿}

    限時: 1.5 小時

    姓名:\hrulefill \hfill 得分:\hrulefill/100

    學校:\hrulefill

    \raggedright

    \subsection*{規則}

    \begin{enumerate}
        \item 此試卷必須使用中文回答。
        \item 除特別指明外,需詳細列出所有算式。
        \item 除特別指明外,數值答案必須用真確值表示。
        \item 本試卷只作\textbf{内部使用}。
        \item 所有試題取自AL/CE/DSE歷届試題,來源: https://www.dse.life/ppindex/m2/
    \end{enumerate}

    \newpage
    \begin{enumerate}
        \item (1990-HL-GEN MATHS \#05)  \begin{enumerate}
            \item 運用數學歸納法,證明對於所有正整數$n$, $$\sum_{r=1}^{n}r^3=\frac{1}{4}n^2(n+1)^2$$
            \item 求 $1^3-2^3+3^3-4^3+\dots+(-1)^{r+1}r^3+\dots-(2n)^3$的值。
        \end{enumerate}\hfill(13分)
        
            \hrulefill
            
            \hrulefill
            
            \hrulefill
            
            \hrulefill
            
            \hrulefill
            
            \hrulefill
            
            \hrulefill
            
            \hrulefill
            
            \hrulefill
            
            \hrulefill
            
            \hrulefill
            
            \hrulefill
            
            \hrulefill
            
            \hrulefill
            
            \hrulefill
            
            \hrulefill
            
            \hrulefill
            
            \hrulefill
            
            \hrulefill
            
            \hrulefill
            
            \hrulefill

        \pagebreak
        \item (2010-CE-A MATH \#05) 若在$(1+4x)^n$的展開式中$x$及$x^2$的係數之和 是180,其中$n$是正整數。求$n$與$x^3$的係數的值。\hfill(10分)
        
        \hrulefill
            
        \hrulefill
        
        \hrulefill
        
        \hrulefill
        
        \hrulefill
        
        \hrulefill
        
        \hrulefill
        
        \hrulefill
        
        \hrulefill
        
        \hrulefill
        
        \hrulefill
        
        \hrulefill
        
        \hrulefill
        
        \hrulefill
        
        \hrulefill
        
        \hrulefill
        
        \hrulefill
        
        \hrulefill
        
        \hrulefill
        
        \hrulefill
        
        \hrulefill
        
        \hrulefill
        
        \hrulefill
        
        \hrulefill
        
        \hrulefill
        
        \hrulefill

    \pagebreak
        \item (2014-DSE-MATH-EP(M2) \#01) 在 $(1-4x)^2(1+x)^n$的展開式中,$x$的係數為1。\begin{enumerate}
            \item 求$n$的值。
            \item 求$x^2$的係數的值。
        \end{enumerate}\hfill(9分)
            
            \hrulefill
            
            \hrulefill
            
            \hrulefill
            
            \hrulefill
            
            \hrulefill
            
            \hrulefill
            
            \hrulefill
            
            \hrulefill
            
            \hrulefill
            
            \hrulefill
            
            \hrulefill
            
            \hrulefill
            
            \hrulefill
            
            \hrulefill
            
            \hrulefill
            
            \hrulefill
            
            \hrulefill
            
            \hrulefill
            
            \hrulefill
            
            \hrulefill
            
            \hrulefill
            
            \hrulefill

        \pagebreak
        \item (2015-DSE-MATH-EP(M2) \#07)  \begin{enumerate}
            \item 證明 $\displaystyle\sin^2{x}\cos^2{x}=\frac{1-\cos{4x}}{8}$.
            \item 設$f(x)=\sin^4{x}+\cos^4{x}$。\begin{enumerate}
                \item 試以 $A\cos{Bx}+C$的形式表示$f(x)$,其中$A,B$和$C$為常數。
                \item 解方程$8f(x)=7$,其中$0\leq x\leq \frac{\pi}{2}$。
            \end{enumerate}
        \end{enumerate}\hfill(12分)
            
            \hrulefill
            
            \hrulefill
            
            \hrulefill
            
            \hrulefill
            
            \hrulefill
            
            \hrulefill
            
            \hrulefill
            
            \hrulefill
            
            \hrulefill
            
            \hrulefill
            
            \hrulefill
            
            \hrulefill
            
            \hrulefill
            
            \hrulefill
            
            \hrulefill
            
            \hrulefill
            
            \hrulefill
            
            \hrulefill
            
            \hrulefill
            
            \hrulefill

        \pagebreak
        \item (2015-DSE-MATH-EP(M2) \#08) \begin{enumerate}
            \item 運用數學歸納法,證明對於所有正整數$n$,$$\sin{\frac{x}{2}}\sum_{k=1}^{n}\cos{kx}=\sin{\frac{nx}{2}}\cos{\frac{(n+1)x}{2}}$$
            \item 利用 (a),計算 $\displaystyle\sum_{k=1}^{567}\cos{\frac{k\pi}{7}}$.
        \end{enumerate}\hfill(13分)
            
            \hrulefill
            
            \hrulefill
            
            \hrulefill
            
            \hrulefill
            
            \hrulefill
            
            \hrulefill
            
            \hrulefill
            
            \hrulefill
            
            \hrulefill
            
            \hrulefill
            
            \hrulefill
            
            \hrulefill
            
            \hrulefill
            
            \hrulefill
            
            \hrulefill
            
            \hrulefill
            
            \hrulefill
            
            \hrulefill
            
            \hrulefill
            
            \hrulefill

        \pagebreak
        \item (2016-DSE-MATH-EP(M2) \#02)證明 $\displaystyle \frac{1}{\sqrt{x}}-\frac{1}{\sqrt{x+h}}=\frac{h}{(x+h)\sqrt{x}+x\sqrt{x+h}}$再從基本原理求$\displaystyle \dfrac{d}{dx}\sqrt{\frac{3}{x}}$。\hfill(10分)
        
        \hrulefill
            
            \hrulefill
            
            \hrulefill
            
            \hrulefill
            
            \hrulefill
            
            \hrulefill
            
            \hrulefill
            
            \hrulefill
            
            \hrulefill
            
            \hrulefill
            
            \hrulefill
            
            \hrulefill
            
            \hrulefill
            
            \hrulefill

            \hrulefill
            
            \hrulefill
            
            \hrulefill
            
            \hrulefill
            
            \hrulefill
            
            \hrulefill
            
            \hrulefill
            
            \hrulefill
            
            \hrulefill
            
            \hrulefill

        \pagebreak
        \item (1994-CE-A MATH 1 \#04) 設$\displaystyle y=\tan{\frac{1}{x}}$。 \begin{enumerate}
            \item 證明 $x^2\dfrac{dy}{dx}+(y^2+1)=0$。
            \item 證明$\displaystyle\dfrac{d^2y}{dx^2}+\frac{2(x+y)}{x^2}\dfrac{dy}{dx}=0$。
        \end{enumerate}\hfill(11分)
        
            \hrulefill
            
            \hrulefill
            
            \hrulefill
            
            \hrulefill
            
            \hrulefill
            
            \hrulefill
            
            \hrulefill
            
            \hrulefill
            
            \hrulefill
            
            \hrulefill
            
            \hrulefill

            \hrulefill
            
            \hrulefill
            
            \hrulefill
            
            \hrulefill
            
            \hrulefill
            
            \hrulefill
            
            \hrulefill
            
            \hrulefill
            
            \hrulefill
            
            \hrulefill

        \pagebreak
        \item (2004-CE-A MATH \#09(Modified)) 設$P(a,b)$為曲綫$C:y=x^3$上的一點 使得$C$在$P$的切綫穿過$(0,2)$。 \begin{enumerate}
            \item 證明$b=3a^3+2$。
            \item 求$a$和$b$的值。
        \end{enumerate}\hfill(11分)
            
            \hrulefill
            
            \hrulefill
            
            \hrulefill
            
            \hrulefill
            
            \hrulefill
            
            \hrulefill
            
            \hrulefill
            
            \hrulefill
            
            \hrulefill
            
            \hrulefill
            
            \hrulefill
            
            \hrulefill
            
            \hrulefill
            
            \hrulefill
            
            \hrulefill
            
            \hrulefill
            
            \hrulefill
            
            \hrulefill
            
            \hrulefill
            
            \hrulefill
            
            \hrulefill

        \pagebreak
        \item (1992-CE-A MATH 1 \#05) 曲綫 $(x-2)(y^2+3)=-8$ 與y軸相交於兩點。求\begin{enumerate}
            \item 相交點的坐標。
            \item 穿過該兩點的切綫方程。
        \end{enumerate}\hfill(11分)
        
        \hrulefill
            
            \hrulefill
            
            \hrulefill
            
            \hrulefill
            
            \hrulefill
            
            \hrulefill
            
            \hrulefill
            
            \hrulefill
            
            \hrulefill
            
            \hrulefill
            
            \hrulefill
            
            \hrulefill
            
            \hrulefill
            
            \hrulefill
            
            \hrulefill
            
            \hrulefill
            
            \hrulefill
            
            \hrulefill
            
            \hrulefill
            
            \hrulefill
            
            \hrulefill
            
            \hrulefill

        \pagebreak
    \end{enumerate}
\end{document}