\documentclass[12pt]{article}
\usepackage{ctex}
\usepackage[english]{babel}
\usepackage{blindtext}
\usepackage{nameref}
\usepackage{fancyhdr}
\usepackage{amsmath,amssymb,amsthm}
\usepackage{graphicx,float}
\usepackage{physics}
\usepackage{pgfplots}
\usepackage[a4paper, total={6in, 9in}]{geometry}

\pagestyle{fancy}
\fancyhf{}
\fancyhf[HL]{函數}
\fancyhf[CF]{\thepage}

\newcommand{\innerprod}[2]{\langle{#1},{#2}\rangle}
\newcommand{\id}{\mathtt{id}}

\newtheorem{definition}{定義}
\newtheorem*{theorem}{定理}
\newtheorem*{corollary}{衍理}
\newtheorem*{lemma}{引理}
\newtheorem*{proposition}{設理}
\newtheorem*{remark}{小記}
\newtheorem*{claim}{主張}
\newtheorem*{example}{例子}
\newtheorem*{axiom}{公設}
\renewenvironment*{proof}{\textit{證明.}}{\hfill$\qed$}

\newenvironment*{sol}{\par \textbf{解}.}{\hfill$\blacksquare$}

\begin{document}
    \begin{enumerate}
        \item \begin{enumerate}
            \item \begin{align*}
                P(1):\sum_{r=1}^1 r^3=\frac{1}{4}(1^2)(1+1)^2\\
                P(n)\implies P(n+1)\\
                \sum_{r=1}^{n+1}r^3&=\frac{1}{4}n^2(n+1)^2+(n+1)^3\\
                &=\frac{1}{4}(n+1)^2(n^2+4n+4)\\
                &=\frac{1}{4}(n+1)^2(n+2)^2
            \end{align*}
            \item \begin{align*}
                1^3-2^3+3^3-4^3+\cdots+(-1)^{r+1}r^3+\cdots-(2n)^3&=\sum_{r=1}^{2n}r^3-2\sum_{r=1}^n(2r)^3\\
                &=\frac{1}{4}(2n)^2(2n+1)^2-\frac{16}{4}n^2(n+1)^2\\
                &=n^2[(2n+1)^2-4(n+1)^2]\\
                &=-n^2(4n+3)
            \end{align*}
        \end{enumerate}
        \item \begin{align*}
            4n+16C_2^n&=180\\
            16n^2-8n-360&=0\\
            n&=5\\
            C_3^5&=10
        \end{align*}
        \item \begin{enumerate}
            \item \begin{align*}
                n-8&=1\\
                n&=9
            \end{align*}
            \item \begin{align*}
                16-8\cdot9+36&=-20
            \end{align*}
        \end{enumerate}
        \item \begin{enumerate}
            \item \begin{align*}
                \sin^2{x}\cos^2{x}&=\frac{1}{4}(\sin{2x})^2\\
                &=\frac{1}{4}\frac{1-\cos{4x}}{2}\\
                &=\frac{1-\cos{4x}}{8}
            \end{align*}
            \item \begin{enumerate}
                \item \begin{align*}
                    f(x)&=\cos^4{x}+\sin^4{x}\\
                    &=1-2\sin^2{x}\cos^2{x}\\
                    &=1-2\frac{1-\cos{4x}}{8}\\
                    &=\frac{3}{4}+\frac{1}{4}\cos{4x}
                \end{align*}
                \item \begin{align*}
                    8f(x)&=7\\
                    \frac{3}{4}+\frac{1}{4}\cos{4x}&=\frac{7}{8}\\
                    \cos{4x}&=\frac{1}{2}\\
                    4x&=\pi/3, -\pi/3\\
                    x&=\pi/12
                \end{align*}
            \end{enumerate}
        \end{enumerate}
        \item \begin{enumerate}
            \item \begin{align*}
                P(1):\sin{\frac{x}{2}}\cos{x}\\
                P(n)\implies P(n+1):\\
                \sin{\frac{x}{2}}\sum_{k=1}^{n+1}\cos{kx}&=\sin{\frac{nx}{2}}\cos{\frac{(n+1)x}{2}}+\sin{\frac{x}{2}}\cos[(n+1)x]\\
                &=\frac{1}{2}[\sin(nx+\frac{x}{2})-\sin(\frac{x}{2})+\sin(nx+\frac{3x}{2})-\sin{nx+\frac{x}{2}}]\\
                &=\sin{\frac{x}{2}}\cos{\frac{(n+2)x}{2}}
            \end{align*}
            \item \begin{align*}
                \sin{\frac{\pi}{14}}\sum_{k=1}^{567}\cos{\frac{k\pi}{7}}&=\sin{\frac{567\pi}{14}}\cos{\frac{568\pi}{14}}\\
                &=\cos{\frac{4\pi}{7}}\\&=-\sin{\frac{\pi}{14}}
                \sum_{k=1}^{567}\cos{\frac{k\pi}{7}}&=-1
            \end{align*}
        \end{enumerate}
        \item \begin{align*}
            \frac{1}{\sqrt{x}}-\frac{1}{\sqrt{x+h}}&=\frac{\sqrt{x+h}-\sqrt{x}}{\sqrt{x+h}\sqrt{x}}\\
            &=\frac{h}{\sqrt{x+h}\sqrt{x}(\sqrt{x+h}+\sqrt{x})}\\
            &=\frac{h}{(x+h)\sqrt{x}+x\sqrt{x+h}}\\
            \dfrac{dy}{dx}\sqrt{\frac{3}{x}}&=\lim_{h\to 0}\frac{\sqrt{3/(x+h)}-\sqrt{3/x}}{h}\\
            &=\sqrt{3}\lim_{h\to 0}\frac{1}{h}\frac{h}{(x+h)\sqrt{x}+x\sqrt{x+h}}\\
            &=\sqrt{3}\lim_{h\to 0}\frac{1}{(x+h)\sqrt{x}+x\sqrt{x+h}}\\
            &=\frac{\sqrt{3}}{2x^{3/2}}
        \end{align*}
        \item \begin{enumerate}
            \item \begin{align*}
                \dfrac{dy}{dx}&=-\frac{1}{x^2}\sec^2{\frac{1}{x}}\\
                x^2\dfrac{dy}{dx}&=-(1+\tan^2{\frac{1}{x}})\\
                x^2\dfarc{dy}{dx}+(y^2+1)&=0
            \end{align*}
            \item \begin{align*}
                x^2\dfrac{d^2y}{dx^2}+2x\dfrac{dy}{dx}+2y\dfrac{dy}{dx}&=0\\
                \dfrac{d^2y}{dx^2}+\frac{2(x+y)}{x^2}\dfrac{dy}{dx}&=0
            \end{align*}
        \end{enumerate}
        \item \begin{enumerate}
            \item \begin{align*}
                y'&=3x^2\\
                y-b&=3a^2(x-a)\\
                2-b&=3a^2(-a)\\
                b&=3a^3+2
            \end{align*}
            \item \begin{align*}
                a^3&=3a^3+2\\
                a&=-1\\
                b&=-1
            \end{align*}
        \end{enumerate}
        \item \begin{enumerate}
            \item \begin{align*}
                y^2+3&=4\\
                y^2&=1\\
                y&=\pm 1
            \end{align*}
            So the points are $\{(0,-1),(0,1)\}$.
            \item \begin{align*}
                (y^2+3)+2y\dfrac{dy}{dx}(x-2)&=0\\
                \dfrac{dy}{dx}|_\{(0,\pm 1)\}&=\pm 1
            \end{align*}
        \end{enumerate}
    \end{enumerate}
\end{document}