\documentclass[12pt]{article}
\usepackage{ctex}
\usepackage[english]{babel}
\usepackage{blindtext}
\usepackage{nameref}
\usepackage{fancyhdr}
\usepackage{amsmath,amssymb,amsthm}
\usepackage{graphicx,float}
\usepackage{physics}
\usepackage{pgfplots}
\usepackage[a4paper, total={6in, 9in}]{geometry}

\pagestyle{fancy}
\fancyhf{}
\fancyhf[HL]{F4 Level 4 M2 mock paper}
\fancyhf[HR]{Time limit: 1 hr 30 mins}
\fancyhf[CF]{\thepage}

\newcommand{\innerprod}[2]{\langle{#1},{#2}\rangle}
\newcommand{\id}{\mathtt{id}}

\newtheorem*{definition}{Definition}
\newtheorem*{theorem}{Theorem}
\newtheorem*{corollary}{Corollary}
\newtheorem*{lemma}{Lemma}
\newtheorem*{proposition}{Proposition}
\newtheorem*{remark}{Remark}
\newtheorem*{claim}{Claim}
\newtheorem*{example}{Example}
\newtheorem*{axiom}{Axiom}

\begin{document}
    \thispagestyle{plain}

    \centering 

    \section*{PRACTICE PAPER\\MATHEMTICS Extended Part\\Module 2 (Algebra and Calculus)\\Question-Answer Book}

    Time allowed: 1.5 hours

    Name:\hrulefill \hfill Marks:\hrulefill/100

    School:\hrulefill

    \raggedright

    \subsection*{Instructions}

    \begin{enumerate}
        \item This paper must be answered in English.
        \item Unless otherwise specified, all working must be clearly shown.
        \item Unless otherwise specified, numerical answers must be exact.
        \item This paper is for \textbf{internal use} only.
        \item All questions are collected from AL/CE/DSE past papers, reference site: https://www.dse.life/ppindex/m2/
    \end{enumerate}

    \newpage
    \begin{enumerate}
        \item (1990-HL-GEN MATHS \#05)  \begin{enumerate}
            \item Prove by mathematical induction that for any positive integer $n$, $$\sum_{r=1}^{n}r^3=\frac{1}{4}n^2(n+1)^2$$
            \item Find $1^3-2^3+3^3-4^3+\dots+(-1)^{r+1}r^3+\dots-(2n)^3$.
        \end{enumerate}\hfill(13 marks)
        
            \hrulefill
            
            \hrulefill
            
            \hrulefill
            
            \hrulefill
            
            \hrulefill
            
            \hrulefill
            
            \hrulefill
            
            \hrulefill
            
            \hrulefill
            
            \hrulefill
            
            \hrulefill
            
            \hrulefill
            
            \hrulefill
            
            \hrulefill
            
            \hrulefill
            
            \hrulefill
            
            \hrulefill
            
            \hrulefill
            
            \hrulefill
            
            \hrulefill
            
            \hrulefill

        \pagebreak
        \item (2010-CE-A MATH \#05) The sum of the coefficients of $x$ and $x^2$ in the expansion of $(1+4x)^n$ is 180, where $n$ is a positive integer. Find the value of $n$ and the coefficient of $x^3$.\hfill(10 marks)
        
        \hrulefill
            
        \hrulefill
        
        \hrulefill
        
        \hrulefill
        
        \hrulefill
        
        \hrulefill
        
        \hrulefill
        
        \hrulefill
        
        \hrulefill
        
        \hrulefill
        
        \hrulefill
        
        \hrulefill
        
        \hrulefill
        
        \hrulefill
        
        \hrulefill
        
        \hrulefill
        
        \hrulefill
        
        \hrulefill
        
        \hrulefill
        
        \hrulefill
        
        \hrulefill
        
        \hrulefill
        
        \hrulefill
        
        \hrulefill
        
        \hrulefill
        
        \hrulefill

    \pagebreak
        \item (2014-DSE-MATH-EP(M2) \#01) In the expansion of $(1-4x)^2(1+x)^n$, the coefficient of $x$ is 1.\begin{enumerate}
            \item Find the value of $n$.
            \item Find the coefficient of $x^2$.
        \end{enumerate}\hfill(9 marks)
            
            \hrulefill
            
            \hrulefill
            
            \hrulefill
            
            \hrulefill
            
            \hrulefill
            
            \hrulefill
            
            \hrulefill
            
            \hrulefill
            
            \hrulefill
            
            \hrulefill
            
            \hrulefill
            
            \hrulefill
            
            \hrulefill
            
            \hrulefill
            
            \hrulefill
            
            \hrulefill
            
            \hrulefill
            
            \hrulefill
            
            \hrulefill
            
            \hrulefill
            
            \hrulefill
            
            \hrulefill

        \pagebreak
        \item (2015-DSE-MATH-EP(M2) \#07)  \begin{enumerate}
            \item Prove that $\displaystyle\sin^2{x}\cos^2{x}=\frac{1-\cos{4x}}{8}$.
            \item Let $f(x)=\sin^4{x}+\cos^4{x}$.\begin{enumerate}
                \item Express $f(x)$ in the form of $A\cos{Bx}+C$, where $A,B$ and $C$ are constants.
                \item Solve the equation $8f(x)=7$, where $0\leq x\leq \frac{\pi}{2}$.
            \end{enumerate}
        \end{enumerate}\hfill(12 marks)
            
            \hrulefill
            
            \hrulefill
            
            \hrulefill
            
            \hrulefill
            
            \hrulefill
            
            \hrulefill
            
            \hrulefill
            
            \hrulefill
            
            \hrulefill
            
            \hrulefill
            
            \hrulefill
            
            \hrulefill
            
            \hrulefill
            
            \hrulefill
            
            \hrulefill
            
            \hrulefill
            
            \hrulefill
            
            \hrulefill
            
            \hrulefill
            
            \hrulefill

        \pagebreak
        \item (2015-DSE-MATH-EP(M2) \#08) \begin{enumerate}
            \item Using mathematical induction, prove that $$\sin{\frac{x}{2}}\sum_{k=1}^{n}\cos{kx}=\sin{\frac{nx}{2}}\cos{\frac{(n+1)x}{2}}$$ for all positive integers $n$.
            \item Using (a), evaluate $\displaystyle\sum_{k=1}^{567}\cos{\frac{k\pi}{7}}$.
        \end{enumerate}\hfill(13 marks)
            
            \hrulefill
            
            \hrulefill
            
            \hrulefill
            
            \hrulefill
            
            \hrulefill
            
            \hrulefill
            
            \hrulefill
            
            \hrulefill
            
            \hrulefill
            
            \hrulefill
            
            \hrulefill
            
            \hrulefill
            
            \hrulefill
            
            \hrulefill
            
            \hrulefill
            
            \hrulefill
            
            \hrulefill
            
            \hrulefill
            
            \hrulefill
            
            \hrulefill

        \pagebreak
        \item (2016-DSE-MATH-EP(M2) \#02)Prove that $\displaystyle \frac{1}{\sqrt{x}}-\frac{1}{\sqrt{x+h}}=\frac{h}{(x+h)\sqrt{x}+x\sqrt{x+h}}$. Hence, find $\displaystyle \dfrac{d}{dx}\sqrt{\frac{3}{x}}$ from first principles.\hfill(10 marks)
        
        \hrulefill
            
            \hrulefill
            
            \hrulefill
            
            \hrulefill
            
            \hrulefill
            
            \hrulefill
            
            \hrulefill
            
            \hrulefill
            
            \hrulefill
            
            \hrulefill
            
            \hrulefill
            
            \hrulefill
            
            \hrulefill
            
            \hrulefill

            \hrulefill
            
            \hrulefill
            
            \hrulefill
            
            \hrulefill
            
            \hrulefill
            
            \hrulefill
            
            \hrulefill
            
            \hrulefill
            
            \hrulefill
            
            \hrulefill

        \pagebreak
        \item (1994-CE-A MATH 1 \#04) Let $\displaystyle y=\tan{\frac{1}{x}}$. \begin{enumerate}
            \item Show that $x^2\dfrac{dy}{dx}+(y^2+1)=0$.
            \item Hence show that $\displaystyle\dfrac{d^2y}{dx^2}+\frac{2(x+y)}{x^2}\dfrac{dy}{dx}=0$.
        \end{enumerate}\hfill(11 marks)
        
            \hrulefill
            
            \hrulefill
            
            \hrulefill
            
            \hrulefill
            
            \hrulefill
            
            \hrulefill
            
            \hrulefill
            
            \hrulefill
            
            \hrulefill
            
            \hrulefill
            
            \hrulefill

            \hrulefill
            
            \hrulefill
            
            \hrulefill
            
            \hrulefill
            
            \hrulefill
            
            \hrulefill
            
            \hrulefill
            
            \hrulefill
            
            \hrulefill
            
            \hrulefill

        \pagebreak
        \item (2004-CE-A MATH \#09(Modified)) Let $P(a,b)$ be a point on the curve $C:y=x^3$ such that the tangent to $C$ at $P$ passes through the point $(0,2)$. \begin{enumerate}
            \item Show that $b=3a^3+2$.
            \item Find the value of $a$ and $b$.
        \end{enumerate}\hfill(11 marks)
            
            \hrulefill
            
            \hrulefill
            
            \hrulefill
            
            \hrulefill
            
            \hrulefill
            
            \hrulefill
            
            \hrulefill
            
            \hrulefill
            
            \hrulefill
            
            \hrulefill
            
            \hrulefill
            
            \hrulefill
            
            \hrulefill
            
            \hrulefill
            
            \hrulefill
            
            \hrulefill
            
            \hrulefill
            
            \hrulefill
            
            \hrulefill
            
            \hrulefill
            
            \hrulefill

        \pagebreak
        \item (1992-CE-A MATH 1 \#05) The curve $(x-2)(y^2+3)=-8$ cuts the y-axis at two points. Find\begin{enumerate}
            \item the coordinates of the two points.
            \item the slope of the tangent to the curve at each of the two points.
        \end{enumerate}\hfill(11 marks)
        
        \hrulefill
            
            \hrulefill
            
            \hrulefill
            
            \hrulefill
            
            \hrulefill
            
            \hrulefill
            
            \hrulefill
            
            \hrulefill
            
            \hrulefill
            
            \hrulefill
            
            \hrulefill
            
            \hrulefill
            
            \hrulefill
            
            \hrulefill
            
            \hrulefill
            
            \hrulefill
            
            \hrulefill
            
            \hrulefill
            
            \hrulefill
            
            \hrulefill
            
            \hrulefill
            
            \hrulefill

        \pagebreak
    \end{enumerate}
\end{document}