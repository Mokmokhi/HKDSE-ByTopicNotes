\documentclass[12pt]{article}
\usepackage{ctex}
\usepackage[english]{babel}
\usepackage{blindtext}
\usepackage{nameref}
\usepackage{fancyhdr}
\usepackage{amsmath,amssymb,amsthm}
\usepackage{graphicx,float}
\usepackage{physics}
\usepackage{pgfplots}
\usepackage[a4paper, total={6in, 9in}]{geometry}

\pagestyle{fancy}
\fancyhf{}
\fancyhf[HL]{The number $e$}
\fancyhf[CF]{\thepage}

\newcommand{\innerprod}[2]{\langle{#1},{#2}\rangle}
\newcommand{\id}{\mathtt{id}}

\newtheorem*{definition}{Definition}
\newtheorem*{theorem}{Theorem}
\newtheorem*{corollary}{Corollary}
\newtheorem*{lemma}{Lemma}
\newtheorem*{proposition}{Proposition}
\newtheorem*{remark}{Remark}
\newtheorem*{claim}{Claim}
\newtheorem*{example}{Example}
\newtheorem*{axiom}{Axiom}

\begin{document}
    \section{The normal tuition of the Euler number $e$ and natural logarithmic function $\ln x$}
    Following our usual tuition system, we introduce the number $e$ as the definition $$e=\lim_{n\to \infty}(1+\frac{1}{n})^n=\sum_{n=0}^\infty \frac{1}{n!}$$ and define the functions \begin{align*}
        e^x&=\lim_{n\to \infty}(1+\frac{x}{n})^n=\sum_{n=0}^\infty \frac{x^n}{n!}\\
        \ln{x}&=u \iff x=e^u
    \end{align*}which is, as `defined' by some natural force or by some certainty. It is also natural to write $$e^{ax}=\lim_{n\to \infty}(1+\frac{ax}{n})^n=\lim_{n\to \infty}(1+\frac{x}{n})^{an}$$ which is just like some transposition of variables. But one must ask: Why is this number defined in this fashion? Does the limit really exist? Is it unique? And the most important thing, why is the log function called `natural'?

    To answer these questions, we must use some higher level techniques, and I would like to inherit the content of the book `Elementary Mathematics from a Higher Standpoint' written by F. Klein, so that the insight become more fruitful.

    \section{Existence and uniqueness of $e$}
    We first tackle the problem of its existence and uniqueness, so that we have interest in discussing the meaning of the number $e$. This time we could follow the definition on limit.

    

    \begin{theorem}
        The infinite sum $\sum_{n=0}^\infty \frac{1}{n!}$ converges.
    \end{theorem}
    \begin{proof}
        By ratio test, we have \[\frac{1}{n+1}\cdot \frac{n}{1}=\]
    \end{proof}

    \begin{theorem}
        The limit $\lim_{n\to \infty}(1+\frac{1}{n})^n$ exists, and is unique.
    \end{theorem}
    \begin{proof}
        To prove the limit exists, we consider its boundedness and monotonicity. 
    \end{proof}

    \section{The Origination: Natural growth rate yields the natural estimation}

    \section{The art of abstraction: Summation form}
\end{document}