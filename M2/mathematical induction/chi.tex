\documentclass[12pt]{article}
\usepackage{ctex}
\usepackage[english]{babel}
\usepackage{blindtext}
\usepackage{nameref}
\usepackage{fancyhdr}
\usepackage{amsmath,amssymb,amsthm}
\usepackage{graphicx,float}
\usepackage{physics}
\usepackage{pgfplots}
\usepackage[a4paper, total={6in, 9in}]{geometry}

\pagestyle{fancy}
\fancyhf{}
\fancyhf[HL]{數學歸納法}
\fancyhf[CF]{\thepage}

\newcommand{\innerprod}[2]{\langle{#1},{#2}\rangle}
\newcommand{\id}{\mathtt{id}}

\newtheorem{definition}{定義}
\newtheorem*{theorem}{定理}
\newtheorem*{corollary}{衍理}
\newtheorem*{lemma}{引理}
\newtheorem*{proposition}{設理}
\newtheorem*{remark}{小記}
\newtheorem*{claim}{主張}
\newtheorem*{example}{例子}
\newtheorem*{axiom}{公設}
\renewenvironment*{proof}{\textit{證明.}}{\hfill$\qed$}

\newenvironment*{sol}{\par \textbf{解}.}{\hfill$\blacksquare$}

\begin{document}
    \section*{求和記法}

    求和記法的作用:避免混淆

    \begin{definition}[求和號]
        設$a_1,a_2,\dots,a_n$是一組數,他們的\textbf{和}記爲$\displaystyle \sum_{i=1}^{n}a_i$,即$$\sum_{i=1}^{n}a_i=a_1+a_2+\cdots+a_n.$$
    \end{definition}

    $\displaystyle \sum_{i=1}^{n}a_i$也可寫作$\displaystyle \sum_{1\leq i\leq n} a_i$ 或 $\displaystyle\sum_{i\in S}a_i$,其中$S=\{1,2,\dots,n\}$;此爲\textbf{條件式求和},即將符合條件的項加總。

    當$i$的範圍不言自明(或無須定義)時,還可簡記為$\displaystyle\sum_{i}a_i$或$\sum_{} a_i$。

    \begin{remark}[啞變量]
        求和變量$i$也可用$j,k$等其他代數表示,就像函數的自變量除了用$x$表示也可用$u,t$等代數表示一樣。因此,$$a_1+a_2+\cdots+a_n=\sum_{i=1}^{n}a_i=\sum_{j=1}^{n}a_j=\sum_{s=1}^{n}a_s=\sum_{X=1}^{n}a_X=\cdots$$
    \end{remark}

    \begin{definition}[雙重求和號]
        若$a_{ij}$是同時由$i,j$定義的一組數,即若
        \begin{center}
            \begin{tabular}{c c c c}
                $a_{11}$&$a_{12}$&$a_{13}$&$\cdots$\\
                $a_{21}$&$a_{22}$&$a_{23}$&$\cdots$\\
                $a_{31}$&$a_{32}$&$a_{33}$&$\cdots$\\
                $\vdots$&$\vdots$&$\vdots$&$\ddots$
            \end{tabular}
        \end{center}
        爲一陣列數,則$$\sum_{i=1}^{m}\sum_{j=1}^{n}a_{ij}=\sum_{i=1}^{m}A_i$$其中$\displaystyle A_i=\sum_{j}^{n}a_{ij}$。
    \end{definition}

    雙重求和的順序由内而外,表示先對$j$進行求和,再對$i$進行求和。

    \begin{proposition}[求和號的性質]
        求和號有以下性質:\begin{enumerate}
            \item $\displaystyle\sum_{i=1}^{n}a_i+\sum_{i=1}^{n}b_i=\sum_{i=1}^{n}(a_i+b_i)$;
            \item $\displaystyle c\sum_{i=1}^{n}a_i=\sum_{i=1}^{n}ca_i$;
            \item $\displaystyle (\sum_{i=1}^{n}a_i)(\sum_{j=1}^{m}b_j)=\sum_{i=1}^{n}\sum_{j=1}^{m}a_ib_j$;
            \item $\displaystyle \sum_{i=1}^{n}\sum_{j=1}^{m}a_ib_j=\sum_{j=1}^{m}\sum_{i=1}^{n}a_ib_j$。
        \end{enumerate}
    \end{proposition}

    \begin{proof}
        留作習題,利用數學歸納法證明。
    \end{proof}

    \section*{求積記法}

    \begin{definition}[求積號]
        設$a_1,a_2,\dots,a_n$是一組數,他們的\textbf{積}記爲$\displaystyle \prod_{i=1}^{n}a_i$,即$$\prod_{i=1}^{n}a_i=a_1a_2\cdots a_n.$$
    \end{definition}

    與求和號相同,$\displaystyle \prod_{i=1}^{n}a_i$也可寫作$\displaystyle \prod_{1\leq i\leq n} a_i$ 或 $\displaystyle\prod_{i\in S}a_i$,其中$S=\{1,2,\dots,n\}$;此爲\textbf{條件式求積},即將符合條件的項乘積。

    當$i$的範圍不言自明(或無須定義)時,還可簡記為$\displaystyle\prod_{i}a_i$或$\prod_{} a_i$。

    \begin{definition}[雙重求積號]
        若$a_{ij}$是同時由$i,j$定義的一組數,即若
        \begin{center}
            \begin{tabular}{c c c c}
                $a_{11}$&$a_{12}$&$a_{13}$&$\cdots$\\
                $a_{21}$&$a_{22}$&$a_{23}$&$\cdots$\\
                $a_{31}$&$a_{32}$&$a_{33}$&$\cdots$\\
                $\vdots$&$\vdots$&$\vdots$&$\ddots$
            \end{tabular}
        \end{center}
        爲一陣列數,則$$\prod_{i=1}^{m}\prod_{j=1}^{n}a_{ij}=\prod_{i=1}^{m}A_i$$其中$\displaystyle A_i=\prod_{j=1}^{n}a_{ij}$。
    \end{definition}

    \begin{proposition}[求積號的性質]
        求積號有以下性質:\begin{enumerate}
            \item $\displaystyle c^n\prod_{i=1}^{n}a_i=\prod_{i=1}^{n}ca_i$;
            \item $\displaystyle (\prod_{i=1}^{n}a_i)(\prod_{j=1}^{m}b_j)=\prod_{i=1}^{n+m}c_i$,其中$c_i=\begin{cases}
                a_i&i\leq n\\
                b_{i-n}&i>n 
            \end{cases}$;
            \item $\displaystyle \prod_{i=1}^{n}\prod_{j=1}^{m}a_ib_j=\prod_{j=1}^{m}\prod_{i=1}^{n}a_ib_j$。
        \end{enumerate}
    \end{proposition}

    \begin{proof}
        留作習題,利用數學歸納法證明。
    \end{proof}

    \section*{數學歸納法原理}

    \begin{definition}[數學陳述]
        \textbf{數學陳述}是一種有關數學,並且能分辨對錯的句子。
    \end{definition}

    \begin{example}
        $1=2$是一種錯誤的數學陳述。
    \end{example}

    \begin{example}
        $1+2=3$是一種正確的數學陳述。
    \end{example}

    \begin{definition}[判斷式]
        \textbf{判斷式}是一種牽涉變量,有待判斷對錯的數學句子。若闡明變量的值,則判斷式可有特定結果,並成爲數學陳述。
    \end{definition}

    \begin{example}
        $P(x)$為包含變量$x$的判斷式。
    \end{example}
    
    \begin{example}
        $P(x,y)$為包含變量$x,y$的判斷式。
    \end{example}

    \begin{axiom}[數學歸納法基本原理]
        設$P(n)$為對於整數$n\geq 1$的判斷式。若以下條件同時成立:\begin{itemize}
            \item $P(1)$成立;
            \item 若$P(n)$對某$n\geq 1$成立,則$P(n+1)$成立。
        \end{itemize}
        則$P(n)$對所有整數$n\geq 1$成立。
    \end{axiom}

    \begin{example}
        設$P(n)$代表對假設$\displaystyle\sum_{i=1}^{n}i=\frac{n(n+1)}{2}$的值。

        $P(1):\sum_{i=1}^{1}i=1=\frac{1(1+1)}{2}$成立。

        假定$P(k)$對某正整數$k$成立,則對於$P(k+1):$

        $$\sum_{i=1}^{k+1}i=(k+1)+\sum_{i=1}^{k}i=(k+1)+\frac{k(k+1)}{2}=(k+1)(1+\frac{k}{2})=\frac{(k+1)(k+2)}{2}$$

        $\therefore$根據基本原理,$P(n)$對於所有正整數$n$成立。
    \end{example}

    若歸納起點並非$n=1$,則可拓展至以下原理:

    \begin{axiom}[數學歸納法第二原理]
        設$P(n)$為對於整數$n\geq n_0$的判斷式。若以下條件同時成立:\begin{itemize}
            \item 基:$P(n_0)$成立;
            \item 渡:若$P(n)$對某$n\geq n_0$成立,則$P(n+1)$成立。
        \end{itemize}
        則$P(n)$對所有整數$n\geq n_0$成立。
    \end{axiom}

    \begin{corollary}[強歸納法原理]
        設$P(n)$為對於整數$n\geq 1$的判斷式。若以下條件同時成立:\begin{itemize}
            \item 基:$P(1)$成立;
            \item 渡:若$P(k)$對某$1\leq k\leq n$成立,則$P(n+1)$成立。
        \end{itemize}
        則$P(n)$對所有整數$n\geq n_0$成立。
    \end{corollary}

    \begin{remark}
        強歸納法原理等價於數學歸納法基本原理。
    \end{remark}

    \section*{習題}

    \begin{enumerate}
        \item 證明求和號的性質:\begin{enumerate}
            \item $\displaystyle\sum_{i=1}^{n}a_i+\sum_{i=1}^{n}b_i=\sum_{i=1}^{n}(a_i+b_i)$;
            \item $\displaystyle c\sum_{i=1}^{n}a_i=\sum_{i=1}^{n}ca_i$;
            \item $\displaystyle (\sum_{i=1}^{n}a_i)(\sum_{j=1}^{m}b_j)=\sum_{i=1}^{n}\sum_{j=1}^{m}a_ib_j$;
            \item $\displaystyle \sum_{i=1}^{n}\sum_{j=1}^{m}a_ib_j=\sum_{j=1}^{m}\sum_{i=1}^{n}a_ib_j$。
        \end{enumerate}
        \item 證明求積號的性質:\begin{enumerate}
            \item $\displaystyle c^n\prod_{i=1}^{n}a_i=\prod_{i=1}^{n}ca_i$;
            \item $\displaystyle (\prod_{i=1}^{n}a_i)(\prod_{j=1}^{m}b_j)=\prod_{i=1}^{n+m}c_i$,其中$c_i=\begin{cases}
                a_i&i\leq n\\
                b_{i-n}&i>n 
            \end{cases}$;
            \item $\displaystyle \prod_{i=1}^{n}\prod_{j=1}^{m}a_ib_j=\prod_{j=1}^{m}\prod_{i=1}^{n}a_ib_j$。
        \end{enumerate}
        \item 證明以下等式:\begin{enumerate}
            \item $\displaystyle(\sum_{i=1}^{m}a_ic^i)(\sum_{i}^{n}b_ic^i)=\sum_{k=0}^{m+n}(\sum_{i+j=k}a_ib_j)c^k$;
            \item $\displaystyle a^n-b^n=(a-b)\sum_{i=0}^{n-1}a^{n-1-i}b^i$;
            \item $\displaystyle (\sum_{i=0}^{n-1}a_i)(\sum_{i=0}^{n+1}a_i)=(\sum_{i=0}^{n}a_i)^2-a^n$;
            \item $\displaystyle (a_1+a_2+\cdots+a_m)^n=\sum_{i_1+i_2+\cdots+i_m=n}\frac{n!}{i_1!i_2!\cdots i_m!}a_1^{i_1}a_2^{i_2}\cdots a_m^{i_m}$。
        \end{enumerate}
        \item \begin{enumerate}
            \item 證明對於所有正整數$n$,$\displaystyle \sum_{i=1}^{n}i^2=\frac{n(n+1)(2n+1)}{6}$;
            \item 由此,求\begin{enumerate}
                \item $101^2+102^2+103^2+\cdots+200^2$;
                \item $20^2+22^2+24^2+\cdots+40^2$;
                \item $3^2+5^2+7^2+\cdots+31^2$。
            \end{enumerate}
        \end{enumerate}
        \item \begin{enumerate}
            \item 證明對於所有正整數$n$,$\displaystyle \sum_{i=0}^{n}3^i=\frac{3^n-1}{2}$;
            \item 證明對於所有正整數$n$,$\displaystyle \sum_{i=1}^{n}i3^i=\frac{3^{n+1}(2n-1)+3}{4}$;
            \item 由此,求$1\times 1+2\times 3+3\times 3^2+4\times 3^3+\cdots +(n+1)\times 3^n$的公式。
        \end{enumerate}
        \item 證明對於所有正整數$n$,$\displaystyle \frac{1}{2\cdot 5}+\frac{1}{5\cdot 8}+\frac{1}{8\cdot 11}+\cdots+\frac{1}{(6n-1)(6n+2)}=\frac{n}{6n+2}$。
        \item 證明對於任意正整數$n$,$n!\leq n^n$。
        \item 證明對於任意正整數$n$及$-1<r<1$,$$\sum_{k=0}^\infty C_k^{n+k-1}r^k=\frac{1}{(1-r)^n}$$
    \end{enumerate}
    
\end{document}