\documentclass[12pt]{article}
\usepackage{ctex}
\usepackage[english]{babel}
\usepackage{blindtext}
\usepackage{nameref}
\usepackage{fancyhdr}
\usepackage{amsmath,amssymb,amsthm}
\usepackage{graphicx,float}
\usepackage{physics}
\usepackage{pgfplots}
\usepackage[a4paper, total={6in, 9in}]{geometry}

\graphicspath{{../image/}}

\pagestyle{fancy}
\fancyhf{}
\fancyhf[HL]{多元極限}
\fancyhf[CF]{\thepage}

\newcommand{\innerprod}[2]{\langle{#1},{#2}\rangle}
\newcommand{\id}{\mathtt{id}}

\newtheorem{definition}{定義}
\newtheorem*{theorem}{定理}
\newtheorem*{corollary}{衍理}
\newtheorem*{lemma}{引理}
\newtheorem*{proposition}{設理}
\newtheorem*{remark}{小記}
\newtheorem*{claim}{主張}
\newtheorem*{example}{例子}
\newtheorem*{axiom}{公設}
\renewenvironment*{proof}{\textit{證明.}}{\hfill$\qed$}

\newenvironment*{sol}{\par \textbf{解}.}{\hfill$\blacksquare$}

\begin{document}
    \section*{二元極限}

    對於多元極限,我們可從二元極限出發:$$\lim_{x\to 0, y\to 0}f(x,y)$$

    為符合極限的唯一性,我們需作$(x,y)\mapsto(r,\theta)$的置換。參考圓形,可得$$\begin{cases}
        x=r\cos{\theta}\\y=r\sin{\theta}
    \end{cases}$$

    則$\displaystyle \lim_{x\to 0, y\to 0}f(x,y)$存在當且僅當$\displaystyle\lim_{r\to 0}f(r\cos{\theta},r\sin{\theta})$存在。

    \begin{theorem}
        若$\displaystyle\lim_{r\to 0}f(r\cos{\theta},r\sin{\theta})$存在,則$$\lim_{x\to 0, y\to 0}f(x,y)=\lim_{x\to 0}\lim_{y\to 0}f(x,y)=\lim_{y\to 0}\lim_{x\to 0}f(x,y)$$
    \end{theorem}

    \begin{proof}
        若$\displaystyle\lim_{r\to 0}f(r\cos{\theta},r\sin{\theta})$存在,則極限受$r$控制,並符合極限唯一性。故從任何方向推演極限必然得到相同結果。故以任何方式求得極限,結果必然相同。
    \end{proof}

    \section*{多元極限}

    面對更高維度的極限時,可作二元分立。設$\mathbf{x}=(x_1,x_2,\dots,x_n)\in\mathbb{R}^n$,$y\in\mathbb{R}$,對於$$\lim_{\mathbf{x}\to 0,y\to 0}f(\mathbf{x},y)$$

    作$(\mathbf{x},y)\mapsto(r,\theta)$的置換,可得如二元極限的變化:$$\begin{cases}
        r^2=\norm*{\mathbf{x}}^2+y^2\\
        \norm*{\mathbf{x}}=r\cos{\theta}\\
        y=r\sin{\theta}
    \end{cases}$$
    其中$\norm*{\mathbf{x}}:=\sqrt{x_1^2+x_2^2+\cdots+x_n^2}$。重複步驟可得多元極限結果。
\end{document}