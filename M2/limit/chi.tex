\documentclass[12pt]{article}
\usepackage{ctex}
\usepackage[english]{babel}
\usepackage{blindtext}
\usepackage{nameref}
\usepackage{fancyhdr}
\usepackage{amsmath,amssymb,amsthm}
\usepackage{graphicx,float}
\usepackage{physics}
\usepackage{pgfplots}
\usepackage[a4paper, total={6in, 9in}]{geometry}

\graphicspath{{../image/}}

\pagestyle{fancy}
\fancyhf{}
\fancyhf[HL]{三角函數}
\fancyhf[CF]{\thepage}

\newcommand{\innerprod}[2]{\langle{#1},{#2}\rangle}
\newcommand{\id}{\mathtt{id}}

\newtheorem{definition}{定義}
\newtheorem*{theorem}{定理}
\newtheorem*{corollary}{衍理}
\newtheorem*{lemma}{引理}
\newtheorem*{proposition}{設理}
\newtheorem*{remark}{小記}
\newtheorem*{claim}{主張}
\newtheorem*{example}{例子}
\newtheorem*{axiom}{公設}
\renewenvironment*{proof}{\textit{證明.}}{\hfill$\qed$}

\newenvironment*{sol}{\par \textbf{解}.}{\hfill$\blacksquare$}

\begin{document}
    \section*{無窮小量與無窮大量}
    
    在高等數學,對於無窮的討論,一般從無窮小量開始。何爲無窮小量?即一個非常接近0的變量不斷向零靠近,而永遠無法到達0,即爲無窮小量。

    我們可以考慮數列$\{a_n\}$, 其中對於任意整數$n$,$a_n=\dfrac{1}{10^n}$。則當$n$越大時,$a_n$越靠近0。對此,記$$a_n\to 0$$考慮對任意$n$,均有$\varepsilon>0$使得$0<\varepsilon<a_n$,則稱變量$\varepsilon$為無窮小量。記$\varepsilon\to 0$。

    相對的,考慮數列$\{A_n\}$, 其中對於任意整數$n$,$A_n=10^n$。則當$n$越大時,$A_n$越靠近$\infty$。對此,記$$A_n\to \infty$$考慮對任意$n$,均有$N>0$使得$A_n<N$,則稱變量$N$為無窮大量。記$N\to \infty$。
    \section*{極限的幾何概念}
    \section*{$\varepsilon-\delta$定義-於無窮小的極限}
    \section*{極限的性質}
    \section*{特殊的極限}
    \section*{於無窮大的極限}
    \section*{連續函數}
    \section*{連續函數的性質}
    \section*{介值定理}
    \section*{單調函數與逆函數}
\end{document}