\documentclass[12pt]{article}
\usepackage{ctex}
\usepackage[english]{babel}
\usepackage{blindtext}
\usepackage{nameref}
\usepackage{fancyhdr}
\usepackage{amsmath,amssymb,amsthm}
\usepackage{graphicx,float}
\usepackage{physics}
\usepackage{pgfplots}
\usepackage[a4paper, total={6in, 9in}]{geometry}

\pagestyle{fancy}
\fancyhf{}
\fancyhf[HL]{M2 測驗}
\fancyhf[CF]{\thepage}

\newcommand{\innerprod}[2]{\langle{#1},{#2}\rangle}
\newcommand{\id}{\mathtt{id}}

\newtheorem*{definition}{Definition}
\newtheorem*{theorem}{Theorem}
\newtheorem*{corollary}{Corollary}
\newtheorem*{lemma}{Lemma}
\newtheorem*{proposition}{Proposition}
\newtheorem*{remark}{Remark}
\newtheorem*{claim}{Claim}
\newtheorem*{example}{Example}
\newtheorem*{axiom}{Axiom}


\begin{document}
    全卷60分,限時24小時。請將答案以pdf形式提交。
    \section*{函數基礎}

    \begin{enumerate}
        \item (10分)設$f:\mathbb{R}\to\mathbb{R}$為實函數。\begin{enumerate}
            \item (4分)證明$\frac{f(x)+f(-x)}{2}$為偶函數;$\frac{f(x)-f(-x)}{2}$為奇函數。
            \item (2分)證明對於任意實函數,均可拆分為奇函數及偶函數兩部分。
            \item 現假設對於任意$x,y\in\mathbb{R}$,函數$f$都符合以下函數等式$$f(x+y)=f(x)f(y)$$\begin{enumerate}
                \item (3分)證明對於任意$x\in\mathbb{R}$, $f(x)\geq 0$;
                \item (1分)由此,證明$f$的偶函數部分必定大於或等於$0$。
            \end{enumerate}
        \end{enumerate}
    \end{enumerate}

    

    \section*{二項式展開}

    \begin{enumerate}
        \item (5分)記$C_r^n=\frac{n!}{(n-r)!r!}$。\begin{enumerate}
            \item (2分)求$\sum_{k=0}^n C_k^n$。
            \item (3分)證明$\sum_{\textrm{奇數}k}^n C_k^n=\sum_{\textrm{偶數}k}^n C_k^n$。
        \end{enumerate}
        \item (5分)記$\begin{pmatrix}
            n\\x_1,x_2,\dots,x_k
        \end{pmatrix}=\frac{n!}{x_1!x_2!\cdots x_k!}$。\begin{enumerate}
            \item (3分)證明若$x_1+x_2+\cdots+x_k=n$,$$\begin{pmatrix}
                n\\x_1,x_2,\dots,x_k
            \end{pmatrix}=C_{x_1}^n C_{x_2}^{n-x_1}\cdots C_{x_{k-1}}^{n-x_1-x_2-\cdots-x_{k-2}}$$
            \item (2分)由此,證明對於任意正整數$n$,$$(a+b+c)^n=\sum_{\substack{i+j+k=n, \\ i,j,k\textrm{為非負整數}}} \begin{pmatrix}
                n\\i,j,k
            \end{pmatrix}a^ib^jc^k$$
        \end{enumerate}
    \end{enumerate}

    \section*{數學歸納法}

    \begin{enumerate}
        \item (2分)利用數學歸納法,證明對於任意正整數$n$,$n!<n^n$。
        \item (3分)利用數學歸納法,證明對於任意正整數$n$及$-1<r<1$,$$\sum_{k=0}^\infty C_k^{n+k-1}r^k=\frac{(-1)^{n-1}}{(1-r)^n}$$
    \end{enumerate}

    \section*{三角函數}

    \begin{enumerate}
        \item (7分)設$i^2=1$。證明對於任意整數$n$,$$(\cos(\theta)+i\sin(\theta))^n=\cos(n\theta)+i\sin(n\theta)$$
        \item (8分)考慮$\cos(72)=\cos(108)$,證明$\cos(18)=\frac{\sqrt{5}-1}{2}$。
    \end{enumerate}

    \section*{極限}

    \begin{enumerate}
        \item (5分)設$f,g,h$為實函數,并且$f(0)=g(0)=h(0)=0$。若$\lim_{x\to 0}\frac{f(x)-g(x)}{h(x)}=9$,求$$\lim_{x\to 0}\frac{e^{f(x)}-e^{g(x)}}{\sin(h(x))}$$
    \end{enumerate}

    \section*{微分原理}

    \begin{enumerate}
        \item (10分)設$f(x)=\sin{\pi x}$。\begin{enumerate}
            \item (2分)求$f'(0)$;
            \item (8分)設$(g_n)$為函數數列,定義$g_1(x)=f(x)$及$g_n(x)=f(g_{n-1}(x))$。證明對於任意正整數$n$,$g_n'(0)=\pi^n$。
        \end{enumerate}
        \item (5分)寫出令$f(x)=ax^3+bx^2+cx+d$沒有頂點的條件,已知$a,b,c,d$均爲實數及$a\neq 0$。
    \end{enumerate}
\end{document}