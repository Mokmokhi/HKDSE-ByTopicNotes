\documentclass[12pt]{article}
\usepackage{ctex}
\usepackage[english]{babel}
\usepackage{blindtext}
\usepackage{nameref}
\usepackage{fancyhdr}
\usepackage{amsmath,amssymb,amsthm}
\usepackage{graphicx,float}
\usepackage{physics}
\usepackage{pgfplots}
\usepackage[a4paper, total={7in, 9in}]{geometry}
\usepackage{multicol}

\graphicspath{ {../images/} }

\pagestyle{fancy}
\fancyhf{}
\fancyhf[HL]{HKDSE Physics}
\fancyhf[HR]{\rightmark}
\fancyhf[CF]{\thepage}
\fancyhf[FL]{\copyright Mok Owen 2024}

\newcommand{\innerprod}[2]{\langle{#1},{#2}\rangle}
\newcommand{\id}{\mathtt{id}}
\newcommand{\cis}[1]{\mathrm{cis}({#1})}
\renewcommand{\d}[1]{\mathrm{d}{#1}}

\newtheorem{definition}{Definition}[section]
\newtheorem*{theorem}{Theorem}
\newtheorem*{corollary}{Corollary}
\newtheorem*{lemma}{Lemma}
\newtheorem*{proposition}{Proposition}
\newtheorem*{remark}{Remark}
\newtheorem*{claim}{Claim}
\newtheorem*{example}{Example}
\newtheorem*{axiom}{Axiom}

\newtheorem{exercise}{Essential Practice}[subsection]
\newenvironment{solution}{\textbf{Solution.} \par}{\hfill \textit{\dots end of solution}}

\begin{document}
    \begin{abstract}
        Physics, as a newly introduced subject apart from junior secondary, its concepts are relatively complicated when there is a lost of mathematical tools and insufficient explanations. In this set of notes, we will go through the concepts in Highschool physics with more mathematical tools.
    \end{abstract}

    \tableofcontents
    \newpage

    \section{Preliminaries}

    \subsubsection*{Differentiation}

    The first tool for us to use is a powerful mathematical tool, namely \textbf{Differentiation}. The principle concept of differentiation is the approximation of the slope at a point of a curve. We then call such slope, if dependent of time, the \textbf{rate of change} of some function $f(t)$. It will be useful in our discussion.

    \begin{definition}[Limit]
        By a limit of a function $f(t)$ as $t$ approaches some fixed value $a$, we denote and define the writing by \[\lim_{t\to a}f(t):=\begin{cases}
            f(a)&\textrm{if } f(a)\,\textrm{ is well-defined at }t=a\\
            g(a)&\textrm{if } f(x)=g(x) \textrm{ near } t=a \textrm{ but } f(a) \textrm{ is undefined}
        \end{cases}\]
    \end{definition}

    Such concept on limit is a basis for differentiation. Now, we recall that for any straight line $L$ if $A(x_a,y_a)$ and $B(x_2,y_2)$ are points on $L$ then the slope of $L$ can be computed as \[m_L=m_{AB}:=\frac{y_2-y_1}{x_2-x_1}.\]

    It is reasonable to assume that every function depends on $t$ is continuously differentiable, so we are able to discuss differentiation on such functions.

    \begin{definition}[Derivative of a function]
        Let $f(t)$ be a function of $t$ and be continuously differentiable. The \textbf{derivative of $f$ at time $t$}, denoted by either $\derivative{f}{t}$, $f'(t)$ or $\Dot{f}$, is defined by the \textbf{first principle} \[\derivative{f}{t}:=\lim_{h\to 0}\dfrac{f(t+h)-f(t)}{h}.\]
    \end{definition}

    \begin{proposition}
        The following are fundamental results from the first principle of differentiation:\begin{enumerate}
            \item $\derivative{t}(C) = 0$.
            \item $\derivative{t}(t^{\alpha}) = \alpha t^{\alpha-1}$ if $\alpha\neq 0$.
            \item $\derivative{t}(\ln{t}) = \dfrac{1}{t}$ for $t > 0$.
            \item $\derivative{t}(e^{\alpha t}) = \alpha e^{\alpha t}$ for $\alpha\in \mathbb{R}$.
            \item $\derivative{t}(\sin{\alpha t}) = \alpha \cos{\alpha t}$ if $\alpha\neq 0$.
            \item $\derivative{t}(\cos{\alpha t}) = -\alpha \sin{\alpha t}$ if $\alpha\neq 0$.
            \item $\derivative{t}(\tan{\alpha t}) = \alpha \sec^2{\alpha t}$ if $\alpha\neq 0$.
        \end{enumerate}
    \end{proposition}

    Some important rules are needed to fasten the implementation of differentiation.

    \begin{theorem}
        The following rules are sufficient to handle all types of one-variable differentiation. Let $f,g$ be continuous functions of $t$, \begin{enumerate}
            \item Addition rule: $\derivative{t}(f\pm g)=\Dot{f}\pm\Dot{g}$.
            \item Product rule: $\derivative{t}(fg) = \Dot{f}g+f\Dot{g}$.
            \item Quotient rule: $\derivative{t}(\frac{f}{g}) = \frac{\Dot{f}g-f\Dot{g}}{g^2}$ if $g\neq 0$.
            \item Chain rule: $\derivative{t}(f(g(t))) = \Dot{f}(g(t))\cdot \Dot{g}(t)$.
        \end{enumerate}
    \end{theorem}

    We also subscribe the partial derivative to enclose this session.

    \begin{definition}[Partial derivative]
        If $f$ is a two-variable function such that $f:(x,t)\mapsto f(x,t)$, then the \textbf{partial derivatives} writes \begin{itemize}
            \item $\partialderivative{f}{x}$ to be differentiating only with respect to $x$;
            \item $\partialderivative{f}{t}$ to be differentiating only with respect to $t$.
        \end{itemize}
        In particular, we mat assume $x$ and $t$ are variables independent to each other, and immediately define $\partialderivative{x}{t}=0$ and $\partialderivative{t}{x}=0$. All operations on partial derivatives are nearly the same as the derivative introduced before.
    \end{definition}

    \subsection*{Integration}

    The purpose of integration deals with infinite sums, and we shall quickly introduce Riemann Sum to get closed to its concept.

    \begin{definition}[Sigma notation / Summation]
        The \textbf{Sigma notation} $\Sigma$ refers to summing over a certain quantity. Let $a_1,a_2,\dots,a_n$ to be $n$ values, either equal or not, and we define the sum over all these values to be \[\sum_{k=1}^{n}a_k\]
    \end{definition}

    By a graph, we can visualize the idea of integration and write down a clear definition.



    For intuition, we should understand an integration as calculating the area under the curve generated by $y=f(t)$ in a certain interval $[a,b]$, as shown in the graph. To approximate such area, we may use rectangles or trapeziums. Such origination is proposed by Riemann, a great mathematician, and publicized by Darboux, another great mathematician.

    \begin{definition}[Riemann Sum / Definite Integration]
        Define the \textbf{Riemann Sum} of $n$-th partition by \[S_n(f):=\sum_{k=0}^{n-1} \frac{f(t_k)+f(t_{k+1})}{2}(t_{k+1}-t_k)\] The \textbf{Riemann Integral} is an infinite partitioning approximation progress \[\int_a^b f(t) \d{t} := \lim_{n\to \infty}S_n(f)\] of the area under the curve $f(t)$ on the interval $[a,b]$.
    \end{definition}

    It is an appreciated result that Darboux and Steiljes fixed Riemann's thought a lot and we honour them by calling the integration result a \textbf{Riemann-Darboux Integration} or \textbf{Riemann-Steiljes Integral}. Of course, we don't care who made it.

    On behalf of Riemann Integral, which we named it a definite integral by its functionality, the \textbf{Indefinite Integral} can be defined as \[I(f):= \int_{k}^{t} f(t) \d{t}\] by abusing the variable $t$ so as to demonstrate a general result. The value at $k$ is then a constant so that \[I(f)=\int f(t) \d{t} + C\] is an acceptable result.

    As $f$ can be understood as some other function's derivatives, i.e. We are summing up rate of changes, the result of Indefinite integral can be thought of as an \textbf{anti-derivative}.

    \begin{proposition}
        The following are fundamental results anti-derivatives:\begin{enumerate}
            \item $\int 0 dx = C$.
            \item $\int t^{\alpha} \d{t} = \frac{t^{\alpha+1}}{\alpha+1} + C$ if $\alpha\neq -1$.
            \item $\int \frac{1}{t} \d{t} = \ln{t} + C$ for $t > 0$.
            \item $\int e^{\alpha t} \d{t} = \frac{1}{\alpha}e^{\alpha t} + C$ for $\alpha\in \mathbb{R}$.
            \item $\int \cos{\alpha t} \d{t} = \frac{1}{\alpha}\sin{\alpha t} + C$.
            \item $\int \sin{\alpha t} \d{t} = -\frac{1}{\alpha}\cos{\alpha t} + C$.
            \item $\int \sec^2{\alpha t} \d{t} = \frac{1}{\alpha}\tan{\alpha t} + C$.
        \end{enumerate}
        And some advanced results with further insight:\begin{itemize}
            \item $\int \tan{t} \d{t} = \ln|\sec{t}| + C$.
            \item $\int \sec{t} \d{t} = \ln|\sec{t} + \tan{t}| + C$.
        \end{itemize}
    \end{proposition}

    \begin{theorem}
        Let $f,g$ be continuous functions of $t$. Then \begin{itemize}
            \item $\int f(t)\pm g(t) \d{t} = \int f(t) \d{t} \pm \int g(t) \d{t}$.
            \item $\int f(g(t))g'(t) \d{t} = \int f(u) du$.
            \item $\int f(t)g'(t) \d{t} = f(t)g(t) - \int g(t)f'(t) \d{t}{t}$.
        \end{itemize}
    \end{theorem}

    And the final relation combining differentiation and integration is the \textbf{Fundamental theorem of Calculus} (FTC).

    \begin{theorem}[Fundamental theorem of Calculus(FTC)]
        Given a continuous function $f$ of $t$ on $[a,b]$. Then \[\derivative{x}(\int_{a}^{x} f(t) \d{t}{t}{t}) = f(x)\]
    \end{theorem}

    \newpage
    \section{Heat and Gases}

    \newpage
    \section{Newtonian Mechanics}

    An apple fell and gravity discovered. Newton was not hit by any apple, but his revolutionary ideas.

    \subsection{Considering quantities in Newtonian mechanics}

    \subsubsection*{Scalar quantity}

    A \textbf{scalar quantity} is a quantity representing rate and ratios. It contains no information about directions.
    
    \subsubsection*{Vector quantity}

    A \textbf{vector quantity} is a quantity representing rate equipped with directions. The equality about vector quantities preserved directional equivalence as long as quantitative equivalence.

    \subsection{Newton's law of motion}

    \subsubsection*{Momentum}

    Newton concerns about the movement of any objects, placing a high value on the properties of any individuals. Let us consider these kinds of thinking intrinsic, depends on the object itself highly. One individual has a physical quantity called \textbf{mass}, which can be thought of as a measure of inertia of one body when there is no external force acting on the body. It immediately takes up a question: What does it mean to have higher inertia? Newton provided a natural thinking on this question, that if one body with higher mass is moving with a certain velocity, the higher tendancy it continues to move in a certain manner. Similarly, for some certain objects with same mass, we may admit that one with higher velocity is much likely to continue its run, or we consider it is harder to be stopped. From this view point, we concluded one measure to combine these assumption, which is called \textbf{momentum}, a vector-valued quantity \[\mathbf{p}=m\mathbf{v}\] given by the product of its mass $m$ and its velocity vector $\mathbf{v}$. The words originated from Latin word `\textit{pellere}' means `push', showing its connection with the tendency of continuing its movement.

    From this point of view, we may also understand the mass as the scaling factor for a given velocity to calculate momentum. For an object $A$ with mass $m_A$ and an object $B$ with mass $m_B$, if $m_A>m_B$, then $A$ has a higher tendency than $B$ to continue its movement after any collision.

    \subsubsection*{Newton's first law of motion: Law of inertia}

    The first law stated as follows:

    \begin{center}
        \textit{A body remains at rest or in uniform motion unless acted upon by a force.}
    \end{center}

    For example, a ball staying on a ground without movement continues to stay at rest on the ground unless there are winds or humans picking it up. Of course if it is rolling then it will keep rolling unless there is something changes it.

    The statement includes the word \textit{force}, which acts as the changer of the momentum. Therefore, in mathematical representation, it is like \[\mathbf{F}=\derivative{\mathbf{p}}{t}\] so that the amount of force given to the object affects the scale and the direction of change in momentum.

    \subsubsection*{Newton's second law of motion: Equation of motion}

    The second law stated as follows:

    \begin{center}
        \textit{A body acted upon by a force moves in such a manner that the time rate of change of momentum equals the force.}
    \end{center}

    It is the wordy version of the formula \[\mathbf{F}=\derivative{\mathbf{p}}{t}\] and if we see $\mathbf{F}$ a instantaneous measure of force and the mass $m$ of an object is fixed under the inertial frame, we have \[\mathbf{F}=m\derivative{\mathbf{v}}{t}=m\mathbf{a}\]

    The formulation tells us the relation between acceleration and force: if we apply a greater force to an object, it will much likely to change its motion with the direction of given force, which mathes our daily life experience very much. For example, the slow-motion of a boxer puching others' faces deals a great impact on one's face and destroy the facial structure in a short period. May the force be with you!

    \subsubsection*{Newton's third law of motion: Action and reaction pair}

    The third law stated as follows:

    \begin{center}
        \textit{If two bodies exert forces on each other these forces are equal in magnitude and opposite in direction.}
    \end{center}

    That is, if one object acts force on some other object, then a force of same magnitude will be reacted immediately back to the original force actor.

    We may formulate the statement in the following form: suppose two objects are colliding with each other, one object is called $A$ and another is called $B$. Denote the force acting on $B$ from $A$ by $\mathbf{F}_{A\to B}$ and similarly for the force acting on $A$ from $B$ we denote it by $\mathbf{F}_{B\to A}$. Then it is natural to write \[\mathbf{F}_{A\to B}=\mathbf{F}_{B\to A}\]

    \begin{example}
        For a person with mass $m$, by believing the gravitational acceleration is actually $g$ ms$^{-2}$, we could write the force acting on the floor by the person is \[\mathbf{F}_{\downarrow}=mg.\] By the Newton's third law, the magnitude of the reacting force from the ground to the person will be as same as the person gives, that is \[\mathbf{F}_{\uparrow}=-\mathbf{F}_{\downarrow}=-mg\] for if we take downward as positive direction.
    \end{example}

    Since the concept of force comes from Newton, we granted the unit of forces to be Newton, denote by N.

    Let us revisit the concept of momentum right here using integration. Recall from newton's second law that \[\mathbf{F}=\derivative{\mathbf{p}}{t}\] We may solve $\mathbf{p}$ by $\mathbf{F}$ in the following way:\begin{align*}
        \mathbf{p}=\int \derivative{\mathbf{p}}{t} \d{t}{t}&=\int \mathbf{F} \d{t}{t}
    \end{align*}

    For a given average force $\mathbf{\bar{F}}$ within a time interval $[0,t]$, we may define \[\mathbf{\bar{F}}:=\frac{1}{t}\int_{0}^{t} \mathbf{\bar{F}} \d{t}{t}\] and the change in momentum $\Delta \mathbf{p} := \mathbf{p} - \mathbf{p}_0 = m(\mathbf{v}-\mathbf{u})$ where $\mathbf{p}$ denotes the final momentum, $\mathbf{p}_0$ denotes the initial momentum, $\mathbf{v}$ denotes the final velocity and $\mathbf{u}$ denotes the initial velocity. The substitution yields \begin{align*}
        \mathbf{\bar{F}}t = \int \mathbf{F} \d{t} &= \Delta \mathbf{p} = m(\mathbf{v}-\mathbf{u})\\
        \mathbf{\bar{F}} &= m\frac{\mathbf{v}-\mathbf{u}}{t}
    \end{align*}
    which is consistent in the connection of average force and average acceleration. We thus find \[\mathbf{v}=\mathbf{u}+\mathbf{a}t\] when $\mathbf{a}$ is a constant vector. This is called the \textbf{equation of uniform acceleration motion}.

    \subsection{Work, Energy and Potential}

    So long we have been taught that we have to be an `energetic' person. However, what energy actually means is quite confusing. I shall refer to Aristotle for his intuition on Energy: 

    \begin{center}
        \textit{Energy is a condition that describes the capacity to do work.}
    \end{center}

    We can easily refer the sentence to our daily experience. We feel tired when we have worked for a long time, and that is a consequence of losing energy.

    \subsubsection*{Work}

    In physics, especially in Mechanics, we define work as a measure of displacement scaling. We examine some theoretical relation:\begin{itemize}
        \item The higher the force acting on the object, the more work is done to the object;
        \item The longer the distance it undergoes, the more work it does;
        \item There is no direct relation between force and distance.
    \end{itemize}
    From the theoretical relation we may obtain the work formula \[\delta W = \mathbf{F} \d{\mathbf{s}}\] of which the $\delta$ means an infinitesimal change. Then we consider working is equivalent to changing energy, so we put power $P$ as the rate of change in energy \begin{align*}
        P&=\frac{\delta W}{\d{t}}=F\derivative{\mathbf{s}}{t}=\mathbf{F\cdot v}\\
        \Delta E&=\int P \d{t} = \int \frac{\delta W}{\d{t}} \d{t} = W
    \end{align*}

    Assume we are delivering a constant force on an object, then the work done to the object is exactly \[W=\int \mathbf{F\cdot v} \d{t} = \int \sum \mathbf{F}_k \mathbf{v}_k \d{t} = \sum \mathbf{F}_k \int \mathbf{v}_k \d{t} = \sum \mathbf{F}_k \mathbf{s}_k = \mathbf{F\cdot s}\]

    \subsubsection*{Kinetic Energy}

    In terms of moving objects, we see $\mathbf{v}$ as the principle variable. That is, with reference to motion, velocity is considered to be a first choice of computation. Consider $\mathbf{F}=\derivative{\mathbf{p}}{t}$, the substitution yields \begin{align*}
        W&=\int \mathbf{F\cdot v} \d{t} = \int \derivative{\mathbf{p}}{t} \cdot \mathbf{v} \d{t} = \int m\d{\mathbf{v}}\cdot \mathbf{v}\\
        &=\int m\sum \mathbf{v_k} \d{\mathbf{v}_k}=m\sum \int \mathbf{v}_k \d{\mathbf{v}_k}\\
        &=\frac{1}{2}m\sum (\mathbf{v}_k^2-\mathbf{u}_k^2) = \frac{1}{2}m\mathbf{v}^2 - \frac{1}{2}m\mathbf{u}^2
    \end{align*}
    where $\mathbf{v}^2:=\mathbf{v}\cdot\mathbf{v}$. This is called the \textbf{kinetic work} of an object. For if we define $E_K:=\frac{1}{2}m\mathbf{v}^2$ to be the \textbf{kinetic energy level} of an object at a certain velocity, we can say \[\Delta E_K = \frac{1}{2}m\mathbf{v}^2 - \frac{1}{2}m\mathbf{u}^2 = W\] with initial velocity $\mathbf{u}$ and final velocity $\mathbf{v}$.

    For if we consider uniform acceleration motion, we may write $\mathbf{a}$ and $\mathbf{F}=m\mathbf{a}$ as constants, and equate the work formulae \begin{align*}
        \mathbf{F}\cdot \mathbf{s} &= \frac{1}{2}m\mathbf{v}^2-\frac{1}{2}m\mathbf{u}^2\\
        m\mathbf{a\cdot s} &= \frac{1}{2}m\mathbf{v}^2-\frac{1}{2}m\mathbf{u}^2\\
        2\mathbf{a\cdot s} &= \mathbf{v}^2-\mathbf{u}^2\\
        \mathbf{v}^2&=\mathbf{u}^2+2\mathbf{a\cdot s}
    \end{align*}

    \subsubsection*{Potential Energy}

    When a system is considered, i.e. a whole space with some movable objects included, potential energy exists. We call something potential is to measure the tendency of leaving its current position or approaching any position. For example, in a streched string, it contains more potential than an unstretched string due to its higher tendency on returning an unstreched state.

    Potential energy is not always measurable. Indeed, computed from the change in other format.

    For a closed system in mechanics, we always assume that the \textbf{total mechanical energy} is constant. Hence, for a one-object system, we can write, if the potential energy is denoted by $U$, then \[\partialderivative{U}{\mathbf{r}}=-\partialderivative{E_K}{\mathbf{r}}\] with $\mathbf{r}$ defining the position vector. That yields \[U+E_K= \mathrm{constant}\]

    Although it is a blurred concept for potential energy, we still have some observable types of potential. One famous potential energy is the gravitational potential energy. Following the work equation, we have \[\begin{cases}
        F=mg\\
        s=h
    \end{cases}\implies U=mgh\] where $m$ is the mass of the object, $g$ is the gravitational acceleration, and $h$ is the hegiht from the ground (relatively 0 height). Of course, we are assuming the change in gravitational acceleration is neglegible, otherwise the system blows. For if we are concerning large scale physics, we shall have \[\begin{cases}
        F=\frac{GMm}{r^2}\\
        \partialderivative{U}{r}=-\mathbf{F\cdot \d{s}}
    \end{cases}\implies U=-\int F \d{s} = -\int \frac{GMm}{r^2} \d{r} = \frac{GMm}{r}\] as a spatial conclusion. We will return to this form in the discussion of Gravity.

    Another type of observable potential energy is the elastic potential. Consider a spring tied with the ceiling at one end, with zero potential at its free state. When a pulling force is acting on the other end, we will see for each magnitude level of the pulling force, there is a corresponding equilibria for the stretching spring. We can model the statement by \[\mathbf{F}=k(\mathbf{x-x}_0)\] where $\mathbf{F}$ denotes the pulling force, $\mathbf{x}$ denotes the current position, $\mathbf{x}_0$ denotes the free equilibrium, and $k$ is a scaling factor related to the variation, called \textbf{elasticity} of the spring. Then the potential stored in a certain position is \[\int \mathbf{F\cdot \d{s}} = \int k(\mathbf{x-x}_0) \d{\mathbf{x}} = \frac{k}{2}(\mathbf{x-x}_0)^2\]

    \subsubsection*{Energy conversion and conservation}

    Can energy be created? A generally accepted answer is negative. Due to classical theory of mechanics, energy can only be transferred or be converted into different format, but can never be created or be destroyed. We then considered it as a basis for classical mechanics and axiomatically stated:

    \begin{center}
        \textit{Energy can neither be created from null nor be destroyed manually.}
    \end{center}

    For if we consider all source of energy by writing a set $S$ as the source set and denote $E_X$ to be the energy in source $X$, we will see that \[\sum_{X\in S}E_X = M\] where $M$ is the total energy in a system. That means $M$ is constant and we may find the change in energy of a closed system follows the relation \[\sum_{X\in S} \Delta E_X = \Delta (\sum_{X\in S} E_X) = \Delta M = 0\]

    It is not the most accurate description of our world as there will never be a truly closed system in our observable universe. However, it is enough for us to do calculation in a short period for a specific collision.

    \begin{example}
        Consider two objects $A$ and $B$ are collided on a horizontal smooth plane with each other with initial speed $u_A$ and $u_B$ respectively. Assume the collision is elastic, i.e.\ energy is totally conserved between $A$ and $B$ without any loss. Then we may set up the system of equations \begin{align*}
            \begin{cases}
                \Delta \mathbf{p} = 0\\
                \Delta E_K = 0
            \end{cases}&\implies \begin{cases}
                m_A u_A + m_B u_B = m_A v_A + m_B v_B\\
                \frac{1}{2}m_A u_A^2+\frac{1}{2}m_B u_B^2=\frac{1}{2}m_A v_A^2+\frac{1}{2}m_B v_B^2
            \end{cases}\\
            &\implies \begin{cases}
                v_A=\dfrac{(m_A-m_B)u_A+2m_B u_B}{m_A+m_B}\\
                v_B=\dfrac{(m_B-m_A)u_B+2m_A u_A}{m_A+m_B}
            \end{cases}
        \end{align*}
        The result of $v_A$ and $v_B$ means the instantaneous velocity of $A$ and $B$ just after the collision, as energy is conserved in a short period closed system. We may also notice that the resultant velocities are symmetric to each other, i.e.\ they interchanged the position of $A$ and $B$, which is a calculation surprise but realistically matched in the view point of relative motion.
    \end{example}

    When our collision system contains energy loss, we will then say the collision is \textbf{inelastic}. There should be other format of source that carries part of the initial energy away. We will identify the source of energy after the calculation.

    A table can be drawn to conclude the types of collision:

    \begin{center}
        \begin{tabular}{|c|c|}
            \hline
            Energy conversion&Types of collision\\
            \hline
            totally conserved&elastic\\
            \hline
            not totally conserved&inelastic\\
            \hline
            attains minimal after collision&completely inelastic\\
            \hline
            outranged the original total energy&superelastic\\
            \hline
        \end{tabular}
    \end{center}

    For a completely inelastic collision, we can deduce that

    \begin{example}
        With the same setup as the previous example, but now energy is minimally conserved, we consider \[\derivative{\Delta E_K}{t}=m_A v_A a_A+m_B v_B a_B = F_A v_A + F_B v_B = 0,\] with Newton's third law we can say \[F_A = -F_B\] and that means \[v_A = v_B\]
    \end{example}

    \subsection{Center of Mass}

    When an object is considered free falling, we usually find its acting force at its \textbf{center of mass}, which is a point with consistent movement. However, where, and how, could we deduce it reasonably?

    \subsubsection*{The turning effect}

    For a rigid body, we may describe it naively as follows:

    \begin{center}
        \textit{A rigidbody maintains its shape in any movement in a certain space. In other words, the deformation of a rigid body is neglegible in Newtonian motions.}
    \end{center}

    For example, a ball keeps a spherical shape in any position unless any force is acted on it. For hard metals and steel-made objects, we can never easily change its shape, and we call them rigidbodies.

    It is of course not the best description of real world objects. But it is enough to declare the phenomenon of rotation.

    To be simple, we may examine a straight rod with uniform mass, rotating in constant angular speed $\omega$. We also consider the arclength of a small sector by \[s=r\theta\] where $\theta$ is the angle of rotation in radian, $r$ is the radius of rotation, and $s$ is the arclength. By differentiation we get \[\mathbf{v}=r\omega\] where $\mathbf{v}$ denotes the speed of rotation. We may now examine the conserved quantity, angular momentum, as in usual description, by taking \[\d{\mathbf{p}}=\mathbf{v}\d{m}\] such that \[\mathbf{p}=\int \mathbf{v} \d{m} = \omega \int r\d{m}=\omega \int r \derivative{m}{r} \d{r}=\frac{\rho}{2}r^2\omega\] where $\rho$ defines the density per unit length of the rod. Then \[\mathbf{F}=\rho r \omega \]

    \subsubsection*{Moment}

    \newpage
    \section{Electricity and Magnetism}

    \newpage
    \section{Waves}

    \newpage
    \section{Nuclear Power}

    \newpage
    \section{Extended Part I: Astronomy}

    \newpage
    \section{Extended Part II: Atomic World}

    \newpage
    \section{Extended Part III: Domestic Electrical Appliances}

    \newpage
    \section{Extended Part IV: Medical Physics}

\end{document}