\documentclass[12pt]{book}
\usepackage{ctex}
\usepackage[english]{babel}
\usepackage{blindtext}
\usepackage{nameref}
\usepackage{fancyhdr}
\usepackage{amsmath,amssymb,amsthm}
\usepackage{graphicx,float}
\usepackage{physics}
\usepackage{pgfplots}
\usepackage[a4paper, total={7in, 9in}]{geometry}
\usepackage{multicol}

\graphicspath{ {../images/} }

\pagestyle{fancy}
\fancyhf{}
\fancyhf[HL]{HKDSE Section A(1) analysis}
\fancyhf[HR]{\rightmark}
\fancyhf[CF]{\thepage}
\fancyhf[FL]{\copyright Mok Owen 2024}

\title{HKDSE Section A(1) analysis}
\author{Mok Owen}
\date{\today}

\newcommand{\innerprod}[2]{\langle{#1},{#2}\rangle}
\newcommand{\id}{\mathtt{id}}

\newtheorem{definition}{Definition}[section]
\newtheorem*{theorem}{Theorem}
\newtheorem*{corollary}{Corollary}
\newtheorem*{lemma}{Lemma}
\newtheorem*{proposition}{Proposition}
\newtheorem*{remark}{Remark}
\newtheorem*{claim}{Claim}
\newtheorem*{example}{Example}
\newtheorem*{axiom}{Axiom}

\newtheorem{exercise}{Essential Practice}[subsubsection]
\newenvironment{solution}{\textbf{Solution.} \par}{\hfill \textit{\dots end of solution.}}

\begin{document}
    \maketitle
    
    \tableofcontents

    \newpage

    \chapter*{Introduction}

    \chapter{Index Law}

    The problems of index laws requires the following knowledge:

    Given $a,b$ be real numbers and $m,n$ be positive integers. Then
    \begin{enumerate}
        \item $a^m\cdot a^n=a^{m+n}$.
        \item $\frac{a^m}{a^n}=a^{m-n}$.
        \item $(a^m)^n=a^{mn}$.
        \item $a^0=1$.
        \item $a^{-n}=\frac{1}{a^n}$.
        \item $(ab)^n=a^n b^n$.
        \item $(\frac{a}{b})^n=\frac{a^n}{b^n}$.
    \end{enumerate}

    \begin{example}[2012-PAPER-1 Q1]
        Simplify $\dfrac{m^{-12}n^8}{n^3}$ and express your answer with positive indices.

        \begin{solution}
            Following the law of index, we deduce
            \begin{align*}
                \frac{m^{-12}n^8}{n^3}&=m^{-12}n^{8-3} &(by\, 2)\\
                &=m^{-12}n^5\\
                &=\frac{n^5}{m^{12}}&(by\, 5)
            \end{align*}

            As the problem requires positive indices, we have to apply the fifth law to remove all negative powers.
        \end{solution}
    \end{example}
    
    \chapter{Subject arrangement}

    Subject arrangement is one of the pure algebraic action. Its concept of arrangement is similar to solving equations with one variable.

    Let's consider Solving $4x+2=6x-3$. We have \begin{align*}
        4x+2&=6x-3\\
        4x+2-2&=6x-3-2\\
        4x&=6x-5\\
        4x-6x&=6x-5-6x\\
        -2x&=-5\\
        x&=\frac{-5}{-2}\\
        &=\frac{5}{2}
    \end{align*}

    Of the same course, we could solve $4x+k=6x+\ell$ by the same procedure:\begin{align*}
        4x+k&=6x+\ell\\
        -2x&=\ell-k\\
        x&=\frac{k-\ell}{2}
    \end{align*}

    Therefore, we could solve for even more variables:\begin{align*}
        ax+k&=bx+\ell\\
        (a-b)x&=\ell-k\\
        x&=\frac{\ell-k}{a-b}
    \end{align*}

    This is called making $x$ to be the subject of an equation, where making subject is equivalent to solving equation by seeing the required variable as the only unknown.

    \begin{example}[2012-PAPER-1 Q1]
        Make $a$ the subject of the formula $\dfrac{3a+b}{8}=b-1$.

        \begin{solution}
            To make $a$ the subject of the formula, we see $a$ as the only variable of the equation.\begin{align*}
                \frac{3a+b}{8}&=b-1\\
                3a+b&=8b-8&(multiply\, both\, sides\, by\, 8)\\
                3a&=7b-8&(isolate\, a)\\
                a&=\frac{7b-8}{3}&(divide\, sides\, by\, 3)
            \end{align*}
        \end{solution}
    \end{example}

    \chapter{Expansion \& Factorization}

    \chapter{Identity}

    \chapter{Percentage}

    \chapter{Polar Coordinates}

    \chapter{Congruent and similar triangle}

    \chapter{Properties of circle}

    \chapter{Measure of Dispersion}

    \chapter{Challenging revision}
\end{document}