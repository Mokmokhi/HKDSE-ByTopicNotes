\documentclass[12pt]{article}
\usepackage{ctex}
\usepackage[english]{babel}
\usepackage{blindtext}
\usepackage{nameref}
\usepackage{fancyhdr}
\usepackage{amsmath,amssymb,amsthm}
\usepackage{graphicx,float}
\usepackage{physics}
\usepackage{pgfplots}
\usepackage[a4paper, total={6in, 9in}]{geometry}

\graphicspath{ {../images/} }

\pagestyle{fancy}
\fancyhf{}
\fancyhf[HL]{Function and graphs}
\fancyhf[CF]{\thepage}

\newcommand{\innerprod}[2]{\langle{#1},{#2}\rangle}
\newcommand{\id}{\mathtt{id}}

\newtheorem*{definition}{Definition}
\newtheorem*{theorem}{Theorem}
\newtheorem*{corollary}{Corollary}
\newtheorem*{lemma}{Lemma}
\newtheorem*{proposition}{Proposition}
\newtheorem*{remark}{Remark}
\newtheorem*{claim}{Claim}
\newtheorem*{example}{Example}
\newtheorem*{axiom}{Axiom}

\begin{document}
    \section*{Learning objectives}
    By studying this unit, we will achieve the following goals:
    \begin{enumerate}
        \item Recognising the intuitive concepts of functions, domains and co-domains, independent and dependent variables
        \item Recognising the notation of functions and use tabular, algebraic and graphical mathods to represent functions.
        \item Understanding the features of the graphs of quadratic functions.
        \item Finding the maximum and minimum values of quadratic functions by the algebraic method.
    \end{enumerate}

    \section*{Background}
    Nowadays, humanity rely on technology very much, and we always have to get outputs from the computers. We input something, wait for the process, and it output something. This  is called the input-process-output procedure. Throw back to a long time ago, whenever there is business, the stratgies are similar. We gave money (input) to the businessman, wait for his process, and he returned some products (output) to us. We don't know how he packed things up, but we simply accept the product and go home. This was the origination of function.

    A function consists of input and output, with no information about the process. Of course, in mathematics, or if we want to complete the concept of function, we still need to know about the process so that we could talk about properties of a function.
\end{document}