\documentclass[12pt]{article}
\usepackage{ctex}
\usepackage[english]{babel}
\usepackage{blindtext}
\usepackage{nameref}
\usepackage{fancyhdr}
\usepackage{amsmath,amssymb,amsthm}
\usepackage{graphicx,float}
\usepackage{physics}
\usepackage{pgfplots}
\usepackage[a4paper, total={6in, 9in}]{geometry}

\graphicspath{ {../images/} }

\pagestyle{fancy}
\fancyhf{}
\fancyhf[HL]{Function and graphs}
\fancyhf[CF]{\thepage}

\newcommand{\innerprod}[2]{\langle{#1},{#2}\rangle}
\newcommand{\id}{\mathtt{id}}

\newtheorem*{definition}{Definition}
\newtheorem*{theorem}{Theorem}
\newtheorem*{corollary}{Corollary}
\newtheorem*{lemma}{Lemma}
\newtheorem*{proposition}{Proposition}
\newtheorem*{remark}{Remark}
\newtheorem*{claim}{Claim}
\newtheorem*{example}{Example}
\newtheorem*{axiom}{Axiom}

\begin{document}
    \section*{Learning objectives}
    By studying this unit, we will achieve the following goals:
    \begin{enumerate}
        \item Recognising the intuitive concepts of functions, domains and co-domains, independent and dependent variables
        \item Recognising the notation of functions and use tabular, algebraic and graphical mathods to represent functions.
        \item Understanding the features of the graphs of quadratic functions.
        \item Finding the maximum and minimum values of quadratic functions by the algebraic method.
    \end{enumerate}
\end{document}