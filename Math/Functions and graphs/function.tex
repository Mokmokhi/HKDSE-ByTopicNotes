\documentclass[12pt]{article}
\usepackage{ctex}
\usepackage[english]{babel}
\usepackage{blindtext}
\usepackage{nameref}
\usepackage{fancyhdr}
\usepackage{amsmath,amssymb,amsthm}
\usepackage{graphicx,float}
\usepackage{physics}
\usepackage{pgfplots}
\usepackage[a4paper, total={7in, 9in}]{geometry}

\graphicspath{ {../images/} }

\pagestyle{fancy}
\fancyhf{}
\fancyhf[HL]{Function and graphs}
\fancyhf[CF]{\thepage}

\newcommand{\innerprod}[2]{\langle{#1},{#2}\rangle}
\newcommand{\id}{\mathtt{id}}

\newtheorem{definition}{Definition}[section]
\newtheorem*{theorem}{Theorem}
\newtheorem*{corollary}{Corollary}
\newtheorem*{lemma}{Lemma}
\newtheorem*{proposition}{Proposition}
\newtheorem*{remark}{Remark}
\newtheorem*{claim}{Claim}
\newtheorem*{example}{Example}
\newtheorem*{axiom}{Axiom}

\newtheorem{exercise}{Essential Practice}[subsubsection]
\newenvironment{solution}{\textbf{Solution.} \par}{\hfill \textit{\dots end of solution}}


\begin{document}
    \begin{abstract}
        In this pieces of notes, we will go through the concepts related to functions and the meaning of graphs of the functions.
    \end{abstract}

    \tableofcontents

    \newpage

    \section{Function and its graph}

    `Functions describes the world!', one Professor in Mathematics of Massachusetts Institute of Technology (a.k.a. MIT) said that. His speech was greatly influential, as I have never heard such conclusive thinking about functions. In fact, in the past few years, whenever I was studying in schools, my thought about functions is always only about projecting elements from oen set to another set, but what he said had a big impact to my knowledge about functions.

    What function talks about, is a subjection of one elements to one another. It can be thought of as a pointing action started from element $A$ to element $B$, which not so far away, if we could think of subjecting a lot of, a bunch of, or a list of, whatever, objects from one collection to another collection, and they can all be matched under this pointing action, then it is a so-called function.

    For example, in an Indian factory, the production of food undergoes many different process. Those can all be called functions. Let say a raw material comes to the factory first, it then be put to a machine to chop into many small pieces. It is the chopping function inside the factory. Next, the chopped material will be put into a pool of yellowish-brownish liquid and be stirred by dirty hands. It is the Mixing function in the factory. After that, the liquid will be drained on the dirty floor and be stepped on by Indian workers so that they can be smell freshed. It is the flavouring function in the factory. Finally, it will be sold to stores, which is the selling function. 

    Another example is what our body does. We eat and drink, going down the digestive system, and we sit on a toilet. Although we stupid human knows nothing about how the digestive system works, we could still name the conversion from food to poops a digestion, which means the digestive function representing the process in our body.

    So we know that function as an english word represents the naming of a process of conversion, it is the time to explore how Math functions works.

    \subsection{What is a Function?}

    A function is defined as follow:

    \begin{definition}[Function]
        Given an input $x$ and an output $y$, a \textbf{function} is a relation between $x$ and $y$ so that we can write $y=f(x)$ to represent the relationship. 
    \end{definition}

    \begin{exercise}
        Write down functions for the following input-output variables:\begin{enumerate}
            \item $u$ as input and $v$ as output;
            \item $b$ as input and $a$ as output;
            \item $n$ as input and $1$ as output (which we call it a constant function);
            \item $x^2$ as input and $y$ as output;
            \item $xy$ as input and $z$ as output;
            \item $2^x$ as input and $k$ as output;
            \item $\sqrt{p}$ as input and $q$ as output;
        \end{enumerate}
    \end{exercise}

    \begin{remark}
        It is notable that we may write functions as $y=g(x)$, $y=h(x)$, $y=d(x)$, \dots as we want. The `naming' of a function is always definitive and up to user's construction.
    \end{remark}

    We can also apply functions after functions. To do so, we have to talk about the following:

    \begin{definition}[Composite functions]
        Let $f$ and $g$ be functions such that $f$ takes $x$ as input and $y$ as output, and $g$ takes $y$ as input and $z$ as output. Then we can say there is a function $h$ that takes $x$ as input and $z$ as output. In other words, $h$ is a \textbf{composite function} such that $z=h(x)=g(f(x))$.
    \end{definition}

    \begin{exercise}
        Let $f$ and $g$ be functions such that $b=f(a), c=g(b)$. Write a function for $a$ as input and $c$ as output.
    \end{exercise}
        

    \subsubsection{Function as an input-output pair}

    We may now consider how functions carry things to things using arrow notations. We may use a stroked arrow $\mapsto$ to emphasis the carrying process. Let's take a look at the following examples.

    \begin{example}
        Given a function that undergoes the process of adding one to the given element. We can say $$\dots, 1\mapsto 2, 2\mapsto 3, 3\mapsto 4, \dots$$

        In other words, we may write $$x\mapsto x+1$$ to generalize the process of adding one to the given element, which is how we usually write to describe any functions.
    \end{example}

    The example shows that we can generalize the process using algebraic notation, in which it is usually in the form of $$x\mapsto f(x)$$ with the function $f$. It is equivalent to say that $$\dots, 1\mapsto f(1), 2\mapsto f(2), \dots$$ but what's important is how explicit the mapping process is done. More examples could be viewed to familiarize with it.

    \begin{example}
        To describe a function undergoes the process `multiplying the element by two and add one to it', we may examine that $$\dots, 0\mapsto 1, 1\mapsto 3, 2\mapsto 5,\dots$$ so that it is equivalent to write $$x\mapsto 2x+1$$ as a generalization of the function. It is equivalent to write $f(x):=2x+1$ to emphasis that we shall call the function $f$ as a naming for the given process, as long as we can write $$x\mapsto f(x), f(x):=2x+1$$
    \end{example}

    \begin{exercise}
        Examine the following functions from 0 to 3, and generalize the function using algebraic notation. You may either choose writing $x\mapsto \square$ directly or $x\mapsto f(x), f(x):=\square$.\begin{enumerate}
            \item Multiply the element by 3 and then add 2 to it.
            \item Divide the element by 10 and then Subtract 7 from it.
            \item Multiply the element by $a$ and then add $b$ to it.
            \item Squaring the element.
            \item Multiply 3 to the square of the element, an then add 5 to it.
            \item Add 1 to the element first, then multiply the square of the result by 6, and then add 1 to it.
            \item Subtract 8 from the element first, then divide the square of the result by 2, and then add 4 to it.
            \item Subtract $h$ from the element first, then multiply the square of the result by $a$, and then add $k$ to it.
        \end{enumerate}
    \end{exercise}

    It is notable that a function can only gives one output for each input, which makes the next page a fruitful discussion.

    \subsubsection{Defining a function with variables}

    So far, we have learnt how writing a generalization of a function is, and have we used algebraic nnotation to shorten the examination, one suggest we can always write algebraic notation for a function definition. In addition, we also find that a function cannot have more than one output, as we see functions as a pointing process from one element to another element. So we have the following definition:

    \begin{definition}[Function]
        A \textbf{function $f$ of $x$} defines the pointing process $x\mapsto f(x)$ is a one-to-one pointing process, which can have only one output.
    \end{definition}

    It is important to note that $f(x)$ is a function of $x$ if and only if one $x$ produce one $f(x)$. The $x$ is called a \textit{dummy variable}, which is a variable that can be changed all the time. For example, writing $f(y)$ or $g(z)$ still makes sense to say a function, but not of $x$.

    From now on, we can determine whether a given relation is a function or not.

    \begin{example}
        Given the relation $y=mx+c$. Since one $x$ can produce only one $y$, $y$ is a function of $x$; on the other hand, we also see one $y$ produces only one $x$, so $x$ is also a function of $y$.
    \end{example}

    \begin{example}
        Given the relation $y=x^2$. Since one $x$ can produce only one $y$, $y$ is a function of $x$; however,  one $y$ may produces more than one $x$, say if $y=4$ then $x$ can be $2$ or $-2$, so $x$ is not a function of $y$.
    \end{example}

    \begin{exercise}
        Determine whether the following given relation between $x$ and $y$ is a function of one another or not. Provide counterexample if it is not a function.\begin{enumerate}
            \item $y=-x$;
            \item $y=4x+3$;
            \item $y=\frac{1}{x}$;
            \item $y=\frac{x}{6}$;
            \item $y^2=x$;
            \item $y^2=x^2$;
            \item $y^3=4x^2-3$.
        \end{enumerate}
    \end{exercise}
    \subsubsection{Domain, Codomain and Range}

    For explicit definition of a function, we need the following concepts to help with: \textit{domain}, \textit{codomain} and \textit{range}.

    A \textbf{domain} is where the input comes from, which is usually half-customized and half-restricted. For example, $f(x)=\frac{1}{x}$ can have input of negative real numbers, positive real numbers, any complex numbers except 0. This means that 0 is naturally restricted by the operation of $\frac{1}{x}$, but other than 0, we can choose freely our input from all complex numbers. Thus, the largest domain of $f(x)=\frac{1}{x}$ is all complex numbers except 0. However, it is not saying that the domain of $f(x)=\frac{1}{x}$ must be all complex numbers except 0, we can still put restrictions on our own, what means by customize, like all real numbers except 0 or all positive real numbers except 0 as its domain, is still a possible choice. Hence, we shall usually talks about the \textit{greatest possible domain of a function} if we need to find the natural restrictions, and the \textit{domain of a function} if we are going to define our source of input.

    \begin{exercise}
        Find the greatest possible domain of the following functions if (i) the output is restricted to complex numbers and (ii) the output is restricted to be real numbers:\begin{enumerate}
            \item $f(x):=x$;
            \item $f(x):=ax+b$;
            \item $f(x):=\frac{a}{x}$;
            \item $f(x):=x^2$;
            \item $f(x):=a(x-h)^2+k$;
            \item $f(x):=\sqrt{x}$;
            \item $f(x):=\sqrt[3]{x}$;
            \item $f(x):=\frac{1}{\sqrt{x}}$;
        \end{enumerate}
    \end{exercise}

    A \textbf{codomain} is where the output can go to. It is more likely a limitation of the output of the function so that we know where our target is. Similarly, we shall usually talks about the \textit{greatest possible codomain of the function} if we need to find the natural restrictions, and the \textit{domain of the function} if we are going to define our target output.

    \begin{exercise}
        Find the greatest possible codomain of the following functions if the input is unrestricted:\begin{enumerate}
            \item $f(x):=x$;
            \item $f(x):=ax+b$;
            \item $f(x):=\frac{a}{x}$;
            \item $f(x):=x^2$;
            \item $f(x):=a(x-h)^2+k$;
            \item $f(x):=\sqrt{x}$;
            \item $f(x):=\sqrt[3]{x}$;
            \item $f(x):=\frac{1}{\sqrt{x}}$;
        \end{enumerate}
    \end{exercise}
    \subsection{Graph of a function}

    \subsubsection{Blobs-and-arrows diagram for discrete functions}

    \subsubsection{The pairing table and xy-coordination}

    \subsection{Fundamental properties of a graph}

    \subsubsection{points lying on the graph}

    \subsubsection{x-intercept}

    \subsubsection{y-intercept}

    \subsubsection{general intersection}

    \subsection{Solving equations using graphs of functions}

    \subsection{Transformation of functions}

    \subsubsection{Translation}

    \subsubsection{Dilation}

    \subsection{Challenging questions}

    \newpage

    \section{Linear functions}

    \subsection{Fundamental concepts of points}

    \subsubsection{Distance between two points}

    \subsubsection{Mid-point and Division points}

    \subsection{Different forms of a linear function}

    \subsubsection{Slope of a line}

    \subsubsection{Two-point form}

    \subsubsection{Point-slope form}

    \subsubsection{Slope-intercept form}

    \subsubsection{General form}

    \subsubsection{Special: Intercept-form}

    \subsection{Parallel lines and Perpendicular lines}

    \subsubsection{Point of intersection}

    \subsubsection{Parallel lines}

    \subsubsection{Perpendicular lines}

    \subsubsection{Number of intersections}

    \subsection{Angle of elevation and depression}

    \subsection{Additional content: Point-line distance}

    By definition, the distance between two points on a $\mathbb{R}^2$ plane is 

    \begin{definition}[Distance between two points]
        Let $A(x_1,y_1)$ and $B(x_2,y_2)$ be two points on $\mathbb{R}^2$-plane. The distance between $A$ and $B$ is computed by the formula $$\mathrm{dist}(A,B):=\sqrt{(x_1-x_2)^2+(y_1-y_2)^2}$$
    \end{definition}

    Now let $L:ax+by+c=0$ be a straight line and $P(x_0,y_0)$ be a point on $\mathbb{R}^2$-plane. Note that with the following axiom

    \begin{axiom}
        The distance between a point $P$ and a line $L$ is defined by the shortest distance between $P$ and a point on $L$ $$\mathrm{dist}(P,L):=\inf\{\mathrm{dist}(P,Q):Q\in L\}$$
    \end{axiom}

    we can choose the perpendicular displacement of $P$ from $L$ to define the distance. Using the line $\Gamma$ perpendicular to $L$ passing through $P$, we can find such $Q$ by following computation:

    \begin{align*}
        \begin{cases}
            L:ax+by+c=0\\
            \Gamma:bx-ay+(ay_0-bx_0)=0
        \end{cases}\implies Q(\frac{b^2x_0-aby_0-ac}{a^2+b^2},\frac{a^2y_0-abx_0-bc}{a^2+b^2})
    \end{align*}

    Therefore, the distance between $P$ an $Q$ is \begin{align*}
        \mathrm{dist}(P,Q)&=\frac{1}{a^2+b^2}\sqrt{[(a^2+b^2)x_0-(b^2x_0-aby_0-ac)]^2+[(a^2+b^2)y_0-(a^2y_0-abx_0-bc)]^2}\\
        &=\frac{1}{a^2+b^2}\sqrt{(a^2x_0+aby_0+ac)^2+(b^2y_0+abx_0+bc)^2}\\
        &=\frac{\sqrt{a^2+b^2}\sqrt{(ax_0+by_0+c)^2}}{a^2+b^2}\\
        &=\frac{\sqrt{(ax_0+by_0+c)^2}}{\sqrt{a^2+b^2}}\\
        &=\left|\frac{ax_0+by_0+c}{\sqrt{a^2+b^2}}\right|
    \end{align*}

    \subsection{Linear inequalities}

    \subsubsection{One variable inequality}

    \subsubsection{Two variable inequality}

    \subsubsection{Linear programming}

    \subsection{Challenging questions}

    \newpage

    \section{Quadratic functions}

    \subsection{Solving Quadratic Equations}

    \subsubsection{The square-root method}

    \subsubsection{General form and Completing the square method}

    \subsubsection{The quadratic formula}

    \subsection{Solvability of Quadratic Equations}

    \subsubsection{Discriminant: the factor affecting solvability}

    \subsubsection{Application of solvability}

    \subsection{Relation between coefficients and roots}

    \subsubsection{What does it mean by a root?}

    \subsubsection{The Vieta formula}

    \subsubsection{Forming quadratic equations with given roots}

    \subsection{Vertex form of a quadratic function}

    \subsubsection{Obtaining vertex form using Transformation}

    \subsubsection{Relation between vertex form and general form}

    \subsubsection{Opening direction}

    \subsubsection{The axis of symmetry}

    \subsubsection{The extremum of quadratic functions}

    \subsection{Quadratic inequalities}

    \subsubsection{Boolean algebra}

    \subsubsection{Solving quadratic inequalities}

    \subsection{Challenging questions}

    \newpage 

    \section{Polynomial functions}

    \subsection{Arithmetic rules for polynomials}

    \subsubsection{Addition, Subtraction and Multiplication}

    \subsubsection{Division}

    \subsection{Divisibility of polynomials}

    \subsubsection{Division algorithm}

    \subsubsection{Remainder Theorem}

    \subsubsection{Factor Theorem and General Vieta formula}

    \subsection{G.C.D. and L.C.M.}

    \subsubsection{Greatest common divisor}

    \subsubsection{Least common multiple}

    \subsection{Positional notation}

    \subsubsection{Binary, Octal, Decimal and Hexadecimal}

    \subsubsection{Conversion between different bases}

    \subsection{Challenging questions}

    \newpage

    \section{Exponential and Logarithmic functions}

    \subsection{Exponential functions}

    \subsubsection{Law of Exponential algebra}

    \subsubsection{Seeing surds as fractional exponents}

    \subsubsection{Solving equations using exponetiation method}

    \subsection{Logarithmic functions}

    \subsubsection{What is Logarithm?}

    \subsubsection{Law of Logarithmic algebra}

    \subsubsection{Solving equations using Logarithmic method}

    \subsection{Challenging questions}

    \newpage

    \section{Sequence as a function of natural numbers}

    \subsection{What is a sequence?}

    \subsection{Arithmetic sequence}

    \subsubsection{General form}

    \subsubsection{Summation of arithmetic sequence}

    \subsection{Geometric sequence}

    \subsubsection{General form}

    \subsubsection{Summation on Geometric sequence}

    \subsubsection{Infinite sum of Geometric sequence}

    \subsection{Additional content: Arithmetic-Geometric sequence}

    \subsection{Challenging questions}
\end{document}