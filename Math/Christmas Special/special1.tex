\documentclass[12pt]{article}
\usepackage{ctex}
\usepackage[english]{babel}
\usepackage{blindtext}
\usepackage{nameref}
\usepackage{fancyhdr}
\usepackage{amsmath,amssymb,amsthm}
\usepackage{graphicx,float}
\usepackage{physics}
\usepackage{pgfplots}
\usepackage[a4paper, total={6in, 9in}]{geometry}

\graphicspath{{../images/}}

\pagestyle{fancy}
\fancyhf{}
\fancyhf[HL]{Christmas Special}
\fancyhf[CF]{\thepage}

\newcommand{\innerprod}[2]{\langle{#1},{#2}\rangle}
\newcommand{\id}{\mathtt{id}}

\newtheorem*{definition}{Definition}
\newtheorem*{theorem}{Theorem}
\newtheorem*{corollary}{Corollary}
\newtheorem*{lemma}{Lemma}
\newtheorem*{proposition}{Proposition}
\newtheorem*{remark}{Remark}
\newtheorem*{claim}{Claim}
\newtheorem*{example}{Example}
\newtheorem*{axiom}{Axiom}

\begin{document}
    Let us think of a scene. Suppose the Santa Claus was coming to town but he was missing direction. By asking the Government he was given a map with four coordinates. They are city $A(3,6)$, city $B(-3,-6)$, city $C(2,-12)$ and city $D(8,0)$.

    You are going to help Santa Claus to figure out some important information about his route of distribution.

    \begin{enumerate}
        \item[(F.1)] Sketch the coordinates and the polygon $ABCD$ on a coordinate plane. You may label the plane by yourself.
        \item[(F.1)] Find the area of the polygon $ABCD$.
        \item[(F.2)] Find the mean of the coordinates, i.e. the mean value of x-coordinates and y-coordinates.
        \item[(F.3)] Find the length of the sides of the polygon $ABCD$.
    \end{enumerate}

    We then suppose the Santa Claus needs to distribute different cities with different weight of presents. For City A (Point A), it is 20kg; For City B (Point B), it is 30kg; For City C (Point C), it is 40kg; For City D (Point D), it is 50kg.

    \begin{enumerate}
        \item[(F.4)] Find the shortest route to reach every city.
        \item[(F.5)] It is known that the energy required for each route ($E$) varies partly with $m$ and partly with $L^2$, where $m$ is the weight of carrying presents and $L$ is the length of the route. If the energy carrying all presents from $A$ to $B$ and carrying half of the presents from $A$ to $D$ are $1300$ and $594$ respectively. Find the energy needed for carrying all presents from $A$ straightly to $C$.
        \item[(F.5)] Find the total number of routes to go through all cities. 
        \item[(F.5)] Hence, find the route requiring the least energy to deliver all presents.
    \end{enumerate}
\end{document}