\documentclass[12pt]{article}
\usepackage{ctex}
\usepackage[english]{babel}
\usepackage{blindtext}
\usepackage{nameref}
\usepackage{fancyhdr}
\usepackage{color,amsmath,amssymb,amsthm,physics}
\usepackage{graphicx,float}
\usepackage{physics}
\usepackage{pgfplots}
\usepackage[a4paper, total={7in, 9in}]{geometry}
\usepackage{multicol}
\usepackage{framed}
\usepackage{xcolor}

\graphicspath{ {../images/} }

\definecolor{shadecolor}{RGB}{220,220,220}

\pagestyle{fancy}
\fancyhf{}
\fancyhf[HL]{A discussion on straight line}
\fancyhf[HR]{\rightmark}
\fancyhf[CF]{\thepage}
\fancyhf[FL]{\copyright Mok Owen 2024}

\newcommand{\innerprod}[2]{\langle{#1},{#2}\rangle}
\newcommand{\id}{\mathtt{id}}
\newcommand{\cis}[1]{\mathrm{cis}({#1})}

\newtheorem{definition}{Definition}[section]
\newtheorem*{theorem}{Theorem}
\newtheorem*{corollary}{Corollary}
\newtheorem*{lemma}{Lemma}
\newtheorem*{proposition}{Proposition}
\newtheorem*{remark}{Remark}
\newtheorem*{claim}{Claim}
\newtheorem*{example}{Example}
\newtheorem*{axiom}{Axiom}

\newtheorem{exercise}{Essential Practice}[subsection]
\newenvironment{solution}{\begin{snugshade*} \underline{\textbf{Solution.}} \par}{\hfill \textit{\dots end of solution} \end{snugshade*}}
\renewenvironment{proof}[1][Proof]{\begin{snugshade*} \underline{\textit{{#1}.}}\\}{\hfill \qedsymbol \end{snugshade*}}


\begin{document}
    \begin{abstract}
        It is quite straight forward to think about straight line and it seems unnecessary to discuss what does it mean by a straight line, as what we could relate when the word `straight forward' appears, however, a straight line is not as trivial as a word phrase did in the study of Mathematics. I will introduce some interesting pure mathematical concepts in relation to straight line. It might sounds like a word game to realize such meaning and their importance to the discussion, but it worths defining such words to unsieze the mystery of Geometry.
    \end{abstract}

    \section{Topological Space}

    A \textbf{topological space} is a basic setup for discussion in pure mathematics. For this discussion I hope we all have included the knowledge about Mathematical sets; otherwise, it is too young to debate.

    \begin{definition}[Topology]
        A \textbf{topology} $\mathcal{T}_X$ on a set $X$ is a collection of subsets that satisfies the following properties:\begin{itemize}
            \item $\emptyset, X\in \mathcal{T}_X$;
            \item $\bigcup_{\alpha\in I} A_{\alpha}\in \mathcal{T}_X$ if $A_{\alpha}\in\mathcal{T}_X$ with arbitrary index set $I$;
            \item $\bigcap_{\alpha\in I} A_\alpha \in \mathcal{T}_X$ for $A_\alpha\in\mathcal{T}_X$ with a finite index set $I$.
        \end{itemize}
    \end{definition}

    A topology can be thought of as a structure presetting what sets are observable. The word `topology' is a composite word formed by `top' and `logic', what I would like to interpret as `a logic of topping sets up'. Therefore we have many works on sets. 

    If $A\in \mathcal{T}_X$ we shall call $A$ an \textbf{open set} of the topology; and if $B\subset X$ and $B^c\in\mathcal{T}_X$, we shall call $B$ a \textbf{closed set} of the topology. It is quite absurd to name the sets open in a topology as there is a similar term called open interval in usual discussion, which was puzzling when I first heard the term. We shall understand it in the following way: An \textbf{open set} of a topology is a set that is \textit{directly observable} through the topology, and a closed set is \textit{indirectly observable} through the topology, in which you can observe it by flipping it (e.g. taking its complement). If a set cannot be observed using any strategies, i.e. not related to the topology, we shall call it \textbf{neither open nor closed}; and if both the set and the complement of the set are able to be observed through the topology directly, we shall call it \textbf{both open and closed}.

    To be simple, it is like if the set is opened to you, then it is an open set, vice versa.

    \begin{definition}[Topological Space]
        A \textbf{topological space} is a pair $(X,\mathcal{T}_X)$ consisting of a set $X$ and a topology $\mathcal{T}_X$ on $X$.
    \end{definition}

    \begin{example}[Trivial topological space]
        Let $X$ be a non-empty set and define $\mathcal{T}_X=\{\emptyset, X\}$. This is a minimal topology on $X$.
    \end{example}

    \begin{proof}[Prove that trivial topology is indeed a topology]
        To prove the case we need to verify all requirements. It is clear that $\emptyset, X\in\mathcal{T}_X$ by definition. We also find $\emptyset\cap X=\emptyset\in\mathcal{T}_X$ and $\emptyset\cup X = X\in\mathcal{T}_X$. The requirement is then satisfied.
    \end{proof}

    The achievement here is we have built a basic space contains only a structure, describing the containment of sets and objects. Could we consider any lines here? Not yet. We shall have no idea on lines if we are not describing geometrical objects.

    \section{Metric Space}

    \section{Euclidean and Non-Euclidean Geometry}

    \section{Curves}

    \section{Curvature}

    \section{Straight line}
    
\end{document}