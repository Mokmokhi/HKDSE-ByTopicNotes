\documentclass[12pt]{article}
\usepackage{ctex}
\usepackage[english]{babel}
\usepackage{blindtext}
\usepackage{nameref}
\usepackage{fancyhdr}
\usepackage{color,amsmath,amssymb,amsthm,physics}
\usepackage{graphicx,float}
\usepackage{physics}
\usepackage{pgfplots}
\usepackage[a4paper, total={7in, 9in}]{geometry}
\usepackage{multicol}
\usepackage{framed}
\usepackage{xcolor}

\graphicspath{ {../images/} }

\definecolor{shadecolor}{RGB}{220,220,220}

\pagestyle{fancy}
\fancyhf{}
\fancyhf[HL]{A discussion on straight line}
\fancyhf[HR]{\rightmark}
\fancyhf[CF]{\thepage}
\fancyhf[FL]{\copyright Mok Owen 2024}

\newcommand{\innerprod}[2]{\langle{#1},{#2}\rangle}
\newcommand{\id}{\mathtt{id}}
\newcommand{\cis}[1]{\mathrm{cis}({#1})}

\newtheorem{definition}{Definition}[section]
\newtheorem*{theorem}{Theorem}
\newtheorem*{corollary}{Corollary}
\newtheorem*{lemma}{Lemma}
\newtheorem*{proposition}{Proposition}
\newtheorem*{remark}{Remark}
\newtheorem*{claim}{Claim}
\newtheorem*{example}{Example}
\newtheorem*{axiom}{Axiom}

\newtheorem{exercise}{Essential Practice}[subsection]
\newenvironment{solution}{\begin{snugshade*} \underline{\textbf{Solution.}} \par}{\hfill \textit{\dots end of solution} \end{snugshade*}}
\renewenvironment{proof}[1][Proof]{\begin{snugshade*} \underline{\textit{{#1}.}}\\}{\hfill \qedsymbol \end{snugshade*}}


\begin{document}
    \begin{abstract}
        It is quite straight forward to think about straight line and it seems unnecessary to discuss what does it mean by a straight line, as what we could relate when the word `straight forward' appears, however, a straight line is not as trivial as a word phrase did in the study of Mathematics. I will introduce some interesting pure mathematical concepts in relation to straight line. It might sounds like a word game to realize such meaning and their importance to the discussion, but it worths defining such words to unsieze the mystery of Geometry.
    \end{abstract}

    \section{Topological Space}

    A \textbf{topological space} is a basic setup for discussion in pure mathematics. For this discussion I hope we all have included the knowledge about Mathematical sets; otherwise, it is too young to debate.

    \begin{definition}[Topology]
        A \textbf{topology} $\mathcal{T}_X$ on a set $X$ is a collection of subsets that satisfies the following properties:\begin{itemize}
            \item $\emptyset, X\in \mathcal{T}_X$;
            \item $\bigcup_{\alpha\in I} A_{\alpha}\in \mathcal{T}_X$ if $A_{\alpha}\in\mathcal{T}_X$ with arbitrary index set $I$;
            \item $\bigcap_{\alpha\in I} A_\alpha \in \mathcal{T}_X$ for $A_\alpha\in\mathcal{T}_X$ with a finite index set $I$.
        \end{itemize}
    \end{definition}

    A topology can be thought of as a structure presetting what sets are observable. The word `topology' is a composite word formed by `top' and `logic', what I would like to interpret as `a logic of topping sets up'. Therefore we have many works on sets. 

    If $A\in \mathcal{T}_X$ we shall call $A$ an \textbf{open set} of the topology; and if $B\subset X$ and $B^c\in\mathcal{T}_X$, we shall call $B$ a \textbf{closed set} of the topology. It is quite absurd to name the sets open in a topology as there is a similar term called open interval in usual discussion, which was puzzling when I first heard the term. We shall understand it in the following way: An \textbf{open set} of a topology is a set that is \textit{directly observable} through the topology, and a closed set is \textit{indirectly observable} through the topology, in which you can observe it by flipping it (e.g. taking its complement). If a set cannot be observed using any strategies, i.e. not related to the topology, we shall call it \textbf{neither open nor closed}; and if both the set and the complement of the set are able to be observed through the topology directly, we shall call it \textbf{both open and closed}.

    To be simple, it is like if the set is opened to you, then it is an open set, vice versa.

    \begin{definition}[Topological Space]
        A \textbf{topological space} is a pair $(X,\mathcal{T}_X)$ consisting of a set $X$ and a topology $\mathcal{T}_X$ on $X$.
    \end{definition}

    \begin{example}[Trivial topological space]
        Let $X$ be a non-empty set and define $\mathcal{T}_X=\{\emptyset, X\}$. This is a minimal topology on $X$.
    \end{example}

    \begin{proof}[Prove that trivial topology is indeed a topology]
        To prove the case we need to verify all requirements. It is clear that $\emptyset, X\in\mathcal{T}_X$ by definition. We also find $\emptyset\cap X=\emptyset\in\mathcal{T}_X$ and $\emptyset\cup X = X\in\mathcal{T}_X$. The requirement is then satisfied.
    \end{proof}

    \begin{example}[Discrete topological space]
        Let $X$ be a non-empty finite set such that $X=\{x_1,x_2,\dots,x_N\}$ for some $N\in\mathbb{N}$. Define the power set $P(X)$ of $X$ to be a collection of all subsets of $X$. Then it is easy to verify that $P(X)$ is a topology on $X$, namely \textbf{discrete topology}, and the space $(X,P(X))$ is called a dicrete topological space.
    \end{example}

    \begin{proof}[Prove that $P(X)$ is a topology on $X$]
        Since $\emptyset\subset X$ and $X\subset X$, we have $\emptyset, X\in P(X)$. Let $A,B\subset X$, then by definition $A\cap B, A\cup B\subset X$. Hence, $A\cap B, A\cup B\in P(X)$. It remains to show $\bigcup_{\alpha\in I}A_{\alpha}\in P(X)$.

        The arbitrary union under $X$ is a consequence of Zorn's lemma. Let $\forall x_k\in X$ we can assign an increasing chain $C(x_k)$ such that for any $A,B\in C(x_k)$ the statement $A\subset B$ or $B\subset A$ is true, and $\forall A\in C(x_k)$ they are subsets of $X$. As long as that increasing chain is collected from $X$, \[\bigcup_{A\in C(x_k)}A\subset X\] and thus \[\bigcup_k \bigcup_{A\in C(x_k)}A\subset X.\] Since any subset $S$ of $X$ satisfies $S\in C(x_k)$ for some positive integer $k\leq N$, We have the arbitrary union $\bigcup_{\alpha\in I} A_{\alpha}$ is a member of $\bigcup_k C(x_k)$ and thus the arbitrary union is a subset of $X$.
    \end{proof}

    \begin{example}[Real standard topological space]
        The \textbf{standard topology} on $\mathbb{R}$ is the topology $T:=\{U\subset \mathbb{R}: \forall x\in U, \exists a,b\in\mathbb{R}, x\in (a,b)\}$. The pair $(\mathbb{R},T)$ is called \textbf{real standard topological space}.
    \end{example}

    \begin{example}[Complex standard topological space]
        Define $B_r^{\times}(x_0):=\{x\in\mathbb{C}:0< |x-x_0| < r\}$ for $r\in\mathbb{R}$, call it \textbf{punctured open r-disc centered at $z_0$}. The \textbf{standard topology} on $\mathbb{C}$ is defined by $T:=\{U\subset\mathbb{C}:\forall z\in U, \exists r>0, \exists z_0\in\mathbb{C}, z\in B_r^{\times}(z_0)\}$. The pair $(\mathbb{C},T)$ is called a \textbf{complex standard topological space}.
    \end{example}

    In the sense of $\mathbb{R}^n$, we usually define a \textbf{usual topology} to cope with standard topology. Such topology is defined similar to the one described in complex case, and call $B_r^{\times}(x_0)$ a \textit{punctured open ball with radius $r$ centered at $x_0$}. I will leave the exploration on it to interested readers.

    The achievement here is we have built a basic space contains only a structure, describing the containment of sets and objects. Could we consider any lines here? Not yet. We shall have no idea on lines if we are not describing geometrical objects.

    In order to proceed to geometrical objects, we need to define the terms \textbf{connectedness}, \textbf{compactness} and \textbf{continuity}.

    \begin{definition}[Connectedness]
        A topological space $(X,T)$ is \textbf{connected} if $X$ cannot be a union of any two open sets, i.e. $\forall A,B \in T$ such that $A\cap B=\emptyset$, $A\cup B\neq X$.
    \end{definition}

    The definition looks abstract nonsense, but bear in mind that our open sets are depends on the topology. Take the following examples for the determination.

    \begin{example}
        The real standard topological space is connected.
    \end{example}

    \begin{proof}
        The complete proof shall be left as an exercise. It is not hard to observe that any open interval $(a,b)\subset \mathbb{R}$ cannot be a disjoint union of two open sets. Hence the whole space.
    \end{proof}

    \begin{example}
        Let $T$ be the \textbf{lower limit topology} on $\mathbb{R}$, i.e. $T:=\{[a,b)\subset \mathbb{R}: a,b\in\mathbb{R}\}$. Then $(\mathbb{R},T)$ is disconnected.
    \end{example}

    \begin{proof}
        It is easy to see that $\mathbb{R}=\displaystyle \lim_{n\to \infty}\bigcup_{k=-n}^n [k,k+1)$ and we are done.
    \end{proof}

    \begin{definition}[Compactness]
        
    \end{definition}

    \section{Metric Space}

    \section{Euclidean and Non-Euclidean Geometry}

    \section{Curves}

    \section{Curvature}

    \section{Straight line}
    
\end{document}