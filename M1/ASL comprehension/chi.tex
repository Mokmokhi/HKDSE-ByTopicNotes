\documentclass[12pt]{article}
\usepackage{ctex}
\usepackage[english]{babel}
\usepackage{blindtext}
\usepackage{nameref}
\usepackage{fancyhdr}
\usepackage{amsmath,amssymb,amsthm}
\usepackage{graphicx,float}
\usepackage{physics}
\usepackage{pgfplots}
\usepackage[a4paper, total={7in, 9in}]{geometry}

\graphicspath{{../img/}}

\pagestyle{fancy}
\fancyhf{}
\fancyhf[HL]{暑期練習:挑戰題}
\fancyhf[CF]{\thepage}

\newcommand{\innerprod}[2]{\langle{#1},{#2}\rangle}
\newcommand{\id}{\mathtt{id}}

\newtheorem{definition}{定義}
\newtheorem*{theorem}{定理}
\newtheorem*{corollary}{衍理}
\newtheorem*{lemma}{引理}
\newtheorem*{proposition}{設理}
\newtheorem*{remark}{小記}
\newtheorem*{claim}{主張}
\newtheorem*{example}{例子}
\newtheorem*{axiom}{公設}
\renewenvironment*{proof}{\textit{證明.}}{\hfill$\qed$}

\newenvironment*{sol}{\par \textbf{解}.}{\hfill$\blacksquare$}

\begin{document}
    \tableofcontents

    \newpage

    \section{二項式定理}

    \begin{enumerate}
        \item \begin{enumerate}
            \item \begin{enumerate}
                \item 展開$(x+y+z)^2$。
                \item 求$(x+y+z)^4$的展開式中$x^3y$,$x^3z$,$xy^3$,$y^3z$,$xz^3$,$yz^3$的係數。
            \end{enumerate}
            \item 若從一個裝有紅色杯子、藍色杯子及綠色杯子的箱子中隨機抽取一個杯子,則抽到紅色杯子、藍色杯子及綠色杯子的概率分別爲$p,q$及$r$。若從中抽取4個杯子,每次抽取後均可把杯子放回箱子内,試以$p,q,r$表示\begin{enumerate}
                \item 抽到至少2個不同顏色的杯子的概率;
                \item 抽到剛好3個相同顔色的杯子的概率。
            \end{enumerate}
        \end{enumerate}
        \item 假設$(1+x)^{\frac{1}{2}}=1+ax+bx^2+cx^3+$更高次冪的項。\begin{enumerate}
            \item 藉考慮$(1+x)=[(1+x)^{\frac{1}{2}}]^2$,求$a$及$b$的值。
            \item 按$x$的升冪展開$e^{-2x}$至$x^3$的項。
            \item 由此,按$x$的升冪展開$\dfrac{(1+x)^{\frac{1}{2}}}{e^{2x}}$至$x^3$的項。
        \end{enumerate}
        \item 設$(1+ax)^8=\sum_{k=0}^{8}\lambda_k x^k$及$(b+x)^9=\sum_{k=0}^{9}\mu_k x^k$,其中$a,b$為實常數。已知$\lambda_2:\mu_7=7:4$及$\lambda_1+\mu_8+6=0$。求$a$的值。
    \end{enumerate}

    \section{指數函數與對數函數}
    \begin{enumerate}
        \item 按$x$的升冪展開$e^{e^x}$至$x^2$的項。
        \item 設$y=\dfrac{1-e^{4x}}{1+e^{8x}}$。\begin{enumerate}
            \item 求$\dfrac{dy}{dx}\bigg|_{x=0}$的值。
            \item 設$(z^2+1)e^{3z}=e^{\alpha+\beta x}$,其中$\alpha,\beta$為常數。\begin{enumerate}
                \item 試表$\ln(z^2+1)+3z$為$x$的綫性函數。
                \item 已知(b)(i)的函數圖像通過原點并且斜率為2。求$\alpha$和$\beta$的值。
                \item 利用(b)(ii)所得的$\alpha$和$\beta$的值,求$\dfrac{dy}{dz}\bigg|_{z=0}$。
            \end{enumerate}
        \end{enumerate}
    \end{enumerate}
\end{document}