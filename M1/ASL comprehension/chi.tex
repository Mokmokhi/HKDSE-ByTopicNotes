\documentclass[12pt]{article}
\usepackage{ctex}
\usepackage[english]{babel}
\usepackage{blindtext}
\usepackage{nameref}
\usepackage{fancyhdr}
\usepackage{amsmath,amssymb,amsthm}
\usepackage{graphicx,float}
\usepackage{physics}
\usepackage{pgfplots}
\usepackage[a4paper, total={7in, 9in}]{geometry}

\graphicspath{{../img/}}

\pagestyle{fancy}
\fancyhf{}
\fancyhf[HL]{暑期練習:挑戰題}
\fancyhf[CF]{\thepage}

\newcommand{\innerprod}[2]{\langle{#1},{#2}\rangle}
\newcommand{\id}{\mathtt{id}}

\newtheorem{definition}{定義}
\newtheorem*{theorem}{定理}
\newtheorem*{corollary}{衍理}
\newtheorem*{lemma}{引理}
\newtheorem*{proposition}{設理}
\newtheorem*{remark}{小記}
\newtheorem*{claim}{主張}
\newtheorem*{example}{例子}
\newtheorem*{axiom}{公設}
\renewenvironment*{proof}{\textit{證明.}}{\hfill$\qed$}

\newenvironment*{sol}{\par \textbf{解}.}{\hfill$\blacksquare$}

\begin{document}
    \tableofcontents

    \newpage

    \section{二項式定理}

    \begin{enumerate}
        \item \begin{enumerate}
            \item \begin{enumerate}
                \item 展開$(x+y+z)^2$。
                \item 求$(x+y+z)^4$的展開式中$x^3y$,$x^3z$,$xy^3$,$y^3z$,$xz^3$,$yz^3$的係數。
            \end{enumerate}
            \item 若從一個裝有紅色杯子、藍色杯子及綠色杯子的箱子中隨機抽取一個杯子,則抽到紅色杯子、藍色杯子及綠色杯子的概率分別爲$p,q$及$r$。若從中抽取4個杯子,每次抽取後均可把杯子放回箱子内,試以$p,q,r$表示\begin{enumerate}
                \item 抽到至少2個不同顏色的杯子的概率;
                \item 抽到剛好3個相同顔色的杯子的概率。
            \end{enumerate}
        \end{enumerate}
        \item 假設$(1+x)^{\frac{1}{2}}=1+ax+bx^2+cx^3+$更高次冪的項。\begin{enumerate}
            \item 藉考慮$(1+x)=[(1+x)^{\frac{1}{2}}]^2$,求$a$及$b$的值。
            \item 按$x$的升冪展開$e^{-2x}$至$x^3$的項。
            \item 由此,按$x$的升冪展開$\dfrac{(1+x)^{\frac{1}{2}}}{e^{2x}}$至$x^3$的項。
        \end{enumerate}
        \item 設$(1+ax)^8=\sum_{k=0}^{8}\lambda_k x^k$及$(b+x)^9=\sum_{k=0}^{9}\mu_k x^k$,其中$a,b$為實常數。已知$\lambda_2:\mu_7=7:4$及$\lambda_1+\mu_8+6=0$。求$a$的值。
    \end{enumerate}

    \section{指數函數與對數函數}
    \begin{enumerate}
        \item 按$x$的升冪展開$e^{e^x}$至$x^2$的項。
        \item 設$y=\dfrac{1-e^{4x}}{1+e^{8x}}$。\begin{enumerate}
            \item 求$\dfrac{dy}{dx}\bigg|_{x=0}$的值。
            \item 設$(z^2+1)e^{3z}=e^{\alpha+\beta x}$,其中$\alpha,\beta$為常數。\begin{enumerate}
                \item 試表$\ln(z^2+1)+3z$為$x$的綫性函數。
                \item 已知(b)(i)的函數圖像通過原點并且斜率為2。求$\alpha$和$\beta$的值。
                \item 利用(b)(ii)所得的$\alpha$和$\beta$的值,求$\dfrac{dy}{dz}\bigg|_{z=0}$。
            \end{enumerate}
        \end{enumerate}
        \item 某研究員正在研究某城市的人口增長與電力消耗。已知該城市的人口$P$可以下式模擬\[P=\frac{ke^{-\lambda t}}{t^2}, 0<t<6,\]其中$k$和$\lambda$為常數,$t$則為從研究開始算起所經歷的時間。\begin{enumerate}
            \item \begin{enumerate}
                \item 試表$\ln{P}+2\ln{t}$為$t$的綫性函數。
                \item 已知(a)(i)的函數圖像的水平截距與垂直截距分別爲$-1.15$及$2.3$,求$k$和$\lambda$的值,答案准確至最接近的整數。
                \item 由此,求該城市的人口的最少值,準確至最接近的百位。
            \end{enumerate}
            \item 該城市的年度電力消耗$E$(兆焦耳每年)可以下式模擬\[\frac{dE}{dt}=hte^{ht}-1.2e^{ht}+4.214, t\geq 0,\]其中$h$為非零常數,$t$則為從研究開始算起所經歷的時間。已知當$t=t_0$時,該城市的人口與年度電力消耗同時達到最少值,而$t=0$時,$E=1$。\begin{enumerate}
                \item 求$h$的值。
                \item 藉考慮$\dfrac{d}{dt}(te^{ht})$,求$\int te^{ht} dt$。
                \item 由此,求該城市在$t=t_0$時的年度電力消耗,準確至最接近的兆焦耳每年。
                \item 某環保運動在$t=t_0$後立即推行以減少年度電力消耗。新的年度電力消耗$F$(兆焦耳每年)可以下式模擬\[F=\frac{6}{1-5e^{rt}+3e^{2rt}}+2, t\geq t_0.\]\begin{enumerate}
                    \item 若新的年度電力消耗在$t=t_0$時與原本的年度電力消耗相同,求$r$的值。
                    \item 新的年度電力消耗會否在某個$t_0$以後的時間上升?
                \end{enumerate}
            \end{enumerate}
        \end{enumerate}
    \end{enumerate}

    \section{微積分}
    \begin{enumerate}
        \item 考慮曲綫$C_1:y=e^{2x}-e^4$及$C_2:e^{x+3}e^{x+1}$。求$C_1$及$C_2$的相交點。
        \item 求以下積分:\begin{enumerate}
            \item $\displaystyle\int_{1}^{3}\frac{t+2}{t^2+4t+11}dt$;
            \item $\displaystyle\int_{1}^{3}\frac{t^2+3t+9}{t^2+4t+11} dt$。
        \end{enumerate}
        \item 一個大缸中的水的體積的變率可以表為\[f(t)=\frac{500}{(t+2)^2e^t},\]其中$t(\geq 0)$為以分鐘量度的時間。\begin{enumerate}
            \item 利用5個區間的梯形法則,估算從$t=1$至$t=11$期間流進缸中的水的總量。
            \item 求$\dfrac{d^2 f(x)}{dt^2}$。
            \item 判斷(a)題的估算為過高估算還是過低估算。
        \end{enumerate}
        \item 某單位正方形標靶的四個頂點分別位於$(0,0), (0,1), (1,0), (1,1)$. 該標靶以曲綫$y=\sqrt{x}$及$y=x^3$劃分爲三個區域,從左上而右下分別爲II區、I區和III區。若飛鏢砸中I區、II區或III區,則分別可得10分、20分和30分。\begin{enumerate}
            \item 求該三區的面積。
            \item 假設某小孩隨機扔出兩道飛鏢而且均命中標靶,求小孩獲得總共40分的概率。
        \end{enumerate}
        \item \begin{enumerate}
            \item 按$x$的遞升序展開$e^{\frac{-x^2}{2}}$至$x^6$的項。
            \item 由此,估算$\displaystyle \int_{0}^{1} e^{\frac{-x^2}{2}}dx$的值。
            \item 已知標準常態分佈在$z=0$至$z=a$之間的曲綫下面積為$\displaystyle \int_{0}^{a} \frac{1}{\sqrt{2\pi}}e^{\frac{-z^2}{2}} dz$。利用(a)的結果與常態分佈圖表估算$\pi$的值,準確至3位小數。
        \end{enumerate}
        \item 某店鋪經理欲推行計劃A或B以提高利潤。設$R$和$Q$(以百萬爲單位)分別爲計劃A和B推行後的纍計每周盈餘。已知\[\frac{dR}{dt}=\begin{cases}
            \ln(2t+1)&0\leq t\leq 6\\
            0&t>6
        \end{cases},\]及\[\frac{dQ}{dt}=\begin{cases}
            45t(1-t)+\frac{1.58}{t+1}&0\leq t\leq 1\\
            \frac{30e^{-t}}{(3+2e^{-t})^2}&t>1
        \end{cases},\]其中$t$為計劃推行後所經過的周數。\begin{enumerate}
            \item 假設執行了A計劃。\begin{enumerate}
                \item 利用6區間的梯形法則,估算計劃開始後首6周的盈餘。
                \item 請問(a)(i)的估算屬於過低估算還是過高估算?試加以解釋。
            \end{enumerate}
            \item 假設執行了B計劃。\begin{enumerate}
                \item 求計劃開始後首周的盈餘。
                \item 藉代入$u=3+2e^{-t}$,或其他方法,求計劃執行後首$n$周的盈餘,其中$n>1$。答案以$n$表示。
            \end{enumerate}
            \item 哪個計劃的長期盈餘更多?試加以解釋。
        \end{enumerate}
    \end{enumerate}
\end{document}