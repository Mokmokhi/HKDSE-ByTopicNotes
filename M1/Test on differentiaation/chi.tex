\documentclass[12pt]{article}
\usepackage{ctex}
\usepackage[english]{babel}
\usepackage{blindtext}
\usepackage{nameref}
\usepackage{fancyhdr}
\usepackage{amsmath,amssymb,amsthm}
\usepackage{graphicx,float}
\usepackage{physics}
\usepackage{pgfplots}
\usepackage[a4paper, total={7in, 9in}]{geometry}

\graphicspath{{../img/}}

\pagestyle{fancy}
\fancyhf{}
\fancyhf[HL]{測驗2:導數及其應用}
\fancyhf[CF]{\thepage}

\newcommand{\innerprod}[2]{\langle{#1},{#2}\rangle}
\newcommand{\id}{\mathtt{id}}

\newtheorem{definition}{定義}
\newtheorem*{theorem}{定理}
\newtheorem*{corollary}{衍理}
\newtheorem*{lemma}{引理}
\newtheorem*{proposition}{設理}
\newtheorem*{remark}{小記}
\newtheorem*{claim}{主張}
\newtheorem*{example}{例子}
\newtheorem*{axiom}{公設}
\renewenvironment*{proof}{\textit{證明.}}{\hfill$\qed$}

\newenvironment*{sol}{\par \textbf{解}.}{\hfill$\blacksquare$}

\begin{document}
    \thispagestyle{empty}

    \centering 

    \section*{測驗2\\數學延伸單元\\單元1 (微積分與統計學)\\試題-答題簿}

    限時: 2 小時

    姓名:\hrulefill \hfill 得分:\hrulefill/51+5

    學校:\hrulefill

    \raggedright

    \subsection*{規則}

    \begin{enumerate}
        \item 此試卷必須使用中文回答。
        \item 除特別指明外,需詳細列出所有算式。
        \item 除特別指明外,數值答案必須用真確值表示。
        \item 本試卷只作\textbf{内部使用}。
        \item 所有試題取自AL/CE/DSE歷届試題,來源: https://www.dse.life/ppindex/m2/
    \end{enumerate}
    \newpage
    \thispagestyle{empty}
    \vspace*{\fill}
    \begin{center}
        -此爲空白頁-
    \end{center}
    \vspace*{\fill}
    \newpage
    \setcounter{page}{1}
    \begin{enumerate}
        \item (12分)已知$t=y^3+1$及$e^t=x^{x^2+1}$\begin{enumerate}
            \item 求$\dfrac{dt}{dy}$。\hfill(1分)
            \item 從以$x$表$t$,求$\dfrac{dt}{dx}$。\hfill(2分)
            \item 以$x$及$y$表$\dfrac{dy}{dx}$。\hfill(2分)
            \item 求當$x=e$時$y$的值,並求出$\dfrac{dy}{dx}\bigg|_{x=e}$的值。\hfill(2分)
            \item 求$x=e$時以上述算式在$xy$平面繪出的曲綫的切綫的$x$軸截距及$y$軸截距。\hfill(3分)
            \item 由此,求(e)所述的切綫與$x$軸及$y$軸包裹區域的面積。\hfill(2分)
        \end{enumerate}

        \hrulefill

        \hrulefill

        \hrulefill

        \hrulefill

        \hrulefill

        \hrulefill

        \hrulefill

        \hrulefill

        \hrulefill

        \hrulefill

        \hrulefill

        \hrulefill

        \hrulefill

        \hrulefill

        \hrulefill

        \hrulefill

        \hrulefill

        \hrulefill

        \hrulefill

        \hrulefill

        \hrulefill

        \hrulefill

        \hrulefill

        \hrulefill

        \hrulefill

        \hrulefill

        \hrulefill

        \hrulefill

        \hrulefill

        \hrulefill

        \hrulefill

        \hrulefill

        \hrulefill

        \hrulefill

        \hrulefill

        \hrulefill

        \hrulefill

        \hrulefill

        \hrulefill

        \hrulefill

        \hrulefill

        \hrulefill

        \hrulefill

        \hrulefill

        \hrulefill

        \hrulefill

        \hrulefill

        \item (15分)某病菌在某時間點$t$(從16/4/2010早上9時起計算的天數,可正可負)的數量$p(t)$可以用下式估算$$p(t)=\frac{a}{b+e^{-t}}+c, -\infty<t<\infty$$其中$a,b,c$為正常數。定義\textit{原始數量}為無限多天之前的病菌數量,以及\textit{終極數量}為無限多天後的病菌數量。\begin{enumerate}
            \item 試以$a,b,c$表下列各數:\begin{enumerate}
                \item $p(t)$的增長速度最快之時;
                \item 原始數量;
                \item 終極數量。
            \end{enumerate}\hfill(5分)
            \item 某科學家通過繪畫$\ln[p(t)-c]$對$\ln(b+e^{-t})$的圖像來研究病菌數量,並發現圖像的縱軸截距為$\ln 8000$。如果16/4/2010早上9時的病菌數量及每天增長速度分別爲6000及2000,求$a,b,c$的值。\hfill(3分)
            \item 另一名科學家認爲當病菌的增長速度達到峰值時,病菌數量將等於原始數量及終極數量的平均值。你同意嗎?試解釋你的答案。\hfill(2分)
            \item 透過以$a,b,c$及$p(t)$表$e^{-t}$,以$\dfrac{-b}{a}[p(t)-\alpha][p(t)-\beta]$的形式表$p'(t)$,其中$\alpha<\beta$。由此以$a,b,c$表$\alpha$及$\beta$。描繪於$\alpha<p(t)<\beta$時$p'(t)$對$p(t)$的圖像並驗證(c)題的答案。\hfill(5分)
        \end{enumerate}

        \hrulefill

        \hrulefill

        \hrulefill

        \hrulefill

        \hrulefill

        \hrulefill

        \hrulefill

        \hrulefill

        \hrulefill

        \hrulefill

        \hrulefill

        \hrulefill

        \hrulefill

        \hrulefill

        \hrulefill

        \hrulefill

        \hrulefill

        \hrulefill

        \hrulefill

        \hrulefill

        \hrulefill

        \hrulefill

        \hrulefill

        \hrulefill

        \hrulefill

        \hrulefill

        \hrulefill

        \hrulefill

        \hrulefill

        \hrulefill

        \hrulefill

        \hrulefill

        \hrulefill

        \hrulefill

        \hrulefill

        \hrulefill

        \hrulefill

        \hrulefill

        \hrulefill

        \hrulefill

        \hrulefill

        \hrulefill

        \hrulefill

        \hrulefill

        \hrulefill

        \hrulefill

        \hrulefill

        \hrulefill

        \hrulefill

        \hrulefill

        \hrulefill

        \hrulefill

        \hrulefill

        \hrulefill

        \hrulefill

        \hrulefill

        \hrulefill

        \hrulefill

        \hrulefill

        \hrulefill

        \hrulefill

        \hrulefill

        \hrulefill

        \hrulefill

        \hrulefill

        \hrulefill

        \hrulefill

        \hrulefill

        \hrulefill

        \hrulefill

        \hrulefill

        \hrulefill

        \hrulefill

        \hrulefill

        \hrulefill

        \hrulefill

        \hrulefill

        \hrulefill

        \hrulefill

        \hrulefill

        \hrulefill

        \hrulefill

        \hrulefill

        \hrulefill

        \hrulefill

        \hrulefill

        \hrulefill

        \hrulefill

        \hrulefill

        \hrulefill

        \hrulefill

        \hrulefill

        \hrulefill

        \hrulefill

        \hrulefill

        \hrulefill

        \item (6分)某倒立正圓錐形容器的底半徑為15 cm且高為20 cm。該容器在初時裝滿水。假設容器内的水正以$2\pi$ cm$^3$/s的速度流走。設$h$ cm為容器内餘下的水的深度,$r$ cm為水面的半徑(如圖),$V$ cm$^3$為水的體積,而$A$ cm$^2$為容器浸濕部分的面積。\begin{enumerate}
            \item 試\textbf{只以}$r$表$V$及$A$。\hfill(2分)
            \item 當$r=3$時,\begin{enumerate}
                \item 求半徑的變率;\hfill(2分)
                \item 求容器浸濕部分的面積的變率。\hfill(2分)
            \end{enumerate}
        \end{enumerate}

        \hrulefill

        \hrulefill

        \hrulefill

        \hrulefill

        \hrulefill

        \hrulefill

        \hrulefill

        \hrulefill

        \hrulefill

        \hrulefill

        \hrulefill

        \hrulefill

        \hrulefill

        \hrulefill

        \hrulefill

        \hrulefill

        \hrulefill

        \hrulefill

        \hrulefill

        \hrulefill

        \hrulefill

        \hrulefill

        \hrulefill

        \hrulefill

        \hrulefill

        \hrulefill

        \hrulefill

        \hrulefill

        \hrulefill

        \hrulefill

        \hrulefill

        \hrulefill

        \hrulefill

        \hrulefill

        \hrulefill

        \hrulefill

        \hrulefill

        \hrulefill

        \hrulefill

        \hrulefill

        \hrulefill

        \hrulefill

        \hrulefill

        \hrulefill

        \hrulefill

        \hrulefill

        \hrulefill

        \hrulefill

        \hrulefill

        \item (8分)計算下列不定積分問題。\begin{enumerate}
            \item $\displaystyle \int 3x^7+4\sqrt{x}+x^{-1}+2 dx$\hfill(2分)
            \item $\displaystyle \int \frac{2x^3+4x^2+6x+3}{x^2+x+1}dx$\hfill(2分)
            \item 已知某曲綫在$(x,y)$的斜率為$\dfrac{dy}{dx}=xe^x$,且$y$截距為0。考慮$\dfrac{d}{dx}(xe^x)$的結果,求曲綫的$x$截距。\hfill(4分)
        \end{enumerate}

        \hrulefill

        \hrulefill

        \hrulefill

        \hrulefill

        \hrulefill

        \hrulefill

        \hrulefill

        \hrulefill

        \hrulefill

        \hrulefill

        \hrulefill

        \hrulefill

        \hrulefill

        \hrulefill

        \hrulefill

        \hrulefill

        \hrulefill

        \hrulefill

        \hrulefill

        \hrulefill

        \hrulefill

        \hrulefill

        \hrulefill

        \hrulefill

        \hrulefill

        \hrulefill

        \hrulefill

        \hrulefill

        \hrulefill

        \hrulefill

        \hrulefill

        \hrulefill

        \hrulefill

        \hrulefill

        \hrulefill

        \hrulefill

        \hrulefill

        \hrulefill

        \hrulefill

        \hrulefill

        \hrulefill

        \hrulefill

        \hrulefill

        \hrulefill

        \hrulefill

        \hrulefill

        \hrulefill

        \hrulefill

        \hrulefill
        \item (10分)按要求回答下列定積分問題。\begin{enumerate}
            \item 利用梯形法則,$n=6$的情況,估算$\displaystyle \int_{1/7}^1 \ln{x} dx$的值,並判斷估算值為過高估算或過低估算。\hfill(4分)
            \item 求$\displaystyle y=\frac{1}{x\ln{x}}$的圖像從$x=e$至$x=e^4$的曲綫下面積。\hfill(2分)
            \item \begin{enumerate}
                \item 證明對於任意實數$a$,$\displaystyle\int_{-a}^{a}\frac{e^{x}}{e^{x}+1}dx=\int_{-a}^{a}\frac{1}{e^{x}+1}dx$.
                \item 由此,求$\displaystyle\int_{-a}^{a}\frac{e^{x}}{e^{x}+1}dx$及$\displaystyle\int_{-a}^{a}\frac{e^{x}-1}{e^{x}+1}dx$的值。
            \end{enumerate}\hfill(4分)
        \end{enumerate}

        \hrulefill

        \hrulefill

        \hrulefill

        \hrulefill

        \hrulefill

        \hrulefill

        \hrulefill

        \hrulefill

        \hrulefill

        \hrulefill

        \hrulefill

        \hrulefill

        \hrulefill

        \hrulefill

        \hrulefill

        \hrulefill

        \hrulefill

        \hrulefill

        \hrulefill

        \hrulefill

        \hrulefill

        \hrulefill

        \hrulefill

        \hrulefill

        \hrulefill

        \hrulefill

        \hrulefill

        \hrulefill

        \hrulefill

        \hrulefill

        \hrulefill

        \hrulefill

        \hrulefill

        \hrulefill

        \hrulefill

        \hrulefill

        \hrulefill

        \hrulefill

        \hrulefill

        \hrulefill

        \hrulefill

        \hrulefill

        \hrulefill

        \hrulefill

        \hrulefill

        \hrulefill

        \hrulefill

        \item[挑戰題I.] (3分)設$f(x)=ax^3+bx^2+cx+d$,其中$a,b,c,d$為常數且$a\neq 0$。試以$a,b,c$表$f(x)$\textbf{沒有極值}的條件。

        \hrulefill

        \hrulefill

        \hrulefill

        \hrulefill

        \hrulefill

        \hrulefill

        \hrulefill

        \hrulefill

        \hrulefill

        \hrulefill

        \hrulefill

        \hrulefill

        \hrulefill

        \hrulefill

        \hrulefill

        \hrulefill

        \hrulefill

        \hrulefill

        \hrulefill

        \hrulefill

        \hrulefill

        \hrulefill

        \hrulefill

        \hrulefill

        \hrulefill

        \hrulefill
        \item[挑戰題II.] (2分)設$f(x)=e^{-x^2}$。求$f(x)$及$-f(x)$所包裹的區域内最大圓形的面積。

        \hrulefill

        \hrulefill

        \hrulefill

        \hrulefill

        \hrulefill

        \hrulefill

        \hrulefill

        \hrulefill

        \hrulefill

        \hrulefill

        \hrulefill

        \hrulefill

        \hrulefill

        \hrulefill

        \hrulefill

        \hrulefill

        \hrulefill

        \hrulefill

        \hrulefill

        \hrulefill

        \hrulefill

        \hrulefill

        \hrulefill

        \hrulefill

        \hrulefill

        \hrulefill

        \hrulefill

        \hrulefill

        \hrulefill

        \hrulefill

        \hrulefill

        \hrulefill

        \hrulefill

        \hrulefill

        \hrulefill

        \hrulefill

        \hrulefill

        \hrulefill

        \hrulefill

        \hrulefill

        \hrulefill

        \hrulefill

        \hrulefill

        \hrulefill

        \hrulefill

        \hrulefill

        \hrulefill

        \hrulefill

        \hrulefill

        \hrulefill

        \hrulefill

        \hrulefill

        \hrulefill

        \hrulefill

        \item[挑戰題III.] (1分)已知$\displaystyle\sum_{k=0}^{n}\int_0^n C_k^n x^k dx=\frac{n}{n+1}$,求$n$的所有可能值。

        \hrulefill

        \hrulefill

        \hrulefill

        \hrulefill

        \hrulefill

        \hrulefill

        \hrulefill

        \hrulefill

        \hrulefill

        \hrulefill

        \hrulefill

        \hrulefill

        \hrulefill

        \hrulefill

        \hrulefill

        \hrulefill

        \hrulefill

        \hrulefill

        \hrulefill

        \hrulefill

        \hrulefill

        \hrulefill

        \hrulefill

        \hrulefill

        \hrulefill

        \hrulefill

        \hrulefill

        \hrulefill

        \hrulefill

        \hrulefill

        \hrulefill

        \hrulefill

        \hrulefill

        \hrulefill

        \hrulefill

        \hrulefill

        \hrulefill

        \hrulefill

        \hrulefill

        \hrulefill

        \hrulefill

        \hrulefill

        \hrulefill

        \hrulefill

        \hrulefill

        \hrulefill

        \hrulefill

        \hrulefill

        \hrulefill

        \hrulefill

        \hrulefill

        \hrulefill

        \hrulefill

        \begin{center}
            -全卷完-
        \end{center}
    \end{enumerate}
\end{document}