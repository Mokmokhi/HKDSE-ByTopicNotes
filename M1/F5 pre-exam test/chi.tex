\documentclass[12pt]{article}
\usepackage{ctex}
\usepackage[english]{babel}
\usepackage{blindtext}
\usepackage{nameref}
\usepackage{fancyhdr}
\usepackage{amsmath,amssymb,amsthm}
\usepackage{graphicx,float}
\usepackage{physics}
\usepackage{pgfplots}
\usepackage[a4paper, total={7in, 9in}]{geometry}

\graphicspath{{../img/}}

\pagestyle{fancy}
\fancyhf{}
\fancyhf[HL]{測驗4:溫習測驗}
\fancyhf[CF]{\thepage}

\newcommand{\innerprod}[2]{\langle{#1},{#2}\rangle}
\newcommand{\id}{\mathtt{id}}

\newtheorem{definition}{定義}
\newtheorem*{theorem}{定理}
\newtheorem*{corollary}{衍理}
\newtheorem*{lemma}{引理}
\newtheorem*{proposition}{設理}
\newtheorem*{remark}{小記}
\newtheorem*{claim}{主張}
\newtheorem*{example}{例子}
\newtheorem*{axiom}{公設}
\renewenvironment*{proof}{\textit{證明.}}{\hfill$\qed$}

\newenvironment*{sol}{\par \textbf{解}.}{\hfill$\blacksquare$}

\begin{document}
    \begin{enumerate}
        \item 設$\dfrac{(2x-a)^6}{e^{kx}}$的$x$變量展開式中$x^2$及$x$的係數分別爲$38$及$10$,求$a$及$k$的值。
        \item 假設$X\sim Po(r)$,其中$r>0$為實變數。\begin{enumerate}
            \item 以$r$表示$P(X=x)$.
            \item 求$\dfrac{dP}{dr}$及$\dfrac{d^2P}{dr^2}$.
            \item 若以梯形法則估算$\int_{0}^{t}P(X=x) dr$的值,其結果為過高估算還是過低估算?試解釋。
        \end{enumerate}
        \item 已知在一場射箭比賽中,A選手在$\ell$米射程的命中率($10\leq \ell \leq 50$)如下\begin{center}
            \begin{tabular}{|c||c|c|c|c|c|c|c|}
                \hline
                命中得分區&靶外(0分)&5分&6分&7分&8分&9分&10分\\
                \hline
                過往概率&$0.001\ell$&0.1&$3p$&0.4&$p$&$0.00005(1000-2\ell)$&$0.05-0.0009\ell$\\
                \hline
            \end{tabular}
        \end{center}
        而B選手的命中率從靶外到10分區呈二項分佈。現記A選手的命中概率分佈為$Z_A$,B選手的命中概率分佈為$Z_B$。\begin{enumerate}
            \item 求$p$的值。
            \item 以$\ell$表示$E[Z_A]$及$E[Z_B]$的值。
            \item 根據香港射箭總會本地賽事守則(戶外定距靶)2017年11月修訂版内容1.13新秀組計分賽的描述,每個選手需要在25米及18米各射程射出36箭以完成比賽,得分較高者獲勝。\begin{enumerate}
                \item 求選手A射出的箭全部命中靶面的概率。
                \item 求選手B的箭無法全部命中靶面的概率。
                \item 已知選手A及B在經歷70箭後暫時平分,而兩人均未嘗射出靶外。兩人現餘下最後兩回18米射程的機會,并且按照兩人的狀態而言,可知選手A在餘下兩箭不會射出靶外,而選手B至少會失手一箭。求A獲勝的概率。
            \end{enumerate}
        \end{enumerate}
        \item 假設某人的作畫速度為平均每小時4頁原稿,其作畫頁數跟從泊松分佈。記連續整點(例如12:00-13:00、14:00-15:00)為一個時段。該人每個工作天作畫4個時段。\begin{enumerate}
            \item 求該人於一個時段内作出多於6頁原稿的概率。
            \item 求該人於一個工作天内最多僅有一個時段作出多於6頁原稿的概率。
            \item 若該人能於一個時段内作出多於6頁原稿,便稱之爲心流時段。若該人能在一個工作天内達成兩個或以上心流時段,便稱之爲高效率工作天,否則為一般工作天。目前距離截稿日尚餘4個工作天。\begin{enumerate}
                \item 求這四個工作天中至少有一天為高效率工作天的概率。
                \item 已知四天中至少有一天為高效率工作天,求四天均爲高效率工作天的概率。
                \item 求一般工作天的產出期望值,由此求高效工作天的產出期望值。
                \item 求四天中有一天為高效工作天時,該四天的產出期望值。
            \end{enumerate}
        \end{enumerate}
        \item \begin{enumerate}
            \item 求$\dfrac{d}{dx}(x^2e^x)$及$\dfrac{d}{dx}(xe^x)$。
            \item 由此,求$\displaystyle \int x^2 e^x dx$。
        \end{enumerate}
    \end{enumerate}
\end{document}