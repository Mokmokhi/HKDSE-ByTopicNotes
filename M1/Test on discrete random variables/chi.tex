\documentclass[12pt]{article}
\usepackage{ctex}
\usepackage[english]{babel}
\usepackage{blindtext}
\usepackage{nameref}
\usepackage{fancyhdr}
\usepackage{amsmath,amssymb,amsthm}
\usepackage{graphicx,float}
\usepackage{physics}
\usepackage{pgfplots}
\usepackage[a4paper, total={7in, 9in}]{geometry}

\graphicspath{{../img/}}

\pagestyle{fancy}
\fancyhf{}
\fancyhf[HL]{測驗}
\fancyhf[CF]{\thepage}



\begin{document}
    根據某非正式統計,任何一個港鐵站於早高峰時段(6:00A.M.-8:00A.M.)可湧現平均每分鐘每個閘門增長6人的人流。港鐵公司安排以固定每三分鐘一班次運載乘客,每班列車有8卡,每卡車有5個閘門。假設列車正從車廠開往首站,需時三分鐘,且車上空無一人。
    \begin{enumerate}
        \item 求首站的一個閘門於列車到站時聚集多於6人的概率。
        \item 求首站的一卡車至少兩個閘門聚集多於6人的概率。
        \item 求首站各卡列車都有至少兩個閘門聚集多於6人的概率。
        \item 已知某沿綫列車途徑共13個車站(包括首站與尾站),每站相距三分鐘車程,求\begin{enumerate}
            \item 該綫於早高峰期間上車人數的期望值及方差;
            \item 其中7站各卡列車都有至少兩個閘門聚集多於6人的概率。
        \end{enumerate}
    \end{enumerate}
\end{document}