\documentclass[12pt]{article}
\usepackage{ctex}
\usepackage[english]{babel}
\usepackage{blindtext}
\usepackage{nameref}
\usepackage{fancyhdr}
\usepackage{amsmath,amssymb,amsthm}
\usepackage{graphicx,float}
\usepackage{physics}
\usepackage{pgfplots}
\usepackage[a4paper, total={7in, 9in}]{geometry}

\graphicspath{{../img/}}

\pagestyle{fancy}
\fancyhf{}
\fancyhf[HL]{測驗4:溫習測驗}
\fancyhf[CF]{\thepage}

\begin{document}
    \begin{enumerate}
        \item 考慮一個隨機樣本集合$X_n$,其概率算法如下:$$P(X_n=x)=C^n_x p^x (1-p)^x$$ 其中$0<p<1$為實數,$n>0$為整數。\begin{enumerate}
            \item 證明$X_n$為概率空間,即證明$\displaystyle\sum_{k=0}^{n}P(X=x)=1$。
            \item 設$E(X_n)$為$X_n$的期望值。已知期望值的算法為$\displaystyle\sum_{x=0}^{n}[x\cdot P(X=x)]$。\begin{enumerate}
                \item 證明對於任意整數$n$及$k<n$,$$k\cdot C_k^n \equiv n\cdot C_{k-1}^{n-1}$$
                \item 化簡$$x\cdot P(X=x)$$
                \item 由此,以$p$及$n$表$E(X_n)$。
            \end{enumerate}
            \item 已知$A=\{x\, |\, 0\leq x \leq 2\}$及$B=\{x\, |\, x\textrm{為0至5以内的單數}\}$為$X_5$内的事件。設$A'$及$B'$分別爲對應的互補事件。\begin{enumerate}
                \item $A$及$B$是否互相獨立?
                \item 求$P(A|B)$的值。
                \item 求$P(A'|B')$的值。
            \end{enumerate}
        \end{enumerate}
        \item 考慮一個隨機樣本集合$X_{\lambda}$,其概率算法如下:$$P(X_{\lambda}=x)=\frac{e^{-\lambda} \lambda^x}{x!}$$其中$\lambda>0$為實數。\begin{enumerate}
            \item 證明$X_\lambda$為概率空間,即證明$\displaystyle\sum_{x=0}^{\infty}P(X_{\lambda}=x)=1$。
            \item \begin{enumerate}
                \item 求$\displaystyle \derivative{\lambda}P(X_{\lambda}=x)$。
                \item 設$\Delta x = P(X_{\lambda}=x) - P(X_{\lambda}=x-1)$。求$$\int_{0}^{1} \Delta x d\lambda$$
            \end{enumerate}
            \item 若$x$為實數,則$P(X_{\lambda}=x)$屬於連續函數。如使用梯形法則計算$\displaystyle \int_1^2 P(X_{\lambda}=x) dx$,問計算結果屬於過高估算還是過低估算?試加以解釋。
        \end{enumerate}
        \item 設某車輛的行駛速度$v(t)$在不同時間點$t>0$可以用以下函式估算:$$v(t)=\frac{kt}{e^{at}+4}$$ 其中$k$和$a$為常數。已知在行車記錄儀内有以下不同時間的數據:\begin{center}
            \begin{tabular}{|c|c|c|c|c|c|}
                \hline
                $t$&5&10&15&20&25\\
                \hline
                $v(t)$&30.90&6.56&0.83&0.09&0.01\\
                \hline
            \end{tabular}
        \end{center}\begin{enumerate}
            \item 繪畫$\displaystyle\ln(\frac{kt}{v(t)}-4)$對$t$的圖像,並求出$a$和$k$的值。
            \item 求$v'(t)$及$v''(t)$,並求$v(t)$的極大值。
            \item 設$L$為$y=v(t)$的圖像在$x=0$的切綫。求$y=v(t)$與$L$所包裹的範圍的面積。
            \item $v(t)$會否小於0.0001?試加以解釋。
        \end{enumerate}
    \end{enumerate}
\end{document}