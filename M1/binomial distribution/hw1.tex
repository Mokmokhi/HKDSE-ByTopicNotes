\documentclass[12pt]{article}
\usepackage{ctex}
\usepackage[english]{babel}
\usepackage{blindtext}
\usepackage{nameref}
\usepackage{fancyhdr}
\usepackage{amsmath,amssymb,amsthm}
\usepackage{graphicx,float}
\usepackage{physics}
\usepackage{pgfplots}
\usepackage[a4paper, total={6in, 9in}]{geometry}

\graphicspath{{../image/}}

\pagestyle{fancy}
\fancyhf{}
\fancyhf[HL]{二項分佈:功課}
\fancyhf[CF]{\thepage}

\newcommand{\innerprod}[2]{\langle{#1},{#2}\rangle}
\newcommand{\id}{\mathtt{id}}

\newtheorem{definition}{定義}
\newtheorem*{theorem}{定理}
\newtheorem*{corollary}{衍理}
\newtheorem*{lemma}{引理}
\newtheorem*{proposition}{設理}
\newtheorem*{remark}{小記}
\newtheorem*{claim}{主張}
\newtheorem*{example}{例子}
\newtheorem*{axiom}{公設}
\renewenvironment*{proof}{\textit{證明.}}{\hfill$\qed$}

\newenvironment*{sol}{\par \textbf{解}.}{\hfill$\blacksquare$}

\begin{document}
    已知HKDSE數學科卷二共有題目45道,每道題目均爲4選1的選擇題。下列三道題目均以此爲前提。\begin{enumerate}
        \item 假設A君是一個學渣,每次考試都是靠純運氣的。假設他在每一題的答對的概率都是\(\dfrac{1}{4}\)。\begin{enumerate}
            \item 求A君在數學科卷二的期望值以及分數分佈方差。
            \item 已知A君答對了至少1道題,求A君答對23道題的概率。
        \end{enumerate}
        \item 假設B君是一個勤奮書生,有70\%的概率發奮圖强。若他發奮圖强,他有能力排除大部分的錯誤答案,使每一道題答對的概率變爲\(\dfrac{1}{2}\)。若他沒有發奮圖强,則和A君一樣,僅有$\dfrac{1}{4}$的概率答對。\begin{enumerate}
            \item 求B君在數學科卷二的期望值以及分數分佈方差。
            \item 已知B君沒拿滿分,求他答對40道題的概率。
        \end{enumerate}
        \item 假設C君是一名學神,從不溫習。在他看來,45道題中,頭15道比較簡單,有90\%的概率想得到解題思路,必然答對,否則仍能去掉必然錯誤的答案,有$\dfrac{1}{2}$的概率答對;中間15道題比較一般,有70\%的概率想得到解題思路,必然答對,否則仍能去掉必然錯誤的答案,有$\dfrac{1}{3}$的概率答對;最後15道是他的樂子,僅有50\%的概率能解成有$\dfrac{1}{2}$的概率答對;否則僅有$\dfrac{1}{4}$的概率答對。\begin{enumerate}
            \item 求C君的期望值。
            \item 若C君改了死性,終於肯溫習,他必然能獲得期望值或以上的分數,求他此時滿分的概率。
        \end{enumerate}
    \end{enumerate}
\end{document}